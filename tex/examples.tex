\documentclass[fleqn,reqno]{amsart}
\usepackage{../lib/radoslav-macro}
\usepackage{../lib/radoslav-more}
\usepackage{times}
\usepackage{natbib}

\begin{document}
%% STRIP END



\begin{paragraf*}
This chapter is the heart of the thesis.
It consists of examples highlighting the results of Chapter~\ref{ch:main-results}
and motivating the results of Chapters \ref{ch:fast-method} and \ref{ch:koszul-bpf}.

We recall that the Macaulay2 code for the examples is available at
\begin{center}
\url{http://www.math.cornell.edu/~rzlatev/phd-thesis/}
\end{center}
\end{paragraf*}



\begin{example}[$\mt{ex301}$]
\label{ex301}
Let $X=\PP^2_{s,t,u}$ and $J=\la tu,su,st,s^2+t^2+u^2\ra$.
Then $\phi$ is basepoint-free and generically 1-1.
The monic implicit equation is given by
\[
	P(\mathbf x)={x}_{0}^{2} {x}_{1}^{2}+{x}_{0}^{2} {x}_{2}^{2}+{x}_{1}^{2}
	{x}_{2}^{2}-{x}_{0} {x}_{1} {x}_{2} {x}_{3}
\]

Setting $\d=1$, we get
\[
	N=
	\bgroup\begin{bmatrix}0&
      {x}_{0}&
      {x}_{1} {x}_{2}\\
      {x}_{1}&
      0&
      {x}_{0} {x}_{2}\\
      {-{x}_{2}}&
      {-{x}_{2}}&
      {x}_{0} {x}_{1}-{x}_{2} {x}_{3}\\
      \end{bmatrix}\egroup
\]
whose determinant is just $P(\mathbf x)$.
The results of \citet{CGZ-00} apply and the matrix $N$ is a variant of the matrix
produced by the method of moving planes and quadrics.

Setting $\d=2$, we get
\[
	N=
	\bgroup\begin{bmatrix}{x}_{2}&
	      0&
	      0&
	      0&
	      {x}_{1}&
	      0&
	      0&
	      0&
	      {x}_{0}\\
	      {-{x}_{3}}&
	      0&
	      0&
	      {x}_{1}&
	      0&
	      0&
	      0&
	      {x}_{0}&
	      0\\
	      0&
	      0&
	      0&
	      {-{x}_{2}}&
	      {-{x}_{3}}&
	      {x}_{0}&
	      {x}_{2}&
	      0&
	      {-{x}_{2}}\\
	      {x}_{2}&
	      0&
	      {x}_{1}&
	      0&
	      0&
	      0&
	      {x}_{0}&
	      0&
	      0\\
	      0&
	      {x}_{1}&
	      {-{x}_{2}}&
	      0&
	      {x}_{2}&
	      0&
	      {-{x}_{3}}&
	      {-{x}_{2}}&
	      0\\
	      {x}_{2}&
	      {-{x}_{2}}&
	      0&
	      0&
	      {x}_{1}&
	      {-{x}_{2}}&
	      {x}_{0}&
	      0&
	      0\\
	      \end{bmatrix}\egroup
\]
which is a $6\times9$ matrix of linear forms.
This was expected --- the results of \cite{BJ-03} also apply and the method
of the approximation complex guarantees a matrix of linear forms.
Accordingly,
\[
	\gcd(\minors(6,N))=P
\]

Note that the claim that $\phi$ is of degree~$1$ follows, a fortiori, from the degree formula
\eqref{eq:degree-formula}. Indeed, we have
\[
	4\deg(\phi)=2^2-0
\]
We confirm this using Proposition \ref{prop:deg-GB} in Example \ref{ex314}.
\end{example}

\begin{example}[$\mt{ex302}$]
\label{ex302}
Clearly, if we replace $s,t,u$ in Example \ref{ex301} by general linear forms $L_0,L_1,L_2$,
effectively changing coordinates on the source, we get the same equation.
In this example we describe what happens if we take $X=\PP^1_{s,u}\times\PP^1_{t,v}$ instead
and let the $L_k$ be $(1,1)$-forms.
Since the algebraic structure of the coordinates is the same, so is the equation of the image,
\[
	Y=V({x}_{0}^{2} {x}_{1}^{2}+{x}_{0}^{2} {x}_{2}^{2}+{x}_{1}^{2}
	{x}_{2}^{2}-{x}_{0} {x}_{1} {x}_{2} {x}_{3})
\]
and $\phi$ is again basepoint-free, basically for the same reason ---
the $L_k$ are general.
However, the self-intersection of the divisor corresponding to the coordinates,
now $[\O(2,2)]$, is $8$.
It follows that $\phi$ is generically 2-1.

For $\d=(1,1)$ we get a square matrix of size $4$ with $h=(2,1,0,1)$.
As expected, up to a unit
\[
	\det(N)=P(\mathbf x)^2
\]

For $\d=(2,1)$ we get a square matrix of size $6$ with $h=(4,2)$, for which the
last equality again applies.
For $\d=(2,2)$ we get a $9\times12$-matrix with $h=(11,1)$ such that
\[
	\gcd(\minors(9,N))=P^2
\]
\end{example}

\begin{example}[$\mt{ex303}$]
\label{ex303}
Suppose that in the situation of Example \ref{ex302} we took the forms $L_k$ from
$\la st,sv,ut\ra$ instead.
Now $\phi$ has the unique basepoint $(0,1)\times(0,1)$ which is c.i. of degree $4$.
Indeed, on the affine open where $u=v=1$, the point looks like $V(st,s^2+t^2)$.
The equation of the image remains the same.
Once again, by \eqref{eq:degree-formula} we know that $\phi$ must be generically 1-1.

For $\d=(1,1)$ we get a $4\times5$-matrix.
Below is the matrix resulting from $(L_0,L_1,L_2)=(st,sv,ut)$,
\[
	N=
	\bgroup\begin{bmatrix}0&
      0&
      0&
      {x}_{0}&
      {x}_{1} {x}_{2}\\
      {x}_{1}&
      {x}_{0}&
      0&
      0&
      0\\
      {-{x}_{2}}&
      0&
      {x}_{0}&
      {-{x}_{2}}&
      {-{x}_{2} {x}_{3}}\\
      0&
      {-{x}_{2}}&
      {-{x}_{1}}&
      0&
      {x}_{1}^{2}+{x}_{2}^{2}\\
      \end{bmatrix}\egroup
\]
and, of course,
\[
	\gcd(\minors(4,N))=P
\]
\end{example}

\begin{example}[$\mt{ex304}$]
\label{ex304}
Let $X=\PP^2$ and $J=\la su^2,t^2(s+u),st(s+u),tu(s+u)\ra$.
Then $\phi$ is generically 1-1 with three basepoints --- $(1,0,0)$, $(0,1,0)$ and $(0,0,1)$,
all c.i. of degree 2, 3, and~1, respectively.

The implicit equation is given by
\[
	P(\mathbf x)={x}_{0} {x}_{1} {x}_{2}+{x}_{0} {x}_{1} {x}_{3}-{x}_{2} {x}_{3}^{2}
\]

As before, both the method of the moving planes and quadrics,
and the method of the approximation complex apply.
For $\d=1$, we get
\[
	N=
	\bgroup\begin{bmatrix}{-{x}_{3}}&
      0&
      {x}_{1}&
      {-{x}_{3}^{2}}\\
      0&
      {-{x}_{3}}&
      {-{x}_{2}}&
      {x}_{0} {x}_{2}+{x}_{0} {x}_{3}\\
      {x}_{2}&
      {x}_{1}&
      0&
      0\\
      \end{bmatrix}\egroup
\]
and for $\d=2$, we get a $6\times 9$-matrix whose entries are all linear.
\end{example}

\begin{example}[$\mt{ex305}$]
\label{ex305}
Let $X=\PP^2$ and $J=\la s^3,tu^2,s^2t+u^3,stu\ra$.
Then $\phi$ is generically 1-1 with a single c.i. basepoint of degree 2.
For $\d=1$ we have
\[
	N=
	\bgroup\begin{bmatrix}{x}_{1}&
      0&
      {-{x}_{3}^{2}}\\
      0&
      {x}_{1} {x}_{2}-{x}_{3}^{2}&
      {x}_{0} {x}_{1}\\
      {{x}_{3}}&
      {x}_{1}^{2}&
      0\\
      \end{bmatrix}\egroup
\]
and $\det(N)={x}_{0} {x}_{1}^{4}-{x}_{1} {x}_{2} {x}_{3}^{3}+{x}_{3}^{5}$,
the implicit equation.

Starting from $\d=2$, in which case we have
\[
	N=
	\bgroup\begin{bmatrix}0&
      0&
      0&
      {x}_{1}&
      {-{x}_{3}}&
      0&
      0\\
      {x}_{3}&
      0&
      {x}_{1}&
      0&
      0&
      {-{x}_{2}}&
      0\\
      0&
      {x}_{1}&
      0&
      {-{x}_{3}}&
      0&
      0&
      0\\
      0&
      0&
      0&
      0&
      0&
      {x}_{0}&
      -{x}_{1} {x}_{2}+{x}_{3}^{2}\\
      {-{x}_{2}}&
      0&
      {-{x}_{3}}&
      0&
      {x}_{0}&
      0&
      {x}_{1}^{2}\\
      {x}_{1}&
      {-{x}_{3}}&
      0&
      0&
      0&
      {x}_{3}&
      0\\
      \end{bmatrix}\egroup
\]
we always have $\mu-1$ linear columns and a single quadratic one.
\end{example}

\begin{example}[$\mt{ex306}$]
\label{ex306}
Let $X=\PP^2$ and $J=\la s^{3},t^{2} u,s^{2} t+u^{3},s t u\ra$.
This is the situation of Example~\ref{ex202}.
We have 
\[
	P(\mathbf x)={x}_{0}^{3} {x}_{1}^{4}-{x}_{0}^{2} {x}_{1}^{3} {x}_{2} {x}_{3}+{x}_{3}^{7}
\]

For $\d=1$, we find that $N$ is square of order $3$ with $h=(1,1,0,1)$.
For $\d=2$, we get a square $6\times6$ matrix with $h=(5,1)$,
and for $\d\geq3$, we get a non-square matrix of linear forms only.
\end{example}

\begin{example}[$\mt{ex307}$]
\label{ex307}
Consider the twisted cubic curve $C$.
It is the image of $X=\PP^1_{s,t}$ under the map
\[
	\phi=(s^3,s^2t,st^2,t^3):\PP^1\To\PP^3
\]
which is birational onto its image.
In light of Theorem \ref{thm:rad-minors}, we can carry out the same calculations
as before, even though the image is of codimension strictly bigger than $1$.

Setting $\d=1$, we get
\[
	N=
	\bgroup\begin{bmatrix}{-{x}_{3}}&
      {-{x}_{2}}&
      {-{x}_{1}}\\
      {x}_{2}&
      {x}_{1}&
      {x}_{0}\\
      \end{bmatrix}\egroup
\]
and
\[
	\minors(2,N)=\la {x}_{2}^{2}-{x}_{1} {x}_{3},{x}_{1} {x}_{2}-{x}_{0}
      {x}_{3},{x}_{1}^{2}-{x}_{0} {x}_{2}\ra
\]
which is the usual equation for $C$ in $\PP^3$. Setting $\d=2$, we get
\[
	N=
	\bgroup\begin{bmatrix}0&
      {-{x}_{3}}&
      {-{x}_{3}}&
      {-{x}_{2}}&
      {-{x}_{2}}&
      {-{x}_{1}}\\
      {-{x}_{3}}&
      {x}_{2}&
      0&
      {x}_{1}&
      0&
      {x}_{0}\\
      {x}_{2}&
      0&
      {x}_{1}&
      0&
      {x}_{0}&
      0\\
      \end{bmatrix}\egroup
\]
We have
\[
	\minors(3,N)=\Bigg\la
	\begin{cases}
	{x}_{2}^{2} {x}_{3}-{x}_{1} {x}_{3}^{2},
	{x}_{1} {x}_{2} {x}_{3}-{x}_{0}{x}_{3}^{2},
	{x}_{1}^{2} {x}_{3}-{x}_{0} {x}_{2} {x}_{3},\\
	{x}_{2}^{3}-{x}_{0}{x}_{3}^{2},
	{x}_{1} {x}_{2}^{2}-{x}_{0} {x}_{2} {x}_{3},
	{x}_{0}{x}_{2}^{2}-{x}_{0} {x}_{1} {x}_{3},\\
	{x}_{1}^{2} {x}_{2}-{x}_{0} {x}_{1}{x}_{3},
	{x}_{0} {x}_{1} {x}_{2}-{x}_{0}^{2} {x}_{3},
	{x}_{1}^{3}-{x}_{0}^{2}{x}_{3},{x}_{0} {x}_{1}^{2}-{x}_{0}^{2} {x}_{2}
	\end{cases}
	\Bigg\ra
\]
and
\[
	\rad(\minors(3,N))=
	\la {x}_{2}^{2}-{x}_{1} {x}_{3},{x}_{1} {x}_{2}-{x}_{0}
	      {x}_{3},{x}_{1}^{2}-{x}_{0} {x}_{2}\ra
\]
showing that the radical is necessary.
\end{example}

\begin{example}[$\mt{ex308}$]
\label{ex308}
Let $X=\PP^2_{s,t,u}$ and $J=\la s^5,t^5,su^4,st^2u^2\ra$.
Then $\phi$ is generically 2-1 map with the unique basepoint $(0,0,1)$
which is c.i. of degree $5$.
The image is defined by
\[
	P(\mathbf x)={x}_{0} {x}_{1}^{4} {x}_{2}^{5}-{x}_{3}^{10}
\]

Setting $\d=1$, we get
\[
	N=
	\bgroup\begin{bmatrix}{x}_{1} {x}_{2}&
      {-{x}_{3}^{8}}&
      0\\
      {-{x}_{3}^{2}}&
      {x}_{0} {x}_{1}^{3} {x}_{2}^{4}&
      0\\
      0&
      0&
      {x}_{0} {x}_{1}^{4} {x}_{2}^{5}-{x}_{3}^{10}\\
      \end{bmatrix}\egroup
\]
and
\[
	\det(N)=P(\mathbf x)^2
\]
One should note that the principal $2$-minor of $N$ is just $P(\mathbf x)$.

The alternative non-Gr\"obner bases approaches would require difficult computations;
the method of the approximation complex would need to find the $\gcd$ of the maximal
minors of a matrix of size $36\times58$, and none of the moving plane and quadrics
methods will work since the map isn't birational
(and (BP3) of \citet{BCD-03} fails in any case).

The practical gain of our method for this example, however, is arguable at best ---
we have to compute a degree $10$ syzygy which is already the degree of the implicit equation.
On the other hand, a better choice for $\d$ might help.
For $\d=3$ we get a $10\times10$-matrix with $h=(4,4,1,0,1)$,
and for $\d=4$ we get a $15\times16$-matrix with $h=(11,4,1)$.
\end{example}

\begin{example}[$\mt{ex309}$]
\label{ex309}
Let $N_1$ be the matrix of linear columns for $\d=1$ in Example \ref{ex304}.
We have that $\det(N_1)=0$. This shows that not all maximal minors need to be nonzero.

This has nothing to do with the fact that $N_1$ is special.
For another example, let us take $\d=4$ in Example \ref{ex308}.
Then $N$ is a $15\times16$-matrix whose columns correspond to the syzygies
\[
	\begin{cases}
	t^{2} u^{2} {x}_{2}-u^{4} {x}_{3},
	t^{3} u {x}_{2}-t u^{3} {x}_{3},
	s t^{2} u{x}_{2}-s u^{3} {x}_{3},
	t^{4} {x}_{2}-t^{2} u^{2} {x}_{3},\\
	s t^{3} {x}_{2}-s tu^{2} {x}_{3},
	s^{2} t^{2} {x}_{2}-s^{2} u^{2} {x}_{3},
	s u^{3} {x}_{1}-t^{3} u{x}_{3},
	s t u^{2} {x}_{1}-t^{4} {x}_{3},\\
	s^{2} u^{2} {x}_{1}-s t^{3}{x}_{3},
	u^{4} {x}_{0}-s^{4} {x}_{2},
	t^{2} u^{2} {x}_{0}-s^{4} {x}_{3},
	s^{2} t u {x}_{1} {x}_{2}-s t^{2} u {x}_{3}^{2},\\
	s^{3} u {x}_{1} {x}_{2}-s^{2} t u
      {x}_{3}^{2},s^{3} t {x}_{1} {x}_{2}-s^{2} t^{2} {x}_{3}^{2},
	  s^{4} {x}_{1}{x}_{2}-s^{3} t {x}_{3}^{2},
	  t u^{3} {x}_{0} {x}_{1} {x}_{2}-s^{3} u{x}_{3}^{3}
	\end{cases}
\]

Let $M$ be the square submatrix of the first 14 columns and the last column of $N$,
that is,
leaving out the column corresponding to the syzygy $s^{4} {x}_{1}{x}_{2}-s^{3} t {x}_{3}^{2}$.
Then $\det(M)=0$.
The same is true for the square submatrix consisting of the first 15 columns
but in that case $M=(N_1~|~N_2)$ which we wanted to avoid.
\end{example}

\begin{example}[$\mt{ex310}$]
\label{ex310}
The following example has been presented elsewhere in the literature as a difficult one to handle.
Let $X=\PP^2$ and take
\begin{align*}
	J=\la &-s^{2} t^{3}+3 s^{2} t^{2} u+s t^{3} u-4 s t^{2} u^{2}-s t
	      u^{3}+2 t^{2} u^{3}-t u^{4}+u^{5}, &\ra\\
		  &s^{2} t^{3}-3 s^{2} t^{2} u+s t^{3} u+3 s t
	      u^{3}-2 t^{2} u^{3}+t u^{4}-u^{5},\\
		  &s^{2} t^{3}-3 s^{2} t^{2} u-s t^{3} u+2
	      s^{2} t u^{2}+4 s t^{2} u^{2}-3 s t u^{3}-2 t^{2} u^{3}+3 t u^{4}-u^{5},\\
		  &-s^{2}
	      t^{3}+3 s^{2} t^{2} u-s t^{3} u-3 s t u^{3}+3 t u^{4}-u^{5}
\end{align*}
so $\phi$ is generically 1-1
with 3 base locus of total degree $17$ and multiplicity 20.
More precisely, the basepoints are the c.i. point $(1,1,1)$ of multiplicity $4$,
the a.c.i. point $(0,1,0)$ of degree 4 and multiplicity 5, and
the a.c.i. point $(1,0,0)$ of degree 9 and multiplicity 11.

Unlike other examples in this section,
already in degree $\d=1$,
we find only linear and quadratic syzygies.
In this sense the example is rather simple.
We find
\begin{align*}
	N=\Bigg[&\bgroup\begin{matrix}{x}_{0}+{x}_{1}&
      0&\ldots\\
      -{x}_{0}-{x}_{2}&
      3 {x}_{0} {x}_{1}-{x}_{1}^{2}+4 {x}_{1} {x}_{2}+3 {x}_{0} {x}_{3}-3 {x}_{1} {x}_{3}+4 {x}_{2} {x}_{3}-2 {x}_{3}^{2}&\ldots\\
      -{x}_{2}+{x}_{3}&
      {x}_{0}^{2}-{x}_{0} {x}_{1}-{x}_{1}^{2}-3 {x}_{0} {x}_{3}-3 {x}_{1} {x}_{3}-{x}_{3}^{2}&\ldots
      \end{matrix}\egroup\\
	  &\bgroup\begin{matrix}
	  \ldots&{x}_{1}^{2}-{x}_{3}^{2}\\
	  \ldots&{x}_{0}^{2}+7 {x}_{0} {x}_{1}-3 {x}_{1}^{2}+{x}_{0} {x}_{2}+10 {x}_{1} {x}_{2}+6 {x}_{0} {x}_{3}-9 {x}_{1} {x}_{3}+9 {x}_{2} {x}_{3}-6 {x}_{3}^{2}\\
	  \ldots&-5 {x}_{0} {x}_{1}-2 {x}_{1}^{2}-3 {x}_{0} {x}_{2}-6 {x}_{1} {x}_{2}-8 {x}_{0} {x}_{3}-5 {x}_{1} {x}_{3}-3 {x}_{2} {x}_{3}
	  \end{matrix}\egroup\Bigg]
\end{align*}
and
\[
	\det(N)=P(\mathbf x)
\]
\end{example}

\begin{example}[$\mt{ex311}$]
\label{ex311}
Let $\phi$ be the rational map from Example \ref{ex310}.
We compute the degree and multiplicity of the base locus $Z$.

Set $q_1=(1,0,0)$, $q_2=(0,1,0)$ and $q_3=(1,1,1)$, so that set-theoretically $Z=\{q_1,q_2,q_3\}$.
Since $J$ is saturated,
the sum of the degrees of the basepoints is just the degree of the base locus,
\[
	\deg(Z\se\PP^2)=\deg(J\se\CC[s,t,u])=17
\]

Because we are in projective space, this can be checked directly in Macaulay2.
However, since $Z$ is supported on multiple points,
we cannot compute the total multiplicity in the same way.
Indeed, running $\mt{multiplicity}(J)$ gives $45$, not the correct $20$.

Let $Q_k$ be the prime ideals corresponding to the points $q_k$, and set
\[
	J_k=J:(J:{Q_k}^\infty)
\]
The degree and multiplicity of $q_k$ can be computed from the ideal $J_k$ ---
those capture the local structure of $Z$ near $q_k$.
We have
\[
	\begin{cases}
	J_1=\la t u^{2},t^{3}-3 t^{2} u,u^{5}\ra\\
	J_2=\la s u,s^{2},u^{3}\ra\\
	J_3=\la t^{2}-2 t u+u^{2},s^{2}-2 s u+u^{2}\ra
	\end{cases}
\]
so in particular, $q_3$ is c.i., while $q_1$ and $q_2$ are a.c.i. points.
\end{example}

\begin{example}[$\mt{ex312}$]
\label{ex312}
Let $Y=V(P)$ for an irreducible form $P(\mathbf x)$ of degree $4$ on $\PP^3$.
Suppose further that $\text{Sing}(Y)$, the singular locus of $Y$,
contains 3 concurrent non-degenerate lines
(that is, passing trough a common point and spanning all of $\PP^3$).
Then $P(\mathbf x)$ is the determinant of a square order-$4$ matrix $M$ of linear forms.

We can prove this claim by a direct calculation.
After a linear change of coordinates on~$\PP^3$,
we can assume that the lines are the 3 coordinate axes in the distinguished $\{x_3\neq0\}$,
that is, the lines are given by $V(x_0,x_1)$, $V(x_1,x_2)$ and $V(x_0,x_2)$.

On the level of ideals, using Euler's identity, the assumption translates to
\[
	\la P_{x_0}(\mathbf x),\ldots, P_{x_3}(\mathbf x) \ra\se\la x_0x_1,x_0x_2,x_1x_2\ra
\]
so writing out $P(\mathbf x)=\sum_{|\alpha|=4} a_\alpha {\mathbf x}^\alpha$
and noting that each of the partials must be zero modulo $\la x_0x_1,x_0x_2,x_1x_2\ra$,
we only need to solve a linear system in the indeterminate coefficients.
We get
\[
	P(\mathbf x)=
	a_1x_0^2x_1^2+a_2x_0^2x_2^2+a_3x_1^2x_2^2+a_4x_0^2x_1x_2+a_5x_0x_1^2x_2+a_6x_0x_1x_2^2+a_7x_0x_1x_2x_3
\]
which is the determinant of
\[
	M=\begin{bmatrix}
	x_0&&&x_1\\
	&x_1&&x_2\\
	&&x_2&x_0\\
	-a_3x_2&-a_2x_0&-a_1x_1& a_4x_0+a_5x_1+a_6x_2+a_7x_3
	\end{bmatrix}
\]

In fact, if $a_7\neq0$ we can do better.
Setting $a_7=1$, a linear change of coordinates by the matrix
\[
\begin{bmatrix}
1&&&-a_4\\
&1&&-a_5\\
&&1&-a_6\\
&&&1
\end{bmatrix}
\]
leaves the singular locus the same but simplifies the general form to
\[
P'(\mathbf x)=a_1x_0^2x_1^2+a_2x_0^2x_2^2+a_3x_1^2x_2^2+x_0x_1x_2x_3
\]
and the matrix $M$ to
\[
M'=\begin{bmatrix}
x_0&&&x_1\\
&x_1&&x_2\\
&&x_2&x_0\\
-a_3x_2&-a_2x_0&-a_1x_1&x_3
\end{bmatrix}
\]
whose entries are scaled variables.
\end{example}

\begin{example}[$\mt{ex313}$]
\label{ex313}
In the situation of Example~\ref{ex201}, we had
\[
	N=\begin{bmatrix}0&
	       {x}_{1}&
	       0&
	       0&
	       {x}_{0} {x}_{2}-{x}_{3}^{2}\\
	       {x}_{2}&
	       {-{x}_{3}}&
	       {-{x}_{1}}&
	       {-{x}_{3}}&
	       {-{x}_{3}^{2}}\\
	       {-{x}_{3}}&
	       {-101 {x}_{1}}&
	       0&
	       {x}_{0}-101 {x}_{1}&
	       -10201 {x}_{1} {x}_{2}-202 {x}_{3}^{2}\\
	       -101 {x}_{2}-{x}_{3}&
	       0&
	       {x}_{0}&
	       0&
	       20402 {x}_{2} {x}_{3}-202 {x}_{3}^{2}\\
	       \end{bmatrix}
\]
where $N=N_1|N_2$ and $\det(N_1)=x_1P(\mathbf x)$.
Let $M$ be the submatrix corresponding to rows $\{2,3,4\}$ and columns $\{1,2,3\}$.
Then $\det(M)=P(\mathbf x)$ but $M$ is not technically a representation matrix~---
its rows cannot be indexed by the monomials of $S_\d$.

The situation of the previous example,~\ref{ex312}, is very similar.
While we were lucky and found a determinantal representation for $P(\mathbf x)$,
it does not come in the form of matrix of syzygies.
There is only $3$ linear syzygies, corresponding to there of the columns of $M$,
but the last one is not.
\end{example}

\begin{example}[$\mt{ex318}$]
\label{ex318}
Change to $\PP^1\times\PP^1$ source. This is not what you had before --- the map
must be of degree 1. Compute it.
\end{example}

\begin{example}[$\mt{ex319}$]
\label{ex319}
Let $X=\PPP$ and $J\se S=\CC[s,u;t,v]$ be given by 4 general $(2,2)$ forms in $\la s^2,st,t^2\ra$.
Then $\phi$ is generically 1-1 with
the unique basepoint $(0,1)\times(0,1)$ of degree $3$ and multiplicity $4$.
After a linear change of coordinates on the target,
we can assume that the $\phi_j$ are of the form outlined in Example~\ref{ex302},
so the representation of Example~\ref{ex312} applies.
\end{example}

\begin{example}[$\mt{ex314}$]
\label{ex314}
We apply Proposition \ref{prop:deg-GB} to compute the degree of the rational map
in Example \ref{ex301}.
We have
\[
	I=\Bigg\la
	\begin{cases}
		{x}_{1} t-{x}_{2} u,\\{x}_{0} s-{x}_{2} u,\\{x}_{2} s^{2}-{x}_{3} s t+{x}_{2}
	     t^{2}+{x}_{2} u^{2},\\{x}_{1} s^{2}-{x}_{3} s u+{x}_{2} t u+{x}_{1} u^{2},\\{x}_{0}
	     t^{2}+{x}_{2} s u-{x}_{3} t u+{x}_{0} u^{2},\\{x}_{1} {x}_{2} s+{x}_{0} {x}_{2}
	     t+({x}_{0} {x}_{1}-{x}_{2} {x}_{3}) u,\\{x}_{0}^{2} {x}_{1}^{2}+{x}_{0}^{2}
	     {x}_{2}^{2}+{x}_{1}^{2} {x}_{2}^{2}-{x}_{0} {x}_{1} {x}_{2} {x}_{3}
	\end{cases}\Bigg\ra
\]
and we see $P(\mathbf x)$ as the last generator.
Next, we calculate a Gr\"obner basis with respect to a product monomial order where
the $s,t,u$ variables come before the $x_j$ variables.
Dropping the implicit equation, we get
\[
	\begin{cases}
		{x}_{1} t-{x}_{2} u,\\{x}_{0}^{2}
	      {x}_{2} t+({x}_{0}^{2} {x}_{1}+{x}_{1} {x}_{2}^{2}-{x}_{0} {x}_{2} {x}_{3})
	      u,\\{x}_{0} s-{x}_{2} u,\\{x}_{1} {x}_{2} s+{x}_{0} {x}_{2} t+({x}_{0}
	      {x}_{1}-{x}_{2} {x}_{3}) u,\\{x}_{0} t^{2}+{x}_{2} s u-{x}_{3} t u+{x}_{0}
	      u^{2},\\{x}_{2} s^{2}-{x}_{3} s t+{x}_{2} t^{2}+{x}_{2} u^{2},{x}_{1}
	      s^{2}-{x}_{3} s u+{x}_{2} t u+{x}_{1} u^{2}
	\end{cases}
\]
Collecting the $S$-part of the leading terms, we get the ideal $\la s,t\ra$ whose degree
is obviously $1$.

We can carry out the same calculation for the map in Example \ref{ex308}.
The Rees ideal and its Gr\"obner basis are as follow,
\[
	I=\Bigg\la\begin{cases}
	{x}_{2} t^{2}-{x}_{3} u^{2},\\{x}_{1} {x}_{2} s-{x}_{3}^{2} t,\\-{x}_{3}
	     t^{3}+{x}_{1} s u^{2},\\-{x}_{2} s^{4}+{x}_{0} u^{4},\\-{x}_{3} s^{4}+{x}_{0} t^{2}
	     u^{2},\\-{x}_{1} s^{5}+{x}_{0} t^{5},\\-{x}_{3}^{3} s^{3}+{x}_{0} {x}_{1} {x}_{2} t
	     u^{2},\\-{x}_{3}^{5} s^{2}+{x}_{0} {x}_{1}^{2} {x}_{2}^{2} u^{2},\\-{x}_{3}^{8}
	     s+{x}_{0} {x}_{1}^{3} {x}_{2}^{4} t,\\{x}_{0} {x}_{1}^{4}
	     {x}_{2}^{5}-{x}_{3}^{10}
	\end{cases}\Bigg\ra,
	\quad
	\begin{cases}
		{x}_{1} {x}_{2} s-{x}_{3}^{2}
		      t,\\{x}_{3}^{8} s-{x}_{0} {x}_{1}^{3} {x}_{2}^{4} t,\\{x}_{2} t^{2}-{x}_{3}
		      u^{2},\\{x}_{3}^{9} t^{2}-{x}_{0} {x}_{1}^{4} {x}_{2}^{4} u^{2},\\{x}_{3}^{7} s
		      t-{x}_{0} {x}_{1}^{3} {x}_{2}^{3} u^{2},\\{x}_{3}^{5} s^{2}-{x}_{0} {x}_{1}^{2}
		      {x}_{2}^{2} u^{2},\\{x}_{3} t^{3}-{x}_{1} s u^{2},\\{x}_{3}^{3} s^{3}-{x}_{0}
		      {x}_{1} {x}_{2} t u^{2},\\{x}_{3}^{4} s^{2} t^{2}-{x}_{0} {x}_{1}^{2} {x}_{2}
		      u^{4},\\{x}_{3}^{2} s^{3} t-{x}_{0} {x}_{1} u^{4},\\{x}_{3} s^{4}-{x}_{0} t^{2}
		      u^{2},\\{x}_{2} s^{4}-{x}_{0} u^{4},\\{x}_{1} s^{5}-{x}_{0} t^{5}
	\end{cases}
\]
This time the ideal of the $S$-part of the leading terms is $\la s,t^2\ra$ which is of degree 2.
\end{example}

\begin{example}[$\mt{ex315}$]
\label{ex315}
While Examples \ref{ex301}, \ref{ex306} and \ref{ex307} support Conjecture \ref{conj:sing-locus},
we point out that the latter is stated in its strongest possible form.
For one, the claim is trivial in the case $\deg(\phi)>1$
for then the comaximal minors vanish on all of $Y$ again.
This can be seen in Example \ref{ex308}.

On the other hand, while it is tempting to conjecture that
\[
	\rad(\la P_{x_j}:j\ra)\se\sat(\minors(r-1,N))
\]
that is, that the inclusion of the conjecture is on the level of schemes,
this is not true as illustrated by Example \ref{ex304}.
In the case $\d=1$, we get
\begin{align*}
	\rad(\la P_{x_j}:j\ra)&=\la {x}_{3},{x}_{1} {x}_{2},{x}_{0} {x}_{2},{x}_{0} {x}_{1}\ra\\
	\sat(\minors(r-1,N))&=\la{x}_{3}^{2},{x}_{2} {x}_{3},{x}_{1} {x}_{3},{x}_{2}^{2},{x}_{1}
      {x}_{2},{x}_{1}^{2}\ra\\
	\rad(\minors(r-1,N))&=\la {x}_{1},{x}_{2},{x}_{3}\ra
\end{align*}
and the only inclusion we have is
\[
	\rad(\la P_{x_j}:j\ra)\se\rad(\minors(r-1,N))
\]
The results for other values of $\d$ are analogous.
\end{example}

\begin{example}[316]
\label{ex316}
The construction of Lemma \ref{lemma:lengthRees} is already apparent in Example \ref{ex207}.
Since $P\se I$ in $R$, the bihomogeneous primes in $R/I=B$
which pull back to $P$ are those in
\[
	(\CC[x_0,x_1,x_2]/\la x_0+x_1-x_2\ra)[s,t]/\la x_1s^2-x_0t^2\ra
\]
which intersect the coefficient ring trivially, so in turn, those in
\[
	(\CC[x_0,x_1,x_2]\setminus\la x_0+x_1-x_2\ra)\inv(\CC[x_0,x_1,x_2]/\la x_0+x_1-x_2\ra)[s,t]/\la x_1s^2-x_0t^2\ra
\]
\end{example}

\begin{example}[317]
\label{ex317}
It may seem at first that taking the smallest possible degree is always our best option,
but Example~\ref{ex308} provides a good counterexample.
The examples in Chapter~\ref{ch:fast-method}, which are computationally more complex,
often have the first few $N_i$ being zero, for low $\d$,
so nonzero matrices tend to be larger in size and degree of their minors.

It is precisely this flexibility to choose a good $\d$,
that is one of the contributions of
our results to the methods of moving hypersurfaces.
\end{example}



%% STRIP BEGIN
%% BIBLIOGRAPHY
\bibliographystyle{unsrtnat}
\bibliography{../lib/refs}

\end{document}