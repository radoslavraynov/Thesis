\documentclass[fleqn,reqno]{amsart}
\usepackage{../lib/radoslav-macro}
\usepackage{../lib/radoslav-more}
\usepackage{times}
\usepackage{natbib}


\newcounter{chapter}
\setcounter{chapter}{2}
\numberwithin{first}{chapter}


\begin{document}
%% STRIP END



\begin{example}[$\mt{ex210}$]
\label{ex210}
Let $X=\PP^2$ and $J=\la su^2,t^2(s+u),st(s+u),tu(s+u)\ra$. Then $\phi$ is birational onto its
image; the basepoints are $(1,0,0)$, $(0,1,0)$ and $(0,0,1)$,
all ci of multiplicities 2, 3, and 1, respectively. For $d=1$ we get
\[
N=\bgroup\begin{bmatrix}{-{x}_{3}}&
0&
{x}_{1}&
{-{x}_{3}^{2}}\\
0&
{-{x}_{3}}&
{-{x}_{2}}&
{x}_{0} {x}_{2}+{x}_{0} {x}_{3}\\
{x}_{2}&
{x}_{1}&
0&
0\\
\end{bmatrix}\egroup
\]
and for $d=2$ we get a $6\times 9$-matrix whose columns are all linear.

The matrix $N$ above also give a small example where the containment
\[
\la\text{Jac}~P\ra\se\mt{minors}(r-1,N)
\]
fails, i.e. the radical on the RHS is necessary, and further the containment of the
radicals is proper.
\end{example}

\begin{example}[$\mt{ex211}$]
Let $X=\PP^2$ and $J=\la s^3,t^2u,s^2t+u^3,stu\ra$. The map is birational
with a single ci basept of multiplicity 2. For $d=1$ we have
\label{ex211}
\[
	N=\bgroup\begin{bmatrix}{x}_{1}&
  {-{x}_{3}^{2}}&
  0\\
  {-{x}_{3}}&
  0&
  {x}_{0}^{2} {x}_{1}^{2}-{x}_{0} {x}_{1} {x}_{2} {x}_{3}\\
  0&
  {x}_{0} {x}_{1}&
  {x}_{3}^{4}\\
  \end{bmatrix}\egroup
\]
and $\det(N)={x}_{0}^{3} {x}_{1}^{4}-{x}_{0}^{2} {x}_{1}^{3} {x}_{2} {x}_{3}+{x}_{3}^{7}$,
the implicit equation.

For $d=2$ we get a square matrix of order $6$ with linear entries only.
\end{example}

\begin{example}[$\mt{ex213}$]
\label{ex213}
This example has been present elsewhere in the literature and is know to break all the available methods.
Let $X=\PP^2$ and take
\begin{align*}
	J=\la &-s^{2} t^{3}+3 s^{2} t^{2} u+s t^{3} u-4 s t^{2} u^{2}-s t
	      u^{3}+2 t^{2} u^{3}-t u^{4}+u^{5}, &\ra\\
		  &s^{2} t^{3}-3 s^{2} t^{2} u+s t^{3} u+3 s t
	      u^{3}-2 t^{2} u^{3}+t u^{4}-u^{5},\\
		  &s^{2} t^{3}-3 s^{2} t^{2} u-s t^{3} u+2
	      s^{2} t u^{2}+4 s t^{2} u^{2}-3 s t u^{3}-2 t^{2} u^{3}+3 t u^{4}-u^{5},\\
		  &-s^{2}
	      t^{3}+3 s^{2} t^{2} u-s t^{3} u-3 s t u^{3}+3 t u^{4}-u^{5}
\end{align*}
so $\phi$ is generically 1-1 with 3 basepoints of total degree $17$
and multiplicity 20. More precisely, the basepoints are the ci point $(1,1,1)$ of multiplicity $4$,
the aci point $(0,1,0)$ of degree 4 and multiplicity 5, and
the aci point $(1,0,0)$ of degree 9 and multiplicity 11.

Irrelevant of any of the hineous basepoints, we get
\[
	N=\bgroup\begin{bmatrix}{x}_{0}+{x}_{1}&
      0&
      {x}_{1}^{2}-{x}_{3}^{2}\\
      -{x}_{0}-{x}_{2}&
      3 {x}_{0} {x}_{1}-{x}_{1}^{2}+4 {x}_{1} {x}_{2}+3 {x}_{0} {x}_{3}-3 {x}_{1} {x}_{3}+4 {x}_{2} {x}_{3}-2 {x}_{3}^{2}&
      {x}_{0}^{2}+7 {x}_{0} {x}_{1}-3 {x}_{1}^{2}+{x}_{0} {x}_{2}+10 {x}_{1} {x}_{2}+6 {x}_{0} {x}_{3}-9 {x}_{1} {x}_{3}+9 {x}_{2} {x}_{3}-6 {x}_{3}^{2}\\
      -{x}_{2}+{x}_{3}&
      {x}_{0}^{2}-{x}_{0} {x}_{1}-{x}_{1}^{2}-3 {x}_{0} {x}_{3}-3 {x}_{1} {x}_{3}-{x}_{3}^{2}&
      -5 {x}_{0} {x}_{1}-2 {x}_{1}^{2}-3 {x}_{0} {x}_{2}-6 {x}_{1} {x}_{2}-8 {x}_{0} {x}_{3}-5 {x}_{1} {x}_{3}-3 {x}_{2} {x}_{3}\\
      \end{bmatrix}\egroup
\]
and
\[
	\det(N)=P(\mathbf x) \demo
\]
\end{example}



%% STRIP BEGIN
%% BIBLIOGRAPHY
\bibliographystyle{unsrtnat}
\bibliography{../lib/refs}

\end{document}
