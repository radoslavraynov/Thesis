\documentclass[fleqn,reqno]{amsart}
\usepackage{../lib/radoslav-macro}
\usepackage{../lib/radoslav-more}
\usepackage{times}
\usepackage{natbib}

\begin{document}
%% STRIP END


\section{Template Proofs}

\begin{paragraf}
\label{par:template-proof}
Results on methods of moving surfaces, at last so far, have focused on
moving planes and moving quadrics.
In those cases, matrices equivalent to our $\phi^{(i)}$ and $\Phi^{(i)}$ ($i=1,2$)
from Section~\ref{sec:strands} have been employed.
Most of the proofs follow the following outline
\begin{enumerate}
\item
Consider a specific $X$ and set some conditions on the base locus of allowed $\phi$,
e.g. empty or 0-dimensional and locally complete intersection;
\item
Choose the degree $\d$ as a fixed function of $\e$;
\item
Use matrices columns from $\phi^{(i)}$ and $\Phi^{(i)}$ ($i=1,2$) to construct a matrix a
square $M$;
\item
Show that $\det(M)$ vanishes on $P$ or $P^{\deg\phi}$;
\item
Show that $M$ is nonsingular.
\end{enumerate}

The conditions on the base locus in (1) and (2) are dictated by
making some numerics in (3) workout.
Step (4) is often automatic, especially if $\phi$ is assumed to be generically 1-1.
Step (5), on the other hand, usually involves a cleverly chosen normal form for the syzygies,
allowing the authors to assert that certain monomial on the $x_j$ shows up nontrivially
in the determinant of $M$.
\end{paragraf}

\begin{paragraf}
From the standing point of this paper, those proofs most often entail
figuring out the $h$-vector, \eqref{par:N-matrix}, of $N$.
This is difficult.
For instance, in \citet{BCD-03},
five conditions on the base locus need to be enforced.
\end{paragraf}

\begin{paragraf*}
However, in the case of base-point free maps, some of the work can be avoided.
Next result is a direct extension to Proposition 2.2 of \citet{CGZ-00}
(see also the discussion of \citet{Cox-01}).
The result there is in the case $X=\PPP$, $k=1$ and $\e=(p,q)$, $\d=(p-1,q-1)$
but the generalization is immediate.
\end{paragraf*}

\begin{proposition}
\label{prop:koszul-syzygies}
Consider the setup of \eqref{par:again} with $\dim(X)=2$.
Assume \itmref{itm:conds:1}{par:conds}.
Suppose that $\phi_0,\phi_1,\phi_2$ are forms of degree $\e$ on $S$ with no common zero locus.
Let $\d$ be any degree such that
\[
	H^1(X,\O_X(\d-2\e))=0
\]
One has that the syzygies on
\[
	M_0=\Phi^{(k)}
\]
over degree $\d$ are Koszul, i.e. are of the form $M_1{\mathbf u}$,
where $\mathbf u\in S_{\e-\d}$
and $M_1$ is the first differential in a resolution of $M_0$
when the $\phi_j$ are thought as variables.
\end{proposition}



\section{Basepoint-Free Maps over $X=\PPP$}
\label{sec:bpf-maps-ppp}

\begin{paragraf*}
In this section we present a proof of Corollary~\ref{cor:moving-quadrics}.
To this end, we assume throughout the section that $X=\PPP$,
$\phi$ is basepoint-free given by sections in degree $\e=(p,q)$
and that $\d=(p-1,q-1)$ is fixed.
Further, we assume that there are no linear syzygies (for the fixed $\d$),
that is, $N_1=0$.
\end{paragraf*}

\begin{paragraph}
\label{par:count-rels}
Let $k$ be a positive integer.
The matrix from \eqref{par:all-phi-k} becomes
\[
	\Phi^{(k)}:S_{p-1,q-1}^{\binom{k+3}{3}}\To S_{(k+1)p-1,(k+1)q-1}^1
\]
that is, of size $(k+1)^2pq\times\binom{k+3}{3}pq$.
But then
\begin{align}
	\dim_{\CC}\ker(\Phi^{(k)})\geq\binom{k+1}{3}pq\label{eqn1}
\end{align}
with equality if and only if $\Phi^{(k)}$ is of maximal rank~--- $(k+1)^2pq$.
\end{paragraph}

\begin{paragraph}
Recall that the columns of $\Phi^{(k)}$ are indexed
by the monomials in $R_{p-1,q-1,k}$
and the elements of the kernel give rise to degree-$k$ syzygies over $S_\d$.
Write such a syzygy as
\begin{align}\label{ebasi}
	\sum_{|\alpha|=k} A_\alpha\cdot
	x_0^{\alpha_0}x_1^{\alpha_1}x_2^{\alpha_2}+
	\big(\sum_{|j|=k-1} B_{j}\cdot
	x_0^{j_0}x_1^{j_1}x_2^{j_2}\big)x_3+
	\big(\sum_{|\gamma|=k-2}C_\gamma x^\gamma\big)x_3^2
\end{align}
where the first two sums correspond to
the columns indexed by monomials involving $x_3$ at most linearly.
Their number is $\binom{k+2}{2}pq+\binom{k+1}{2}pq=(k+1)^2pq$,
which is also the number of rows.
Let $\Psi^{(k)}$ be the square submatrix of $\Phi^{(k)}$ formed by those columns.
If the former is nonsingular, the latter is of maximal rank.
\end{paragraph}

\begin{lemma}
\label{lemma:Psi-nonsing}
The matrix $\Psi^{(k)}$ is nonsingular for every positive $k$.
\end{lemma}

\begin{proof}
If $k=1$, then $\Psi^{(1)}$ is nonsingular by assumption.
It it were not, a kernel element would give rise to nonzero $N_1$.

The case $k>1$ is a direct consequence of the case $k=1$
and the Proposition~\ref{prop:koszul-syzygies}.
Let
\[
	V=\begin{bmatrix}{V'}\\{V''}\end{bmatrix}
\]
be a kernel element of $\Psi^{(k)}$ with $V'$ a column-$\binom{k+2}{2}$-vector over $\CC$
and $V''$ is a column-$\binom{k+1}{2}$-vector over $\CC$, corresponding to the coefficients
not involving $x_3$ and the coefficients involving $x_3$, respectively.
We can multiply out to get a $k$-syzygy on the $\phi_j$ over $S_{p-1,q-1}$, i.e.
\[
	\sum_{|\alpha|=k} A_\alpha\cdot
	\phi_0^{\alpha_0}\phi_1^{\alpha_1}\phi_2^{\alpha_2}+
	\big(\sum_{|j|=k-1} B_{j}\cdot
	\phi_0^{j_0}\phi_1^{j_1}\phi_2^{j_2}\big)\phi_3=0
\]
as an equality in $S$, where the coefficients $A_\alpha, B_j$ are in $S_{p-1,q-1}$.
We rewrite the above as
\[
	\sum_{|j|=k-1}\big(
	C_{0,j}\phi_0+C_{1,j}\phi_1+C_{2,j}\phi_2+C_{3,j}\phi_3\big)
	\phi_0^{j_0}\phi_1^{j_1}\phi_2^{j_2}=0
\]
which is again an equality in $S$, and again the coefficients $C_{j,j}$ are in $S_{p-1,q-1}$.
Note also that $C_{3,j}=B_j$.

By Proposition~\ref{prop:koszul-syzygies}
we can rewrite the coefficients as linear combinations of their Koszul syzygies,
i.e. for each $j$ we get an equality in $S$ of the form
\[
	C_{0,j}\phi_0+C_{1,j}\phi_1+C_{2,j}\phi_2+C_{3,j}\phi_3=
	D_{0,j}\phi_0+D_{1,j}\phi_1+D_{2,j}\phi_2
\]
where $D_0,D_1,D_2$ are in $S_{p-1,q-1}$.
By the assumption on the independence of the $\phi_j$ over $S_{p-1,q-1}$,
we must have that $C_{3,j}=0$.
Equivalently, $B_j=0$ for every $j$, so $V''=0$.

To conclude the proof, note further that any nontrivial syzygy
of the monomials in $\phi_0,\phi_1,\phi_2$
in any degree $k\geq1$ must be of degree degree at least $(p,q)$,
i.e. must be a combination of Koszul syzygies again.
Since $\deg(A_\alpha)=(p-1,q-1)$, it follows that $V'=0$ as well.
\end{proof}

\begin{paragraph}
Paragraph \ref{par:count-rels} and Lemma \ref{lemma:Psi-nonsing} show that there are exactly
$\binom{k+1}{3}pq$ linearly independent syzygies of degree $k$, so $pq$ quadratic syzygies.
Clearly degree-$(k-2)$ $T$-combinations of quadratic syzygies introduce degree-$k$ syzygies.
Next lemma shows that all syzygies of degree $k>2$ arise in this way.
\end{paragraph}

\begin{lemma}
\label{another-lemma}
Any degree $k>2$ syzygy of degree $d=(p-1,q-1)$ on the source is a degree-$(k-2)$
$T$-combination of quadratic syzygies.
\end{lemma}

\begin{proof}
Since $\Psi^{(k)}$ is square and of maximal rank by Lemma \ref{lemma:Psi-nonsing},
it is nonsingular, so $\Phi^{(k)}$ is of maximal rank, $(k+1)^3pq$.
This means that there are $pq$ linearly independent quadratic syzygies
and by the same lemma any one of them must involve $x_3^2$ nontrivially.
It follows that, up to scaling, any nonzero quadratic syzygy is of the form
\[
	g_\nu(\mathbf s,\mathbf x)=\ldots+{\mathbf{s}}^\nu x_3
\]
where the $\mathbf{s}^\nu$ are the monomials of degree $(p-1,q-1)$ on the source.

We can rewrite an arbitrary degree-$(k>2)$ syzygy as
\[
	g(\mathbf s, \mathbf x)=\sum_{|\alpha|=k} A'_\alpha\cdot
	x_0^{\alpha_0}x_1^{\alpha_1}x_2^{\alpha_2}+
	\big(\sum_{|j|=k-1} B'_{j}\cdot
	x_0^{j_0}x_1^{j_1}x_2^{j_2}\big)x_3+
	\big(\sum_\nu \mathbf{s}^\nu f_\nu\big)x_3^2
\]
where each $f_\nu$ is a form degree $k-2$ in $T$, so the syzygy
\[
	g(\mathbf s, \mathbf x)-\sum_\nu f_\nu\cdot g_\nu(\mathbf s, \mathbf x)
\]
has trivial $x_3^k$ part in the sense of the discussion above.
By Lemma \ref{lemma:Psi-nonsing} again, this syzygy must be zero,
so every syzygy of degree $k$ is a $T$-combination of quadratic syzygyes.
\end{proof}

\begin{proof}[\titlestyle Proof of Theorem \ref{thm:rel-moving-quadrics}]
	Since there are no syzygies in degree $1$ and all syzygyes in degree $k>2$ are $T$-combinations
	of quadratic syzygyes by Lemma \ref{another-lemma},
	we have $N=N_2$. Since the number of linearly independent syzygies
	of degree $2$ is $pq$, the matrix $N_2$ is square.
	
	To finish the prove we note that necessarily $\reg(B_P)\leq(p-1,q-1)$ since the ideal of points
	$I_P$ is generated in degree $(p,q)$ over $K(T/P)$,
	so Theorem \ref{thm:gcd-minors} applies.
	
	Finally, birationallity follows by comparing the degree on both hand-sides of Theorem \ref{thm:gcd-minors},
	\[
		\deg(Y)\deg(\phi)=\deg\det (N)=\deg\det(N_2)=2pq
	\]
	and $\deg(Y)=2pq$ since the parametrization is basepoint-free, so $\deg\phi=1$.
\end{proof}



\section{Basepoint-Free Maps over $X=\PP^2$}
\label{sec:bpf-maps-pp2}

\begin{lemma}
	\label{lemma:rel-moving-planes-quadrics-min-dims}
	For every $k>0$, the number of linearly independent degree-$k$ syzygies over $S_{d}$,
	$d=p-1$, is at least
	\[
		\frac{k(k+1)p}{12}(kp-p+k+5)
	\]
\end{lemma}

\begin{proof}
	As before, we consider the $S$-linear map
	\[
		\begin{bmatrix}\phi_0^k& \phi_0^{k-1}\phi_1& \cdots& \phi_3^k\end{bmatrix}:
		S^{\binom{k+3}{3}}\To S^1
	\]
	and construct its $\CC$-linearization,
	exactly as we did for $\Phi^{(k)}$ in Section \ref{sec:rel-moving-quadrics}.
	It is a map of complex vector spaces
	\[
		S_{p-1}^{\binom{k+3}{3}}\To S_{(k+1)p-1}^1
	\]
	so a matrix of size $\binom{kp+p+1}{2}\times\binom{k+3}{3}\binom{p+1}{2}$.
	Since there are at least as many columns as rows,
	the dimension of the kernel is at least
	\[
		\binom{k+3}{3}\binom{p+1}{2}-\binom{kp+p+1}{2}=\frac{k(k+1)p}{12}(k+5+kp-p)
	\]
	establishing the claim.
\end{proof}

\begin{remark}
	\label{rem:rel-moving-planes-quadrics-num-syz}
	By Lemma \ref{lemma:rel-moving-planes-quadrics-min-dims} there are
	at least $p$ linear syzygies and at least $p(p+7)/2$ quadratic syzygies.
\end{remark}

\begin{paragraph*}
	The main result of this section states that the matrices which our method constructs
	are the same, up to formatting, as those constructed by the method of \citet{00-CGZ-JSC}.
\end{paragraph*}

\begin{theorem}
	\label{thm:rel-moving-planes-quadrics}
	Let $X=\PP^2$, $\phi$ be basepoint-free whose entries are of degree $p$,
	and suppose there are exactly $p$ linear syzyges for $d=p-1$, i.e. the minimal possible number.
	Then $N=N_1|N_2$, $N$ is square and $\phi$ is birational. In particular,
	\[
		\det(N)=\det(N_1|N_2)=P(\mathbf x)
	\]
\end{theorem}

\begin{lemma}
	\label{lemma:rel-moving-planes-quadrics-main}
	Suppose that there are exactly $p$ linearly independent linear syzygies, i.e. the minimal possible number.
	Then the $4p$ quadratic syzygies of the form $L_i(\mathbf s,\mathbf x)x_j$ for the $p$ linear syzygies $L_i$
	are linearly independent. The number of linearly independent quadratic syzygies not emerging in this way
	is $\binom{p}{2}$. The vector subspace of $S_{p-1}$ spanned by the coefficient of $x_3^2$ among all
	quadratic syzygies is all of $S_{p-1}$.
\end{lemma}

\begin{proof}
	Since $\phi_0,\phi_1,\phi_2$ form a regular sequence in degree $p$,
	a nonzero linear syzygy over $S_{p-1}$ must involve $x_3$ nontrivially.
	Let $V_1$ be the linear subspace of $S_{p-1}$ spanned by the
	coefficient of $x_3$ among all linear syzygies, i.e.
	\[
		V_1=\Span\{a_3:a_0x_1+\ldots+a_3x_3 \text{ is a linear syzygy}\}
	\]
	(and note that the {\em span} keyword isn't necessary).
	Since there are exactly $p$ of those, by the observation just made, $\dim_\CC V_1=p$.
	
	Any linear syzygy $L(\mathbf s, \mathbf x)$ gives rise to a quadratic syzygy of the form
	$L(\mathbf s, \mathbf x)x_j$ for each $j$.
	Let $g(\mathbf s,\mathbf x)$ be a quadratic syzygies not arising in this way.
	We know that $g$ must involve $x_3^2$ nontrivially,
	for else, it is of the form
	\[
		\sum_{j=0,1,2} (B_{j,0}x_0+B_{j,1}x_1+B_{j,2}x_2+B_{j,3}x_3) x_j
	\]
	and since $\phi_0,\phi_1,\phi_2$ form a regular sequence in degree $p$,
	by the Koszul theorem (e.g. \ref{prop:key-koszul}),
	we have
	\[
		B_{j,0}x_0+B_{j,1}x_1+B_{j,2}x_2+B_{j,3}x_3=C_{j,0}x_0+C_{j,1}x_1+C_{j,2}x_2
	\]
	In particular,
	\[
		L_j(\mathbf s,\mathbf x)=B_{j,0}x_0+B_{j,1}x_1+B_{j,2}x_2+B_{j,3}x_3-C_{j,0}x_0-C_{j,1}x_1-C_{j,2}x_2
	\]
	is a linear syzygies on the $\phi_j$.
	But then $g-\sum_j L_jx_j$ is a quadratic syzygy only involving $x_0,x_1,x_2$,
	and so must be $0$, contradicting the assumption that $g$ was not $T$-generated by linear syzygies.
	
	Let $V_2$ be the linear subspace of $S_{p-1}$ spanned by the coefficient of $x_3^2$ for quadratic
	sygyzygies $g$ as in the previous paragraph. The same argument shows that we cannot have $V_1\cap V_2\neq0$.
	
	We finish the proof by an easy dimension count.
	The discussion so far gives us that the number of {\em new} quadratic syzygies is at most
	$\dim_\CC V_2\leq\binom{p+1}{2}-p$,
	so even if all the $4p$ quadratic syzygies generated by the linear syzygies
	and the new quadratic syzygies are linearly independent altogether,
	we get at most $4p+\binom{p+1}{2}-p=p(p+7)/2$-many of them.
	On the other hand,
	the number of linearly independent quadratic syzygies must be at least $p(p+7)/2$
	by Remark \ref{rem:rel-moving-planes-quadrics-num-syz}.
	
	It follows that there are $\binom{p+1}{2}-p=\binom{p}{2}$ {\em new} quadratic syzygies,
	which along with the $4p$ pushed linear syzygies are linearly independent altogether.
	Also, the linear span of the $S_{p-1}$ coefficient of $x_3^2$ among the quadratic syzygies
	is $V_1\oplus V_2=S_{p-1}$.
\end{proof}

\begin{lemma}
	\label{lemma:rel-moving-planes-quadrics-second}
	Let $g(\mathbf s, \mathbf x)$ be a syzygy of degree $k>2$.
	Then $g$ is a $T$-combination of linear and quadratic syzygies.
\end{lemma}

\begin{proof}
	The proof uses the same arguments as in the proof of Lemma \ref{lemma:rel-moving-planes-quadrics-main}.
	Since the $S_{p-1}$-coefficients of $x_3^2$ among the quadratic syzygies span $S_{p-1}$,
	we can find among them
	\[
		g_\nu(\mathbf s,\mathbf x)=(\text{terms involving $x_3$ at most linearly})+{\mathbf s}^\nu x_3^2
	\]
	for all $|\nu|=p-1$. We can then rewrite
	\[
		g(\mathbf s,\mathbf x)=(\text{terms involving $x_3$ at most linearly})+(\sum_\nu {\mathbf s}^\nu\cdot h_\nu(\mathbf x))x_3^2
	\]
	where $h_v(\mathbf x)$ is a degree-$(k-2)$ homogeneous polynomial. But now
	\[
		g-\sum_\nu h_\nu g_\nu=\sum_{|\beta|=k-1} (A_{\beta,0}x_0+A_{\beta,1}x_1+A_{\beta,2}x_2+A_{\beta,3}x_3) x_0^{\beta_0}x_1^{\beta_1}x_2^{\beta_2}
	\]
	where $A_{\beta,j}$ is a $(p-1)$-form in $S$.
	Lemma \ref{prop:key-koszul} and the same argument as in the proof of Lemma \ref{lemma:rel-moving-planes-quadrics-main}
	show that the above must be $T$-generated by linear syzygies.
	The result now follows.
\end{proof}

\begin{proof}[\bf Proof of Theorem \ref{thm:rel-moving-planes-quadrics}]
	By Lemma \ref{lemma:rel-moving-planes-quadrics-second} we know that $N=N_1|N_2$
	and since there are $p$ linear and $\binom{p}{2}$ quadratic syzygies
	by Lemma \ref{lemma:rel-moving-planes-quadrics-main}, $N$ is square.
	By Theorem \ref{thm:gcd-minors},
	\[
		\det(N)=\det(N_1|N_2)=P^{\deg\phi}
	\]
	and comparing the degrees on both sides, we see that $\deg(\phi)=1$.
\end{proof}


%% STRIP BEGIN
%% BIBLIOGRAPHY
\bibliographystyle{unsrtnat}
\bibliography{../lib/refs}

\end{document}