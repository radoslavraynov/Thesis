\documentclass[fleqn,reqno]{amsart}
\usepackage{../lib/radoslav-macro}
\usepackage{../lib/radoslav-more}
\usepackage{times}
\usepackage{natbib}

\newcounter{chapter}
\setcounter{chapter}{8}
\numberwithin{first}{chapter}
\numberwithin{section}{chapter}
\numberwithin{equation}{first}

\begin{document}
%% STRIP END


\section{Template Proofs}

\begin{paragraf}
Most of the proofs of the ``linear and quadratic syzygies''-type results consist, in one form or another,
of the following steps
\begin{enumerate}
\item
consider specific $X$ and set some conditions on the base locus
\item
choose $\d$ as a fixed function on $\mathbf e$
\item
come up with a way to construct a square matrix by selecting a number of linear and quaratic syzygies
\item
show that the determinant is in $P$ or $P^{\deg\phi}$
\item
show that the determinant is nonzero
\end{enumerate}

Examples of this approach are [give a long list of refereces].

The conditions on the baselocus in (1) and (2) are dictated by making the numerics in (3) workout.
Step (4) is often automatic, for example when map is assumed birational,
while step (5) usually involves a cleverly chosen normal form for the syzygies,
allowing the authors to assert that the determinant is nonzero.
\end{paragraf}

\begin{paragraf*}
Clearly, the above pattern is just a statement about the matrix $N$ in a specific degree~$\mathbf d$.
By Theorem \ref{thm:alg-structure-baselocus},
the $h$-vector of $N$ depends only on the base locus of $\phi$,
but getting a hold of this relation is difficult.
An example of this is the long lists of conditions in [the paper with B1-B6]
in case the basepoints (over $\PP^2$ and $\PPP$) are complete intersection points.

In the two simplest cases, however, of basepoint-free maps over $\PPP$ and $\PP^2$,
there are very few extra conditions, and it is desirable to know if our matrices match the
ones described elsewhere.
We show that this is indeed the case in Sections \ref{sec:bpf-maps-ppp} and \ref{sec:bpf-maps-pp2}.
Because of the simplicity of the baselocus, or rather the absence there of,
our proofs are streamlined by the fact that syzygies of low degree exhibit a property akin to Koszulness.
\end{paragraf*}



\section{Koszul Syzygies in Low Degree}
\label{sec:bpf-low-koszul}

\begin{proposition}
\label{prop:koszul-syzyiges}
Let $X$ be as in Section \ref{hui}.
Let $\phi_0,\ldots,\phi_n$ be $n$ forms in degree $\mathbf p$ having no common zero on $X$,
and let $\mathbf q$ be any degree such that
\[
H^1(X,\O_X(\mathbf q-(n-1)\mathbf p))=0
\]
Then for any $k\geq1$ the syzygies of
\[
M_0=\begin{bmatrix}\phi_0^k& \phi_0^{k-1}\phi_1& \ldots& \phi_n^k\end{bmatrix}
\]
in degree $\mathbf q$ are Koszul.
\end{proposition}

\begin{proof}
	Let $M_\ell$, $\ell=1,\ldots,n-1$, be matrices resolving $M_0$ when
	the $\phi_j$ are thought of as variables.
	Writing $\star$ for any positive integer which is not essential for the proof,
	we get a complex of coherent sheaves of modules on $X$,
	\[
		\O_X^1
		\xleftarrow{~\widetilde{M}_0~}\O_X(-k\mathbf p)^\star
		\xleftarrow{~\widetilde{M}_1~}\O_X(-(k+1)\mathbf p)^\star
		\xleftarrow{~\widetilde{M}_2~}\cdots
		\xleftarrow{~\widetilde{M}_{n-1}~}\O_X(-(k+n-1)\mathbf p)^\star
		\Ot0
	\]
	and tensoring with the locally free $\O_X(k\mathbf p+\mathbf q)^1$,
	we get the following complex ${\mathcal K}_\bullet$
	\[
		\O_X^1(k\mathbf p+\mathbf q)
		\xleftarrow{~\widetilde{M}_0~}\O_X(\mathbf q)^\star
		\xleftarrow{~\widetilde{M}_1~}\O_X(\mathbf q-\mathbf p)^\star
		\xleftarrow{~\widetilde{M}_2~}\cdots
		\xleftarrow{~\widetilde{M}_{n-1}~}\O_X(\mathbf q-(n-1)\mathbf p)^\star
		\Ot0
	\]
	
	Let $U$ be a distinguished affine open in $X$.
	Restring ${\mathcal K}_\bullet$ to $U$ is equivalent to first restricting
	the $\phi_j$ to $U$ and then constructing the $M_\ell$.
	But since the restrictions $\phi_j$ have no common zero on $U$,
	and $U\cong\AA^{n-1}$, the $n$ restrictions form a regular sequence there.
	It follows that the restriction of ${\mathcal K}_\bullet$ to $U$ is acyclic.
	Since $X$ has is covered by such opens $U$,
	${\mathcal K}_\bullet$ itself must be acyclic.
	
	By the last paragraph, ${\mathcal K}_\bullet$ is exact,
	so we can split it into two exact sequences, as the commutative
	diagram below shows,

	\begin{tikzcd}[]
		\O_X^1(k\mathbf p+\mathbf q)
			\arrow[leftarrow]{r}{\widetilde{M}_0}&
		\O_X(\mathbf q)^\star& &
		\O_X(\mathbf q-\mathbf p)^\star
			\arrow[dashed]{ll}{}[swap]{\widetilde{M}_1}&
		\ldots
			\arrow[]{l}{}[swap]{\widetilde{M}_2}\\
		&& \theIm(\widetilde{M}_1)
			\arrow[]{lu}{}[swap]{\iota}
			\arrow[leftarrow]{ru}{\pi}
			\arrow[leftarrow]{rd}
			\arrow[]{ld} &&\\
		&0&&0&
	\end{tikzcd}
	
	Consider the long exact sequences on cohomology of the exact sequences
	of sheaves above.
	For the one involving $\iota$, the releavant part is
	\[
		\ldots
		\xleftarrow{~H^0\widetilde{M}_0~} H^0\O_X(\mathbf q)^\star
		\xleftarrow{~H^0\iota~} H^0\O_X(\mathbf q)^\star
		\Ot 0
	\]
	while for the other one, we look at
	\[
		\ldots
		\xleftarrow{~H^1\widetilde{M}_2~} H^1\O_X(\mathbf q-(n-1)\mathbf p)^\star
		\xleftarrow{~\delta~} H^0\theIm(\widetilde{M}_1)^\star
		\xleftarrow{H^0\pi} H^0\O_X(\mathbf q-\mathbf p)^\star
		\xleftarrow{H^0\widetilde{M}_2} \ldots
	\]
	
	Finally, let $\mathbf v\in S_{\mathbf q}^{\binom{k+n-1}{n-1}}=H^0\O_X(\mathbf q)^\star$ be
	a syzygy on $M_0$,
	and recall that $H^0\widetilde{M}_\ell=M_\ell$ and
	$\iota\circ\pi=\widetilde{M}_1$.
	Since $M_0 \mathbf v=0$, then $\mathbf v=H^0\iota(\mathbf w)$
	for some $\mathbf w\in \theIm(\widetilde{M}_1)$.
	Since $H^1\O_X(\mathbf q-(n-1)\mathbf p)=0$,
	so is $H^1\O_X(\mathbf q-(n-1)\mathbf p)^\star$,
	and it follows that $H^0\pi$ is surjective.
	In particular, $\mathbf w=H^0\pi(\mathbf u)$ for some $u\in H^0\O_X(\mathbf q-\mathbf p)^\star$.
	But then
	\[
		\mathbf w=H^0\iota\circ H^0\pi(\mathbf u)=H^0\widetilde{M}_1(\mathbf u)=M_1\mathbf u
	\]
	establishing the claim.
\end{proof}

\begin{remarkhint}
Note that, in the last paragraph of the proof,
we cannot use $\theIm(\widetilde{M}_1)=\widetilde{\theIm(M_1)}$
since $X$ isn't affine---this is the whole point of the proof.
\end{remarkhint}

\begin{example}
	\label{ex:koszul-syzygies}
	In the case $n=3$ and $k=2$, the matrices $M_\ell$ can be taken as
	\begin{align*}
		&M_0=\begin{bmatrix}\phi_{0}^{2}&
		      \phi_{0} \phi_{1}&
		      \phi_{1}^{2}&
		      \phi_{0} \phi_{2}&
		      \phi_{1} \phi_{2}&
		      \phi_{2}^{2}\\
		      \end{bmatrix}\nonumber\\
		&M_1=\begin{bmatrix}{-{\phi}_{1}}&
	      0&
	      {-\phi_{2}}&
	      0&
	      0&
	      0&
	      0&
	      0\\
	      \phi_{0}&
	      {-\phi_{1}}&
	      0&
	      {-\phi_{2}}&
	      0&
	      0&
	      0&
	      0\\
	      0&
	      \phi_{0}&
	      0&
	      0&
	      0&
	      {-\phi_{2}}&
	      0&
	      0\\
	      0&
	      0&
	      \phi_{0}&
	      \phi_{1}&
	      {-\phi_{1}}&
	      0&
	      {-\phi_{2}}&
	      0\\
	      0&
	      0&
	      0&
	      0&
	      \phi_{0}&
	      \phi_{1}&
	      0&
	      {-\phi_{2}}\\
	      0&
	      0&
	      0&
	      0&
	      0&
	      0&
	      \phi_{0}&
	      \phi_{1}\\
	      \end{bmatrix}\\
		  &M_2=\begin{bmatrix}\phi_{2}&
	      0&
	      0\\
	      0&
	      \phi_{2}&
	      0\\
	      {-\phi_{1}}&
	      0&
	      0\\
	      \phi_{0}&
	      {-\phi_{1}}&
	      0\\
	      0&
	      {-\phi_{1}}&
	      \phi_{2}\\
	      0&
	      \phi_{0}&
	      0\\
	      0&
	      0&
	      {-\phi_{1}}\\
	      0&
	      0&
	      \phi_{0}\\
      \end{bmatrix}\nonumber
	\end{align*}
	
	Note that the size of $M_1$ is $\binom{n-1+k}{n-1}\times$,
	and the size of $M_{n-1}$ is $\binom{n-2+k}{n-1}\times$
\end{example}

\begin{question}
	What is the size of $M_\ell$ in general? This should come from a cellular resolution
	of the ideal of all monomials in $n$ variables in degree $k$, right?
	Also, it's important to say that this is a linear type resolution---all the maps are
	of degree $1$, except for $M_0$, obviously.
\end{question}

\begin{corollary}
	Let $X=\PP^{n-1}$ be the projective $(n-1)$-space or
	$X=\PP^1\times\cdots\PP^1$ be the product of $n-1$ copies of $\PP^1$,
	and let $\phi:X\to\PP^n$ be basepoint-free.
	Let the coordinates of $\phi$ be in degree $\mathbf p$
	and suppose that there are as few linear syzygies as possible.
	Then, after a linear change of coordinates on the target,
	$\phi_0,\ldots,\phi_{n-1}$ satisfy the assumption of Proposition \ref{prop:key-koszul}.
\end{corollary}

\begin{proof}
	work out
\end{proof}

\begin{paragraf}[Template Proof]
\end{paragraf}



\section{Basepoint-Free Maps over $X=\PPP$}
\label{sec:bpf-maps-ppp}

\begin{paragraph*}
	In this section and next section we workout the relation between our method and
	the ones described in \citet{00-CGZ-JSC}, whenever the latter can be applied.
	In particular, we first show that the matrices which our method constructs are the same, up to
	reformatting, as the matrices constructed by the referenced methods,
	and then show how our streamlined arguments apply to a more general situation.
\end{paragraph*}

\begin{theorem}
	\label{thm:rel-moving-quadrics}
	Let $X=\PP^1\times\PP^1$, $\phi$ be basepoint-free whose entries are of degree $(p,q)$,
	and suppose there are no linear syzyges for $d=(p-1,q-1)$.
	Then $N_2$ is square, $N=N_2$ and $\phi$ is birational. In particular,
	\[
		\det(N)=\det(N_2)=P(\mathbf x)
	\]
\end{theorem}

\begin{remark}
	Theorem \ref{thm:rel-moving-quadrics} implies that if the map $\phi$ is basepoint-free but not birational,
	then there must exist linear syzygies in degree $(p-1,q-1)$.
\end{remark}

\begin{paragraph}
\label{par:count-rels}
Let $k$ be a positive integer. We set the matrix $\Phi^{(k)}$,
\[
	\Phi^{(k)}:S_{p-1,q-1}^{\binom{k+3}{3}}\To S_{(k+1)p-1,(k+1)q-1}^1
\]
to be a $\CC$-linearization of the $S$-linear map
\[
	\phi^{(k)}=\begin{bmatrix}\phi_0^k& \phi_0^{k-1}\phi_1& \cdots& \phi_3^k\end{bmatrix}:
	S^{\binom{k+3}{3}}\To S^1
\]

Specifically, $\Phi^{(k)}$ is of size $(k+1)^2pq\times\binom{k+3}{3}pq$, and so
\begin{align}
	\dim_{\CC}\ker(\Phi^{(k)})\geq\binom{k+1}{3}pq\label{eqn1}
\end{align}
with equality if and only if $\Phi^{(k)}$ is of maximal rank, that is, of rank $(k+1)^2pq$.
\end{paragraph}

\begin{paragraph}
	The columns of $\Phi^{(k)}$ are indexed by the monomials in $R_{p-1,q-1,k}$ and the elements
	of the kernel give rise to linear combinationations of those which are exactly
	the syzygies of degree $k$ for the fixed $d=(p-1,q-1)$. We can write any such syzygy as
	\begin{align}\label{ebasi}
		\sum_{|\alpha|=k} A_\alpha\cdot
		x_0^{\alpha_0}x_1^{\alpha_1}x_2^{\alpha_2}+
		\big(\sum_{|j|=k-1} B_{j}\cdot
		x_0^{j_0}x_1^{j_1}x_2^{j_2}\big)x_3+
		\big(\sum_{|\gamma|=k-2}C_\gamma x^\gamma\big)x_3^2
	\end{align}
	in which the first two summations corresponds to number of columns indexed by
	monomials involving $x_3$ at most linearly.
	This number is $\binom{k+2}{2}pq+\binom{k+1}{2}pq=(k+1)^2pq$, which is the number of rows too.
	Set $\Psi^{(k)}$ to be the square submatrix of $\Phi^{(k)}$ formed by those columns.
	
	Note that if the former is nonsingular, then the latter is of maximal rank. The proof only
	depends on $\Psi^{(k)}$ being of maximal rank though, not the fact that it is square,
	so we procede in this manner.
\end{paragraph}

\begin{paragraph*}
	We begin by proving that the kernel elements of the $S$-linear version of $\Phi^{(k)}$
	in degree $(2p-1,2q-1)$ are Koszul,
	i.e. are $S_{p-1,q-1}$-linear combinations of the Koszul syzygies on the $\phi_j$
	appropriately inflated to match $\Phi^{(k)}$.
	
	While the proof is pretty much the same, as in \citet{00-CGZ-JSC},
	we include it here fitted for our notation and generalized slightly.
\end{paragraph*}

\begin{lemma}
	\label{lemma:Psi-nonsing}
	The matrix $\Psi^{(k)}$ is of maximal rank for every positive $k$.
\end{lemma}

\begin{proof}
	The case $k=1$ is just the assumption that the $\phi_j$ there are no linear syzygies
	in degree $(p-1,q-1)$ on the source,
	i.e. if
	\[
		a_0\phi_0+a_1\phi_1+a_2\phi_2+a_3\phi_3=0
	\]
	in $S$ where $a_j\in S_{p-1,q-1}$, then necessarily $a_j=0$.
	
	The case $k>1$ is a direct consequence of the case $k=1$ and the fact that
	$\phi_0,\phi_1,\phi_2$ form a regular sequence. Let $V$,
	\[
		V=\begin{bmatrix}{V'}\\{V''}\end{bmatrix}
	\]
	be a kernel element of $\Psi^{(k)}$ with $V'$ a column-$\binom{k+2}{2}$-vector over $\CC$
	and $V''$ is a column-$\binom{k+1}{2}$-vector over $\CC$, corresponding to the coefficients
	not involving $x_3$ and the coefficients involving $x_3$, respectively.
	We can multiply out to get a $k$-syzygy on the $\phi_j$ over $S_{p-1,q-1}$, i.e.
	\[
		\sum_{|\alpha|=k} A_\alpha\cdot
		\phi_0^{\alpha_0}\phi_1^{\alpha_1}\phi_2^{\alpha_2}+
		\big(\sum_{|j|=k-1} B_{j}\cdot
		\phi_0^{j_0}\phi_1^{j_1}\phi_2^{j_2}\big)\phi_3=0
	\]
	as an equality in $S$, where the coefficients $A_\alpha, B_j$ are in $S_{p-1,q-1}$.
	We rewrite the above as
	\[
		\sum_{|j|=k-1}\big(
		C_{0,j}\phi_0+C_{1,j}\phi_1+C_{2,j}\phi_2+C_{3,j}\phi_3\big)
		\phi_0^{j_0}\phi_1^{j_1}\phi_2^{j_2}=0
	\]
	which is again an equality in $S$, and again the coefficients $C_{j,j}$ are in $S_{p-1,q-1}$.
	Note also that $C_{3,j}=B_j$.
	
	Since $\phi_0,\phi_1,\phi_2$ form a regular sequence over $S$,
	we can rewrite the coefficients as linear combinations of their Koszul syzygies,
	i.e. for each $j$ we get an equality in $S$ of the form
	\[
		C_{0,j}\phi_0+C_{1,j}\phi_1+C_{2,j}\phi_2+C_{3,j}\phi_3=
		D_{0,j}\phi_0+D_{1,j}\phi_1+D_{2,j}\phi_2
	\]
	where $D_0,D_1,D_2$ are in $S_{p-1,q-1}$.
	By the assumption on the independence of the $\phi_j$ over $S_{p-1,q-1}$,
	we must have that $C_{3,j}=0$.
	Equivalently, $B_j=0$ for every $j$, so $V''=0$.
	
	To conclude the proof, note further that any nontrivial syzygy
	of the monomials in $\phi_0,\phi_1,\phi_2$
	in any degree $k\geq1$ must be of degree degree at least $(p,q)$,
	i.e. must be a combination of Koszul syzygies again.
	Since $\deg(A_\alpha)=(p-1,q-1)$, it follows that $V'=0$ too.
\end{proof}

\begin{paragraph}
	Paragraph \ref{count-rels} and Lemma \ref{lemma:Psi-nonsing} show that there are exactly
	$\binom{k+1}{3}pq$ linearly independent syzygies of degree $k$, so $pq$ quadratic syzygies.
	Clearly degree-$(k-2)$ $T$-combinations of quadratic syzygies introduce degree-$k$ syzygies.
	Next lemma shows that all syzygies of degree $k>2$ arise in this way.
\end{paragraph}

\begin{lemma}
	\label{another-lemma}
	Any degree $k>2$ syzygy of degree $d=(p-1,q-1)$ on the source is a degree-$(k-2)$
	$T$-combination of quadratic syzygyes.
\end{lemma}

\begin{proof}
	Since $\Psi^{(k)}$ is square and of maximal rank by Lemma \ref{lemma:Psi-nonsing},
	it is nonsingular, so $\Phi^{(k)}$ is of maximal rank, $(k+1)^3pq$.
	This means that there are $pq$ linearly independent quadratic syzygies
	and by the same lemma any one of them must involve $x_3^2$ nontrivially.
	It follows that, up to scaling, any nonzero quadratic syzygy is of the form
	\[
		g_\nu(\mathbf s,\mathbf x)=\ldots+{\mathbf{s}}^\nu x_3
	\]
	where the $\mathbf{s}^\nu$ are the monomials of degree $(p-1,q-1)$ on the source.
	
	We can rewrite an arbitrary degree-$(k>2)$ syzygy as
	\[
		g(\mathbf s, \mathbf x)=\sum_{|\alpha|=k} A'_\alpha\cdot
		x_0^{\alpha_0}x_1^{\alpha_1}x_2^{\alpha_2}+
		\big(\sum_{|j|=k-1} B'_{j}\cdot
		x_0^{j_0}x_1^{j_1}x_2^{j_2}\big)x_3+
		\big(\sum_\nu \mathbf{s}^\nu f_\nu\big)x_3^2
	\]
	where each $f_\nu$ is a form degree $k-2$ in $T$, so the syzygy
	\[
		g(\mathbf s, \mathbf x)-\sum_\nu f_\nu\cdot g_\nu(\mathbf s, \mathbf x)
	\]
	has trivial $x_3^k$ part in the sense of the discussion above.
	By Lemma \ref{lemma:Psi-nonsing} again, this syzygy must be zero,
	so every syzygy of degree $k$ is a $T$-combination of quadratic syzygyes.
\end{proof}

\begin{proof}[\titlestyle Proof of Theorem \ref{thm:rel-moving-quadrics}]
	Since there are no syzygies in degree $1$ and all syzygyes in degree $k>2$ are $T$-combinations
	of quadratic syzygyes by Lemma \ref{another-lemma},
	we have $N=N_2$. Since the number of linearly independent syzygies
	of degree $2$ is $pq$, the matrix $N_2$ is square.
	
	To finish the prove we note that necessarily $\reg(B_P)\leq(p-1,q-1)$ since the ideal of points
	$I_P$ is generated in degree $(p,q)$ over $K(T/P)$,
	so Theorem \ref{thm:gcd-minors} applies.
	
	Finally, birationallity follows by comparing the degree on both hand-sides of Theorem \ref{thm:gcd-minors},
	\[
		\deg(Y)\deg(\phi)=\deg\det (N)=\deg\det(N_2)=2pq
	\]
	and $\deg(Y)=2pq$ since the parametrization is basepoint-free, so $\deg\phi=1$.
\end{proof}

\begin{paragraph}
	\label{par:extended-moving-quadrics}
	In the more general setting when $X$ is an $(n-1)$-fold product of $\PP^1$s
	and the coordinates of $\phi$ be of $(n-2)$-degree $(p_1,\ldots,p_{n-1})$.
	Now the sizes of $\Phi^{(k)}$ and $\Psi^{(k)}$ are
	\[
		(k+1)^{n-1}p_1\cdots p_{n-1}\times\binom{k+n}{n}p_1\cdots p_{n-1}
	\]
	and
	\[
		(k+1)^{n-1}p_1\cdots p_{n-1}\times(\binom{k+n-1}{n-1}+\binom{k-1+n-1}{n-1})p_1\cdots p_{n-1}
	\]
	respectively.
	If $\phi_0,\ldots,\phi_{n-1}$ form a regular sequence,
	then the arguments of Lemma \ref{lemma:Psi-nonsing} apply verbatim,
	and $\Psi^{(k)}$ is of maximal rank, although not square for $n>3$.
	
	The arguments of Lemma \ref{another-lemma} apply also, as long as the number of
	linearly independent quadratic syzygies is the number of monomials of degree
	$(p_1-1,\ldots,p_{n-1}-1)$ on the source, $p_1\cdots p_{n-1}$.
	Since this is just the difference between the number of columns of $\Phi^{(2)}$ and $\Psi^{(2)}$,
	\[
	(\binom{n}{2}-\binom{n+1}{2}-\binom{n}{1})p_1\cdots p_{n-1}
	\]
	the following result follows.
	
	The problem to extend the result is however the fact that $\Psi^{(k)}$ is no longer square,
	more precisely, the number of columns is less than the number of rows for $n>3$,
	so we cannot infer that the ranks of $\Psi^{(k)}$ and $\Phi^{(k)}$ are the same.
	The latter was necessary to find at least (and so exactly) $\dim_\CC(S_{p_1-1,\ldots,p_{n-1}-1})$ many
	quadratic syzygyes, which in turn is integral for the proof of Lemma \ref{another-lemma}.
	
	The following example shows this situation does occur.
\end{paragraph}

\begin{example}
	... Then there are no quadratic syzygies.
\end{example}

\begin{example}
	... is there an example for which the matrix is square?
\end{example}

\begin{paragraph}
	At any rate, the issue with the number of quadratic syzygies is the only problem
	to extending Theorem \ref{thm:rel-moving-quadrics} to any product of $\PP^1$s or to any degree-$k$ syzygies,
	so by adding the latter as an assumption,
	we have the following provisional extension.
\end{paragraph}

\begin{remark}
	One can hope that something similar would work in higher dimension.
	For example, let $X=(\PP^1)^{n}$ and $\L=\O(p_1,\ldots,p_{n})$.
	Assuming that there are no degree-$(n!-1)$ syzygies over $S_{p_1-1,\ldots,p_{n}-1}$
	we may expect to find $p_1\cdots p_n$ syzygies of degree $n!$
	which to conviniently organize in an $p_1\cdots p_n$-square matrix.
	The determinant of this matrix would be homogeneous of degree $n!\cdot p_1\cdots p_n$
	which is exactly the degree of the image $Y\se\PP^{n+1}$ in the absense of basepoints.
	While one would still need to argue that $N_k=0$ for $k>n!$,
	this is the gist of what happend in the case $n=2$.
	
	However, already the case $n=3$ and $\mathbf p=(2,2,2)$ shows otherwise.
	We get no sextic syzygies and, in fact, degree-$k$ syzygies over $S_{1,1,1}$ for $k\leq8$.
	The latter calculation is outlined in $\mt{(ex101)}$.
\end{remark}

\begin{theorem}
	Let $X=\PP^1\times\cdots\times\PP^1$ be an $(n-1)$-fold product and $\phi$ be basepoint-free
	with coordinates in degree $(p_1,\ldots,p_{n-1})$ such that there are no degree-$(m-1)$ syzygies in
	degree $d=(p_1-1,\ldots,p_{n-1}-1)$ on the source
	and such that the number of linearly independent degree-$m$ syzygies is $p_1\cdots p_{n-1}$.
	Then $N=N_m$ is square and
	\[
		\det(N)=P^{\deg(\phi)}
	\]
\end{theorem}






\section{Basepoint-Free Maps over $X=\PP^2$}
\label{sec:bpf-maps-pp2}

\begin{lemma}
	\label{lemma:rel-moving-planes-quadrics-min-dims}
	For every $k>0$, the number of linearly independent degree-$k$ syzygies over $S_{d}$,
	$d=p-1$, is at least
	\[
		\frac{k(k+1)p}{12}(kp-p+k+5)
	\]
\end{lemma}

\begin{proof}
	As before, we consider the $S$-linear map
	\[
		\begin{bmatrix}\phi_0^k& \phi_0^{k-1}\phi_1& \cdots& \phi_3^k\end{bmatrix}:
		S^{\binom{k+3}{3}}\To S^1
	\]
	and construct its $\CC$-linearization,
	exactly as we did for $\Phi^{(k)}$ in Section \ref{sec:rel-moving-quadrics}.
	It is a map of complex vector spaces
	\[
		S_{p-1}^{\binom{k+3}{3}}\To S_{(k+1)p-1}^1
	\]
	so a matrix of size $\binom{kp+p+1}{2}\times\binom{k+3}{3}\binom{p+1}{2}$.
	Since there are at least as many columns as rows,
	the dimension of the kernel is at least
	\[
		\binom{k+3}{3}\binom{p+1}{2}-\binom{kp+p+1}{2}=\frac{k(k+1)p}{12}(k+5+kp-p)
	\]
	establishing the claim.
\end{proof}

\begin{remark}
	\label{rem:rel-moving-planes-quadrics-num-syz}
	By Lemma \ref{lemma:rel-moving-planes-quadrics-min-dims} there are
	at least $p$ linear syzygies and at least $p(p+7)/2$ quadratic syzygies.
\end{remark}

\begin{paragraph*}
	The main result of this section states that the matrices which our method constructs
	are the same, up to formatting, as those constructed by the method of \citet{00-CGZ-JSC}.
\end{paragraph*}

\begin{theorem}
	\label{thm:rel-moving-planes-quadrics}
	Let $X=\PP^2$, $\phi$ be basepoint-free whose entries are of degree $p$,
	and suppose there are exactly $p$ linear syzyges for $d=p-1$, i.e. the minimal possible number.
	Then $N=N_1|N_2$, $N$ is square and $\phi$ is birational. In particular,
	\[
		\det(N)=\det(N_1|N_2)=P(\mathbf x)
	\]
\end{theorem}

\begin{lemma}
	\label{lemma:rel-moving-planes-quadrics-main}
	Suppose that there are exactly $p$ linearly independent linear syzygies, i.e. the minimal possible number.
	Then the $4p$ quadratic syzygies of the form $L_i(\mathbf s,\mathbf x)x_j$ for the $p$ linear syzygies $L_i$
	are linearly independent. The number of linearly independent quadratic syzygies not emerging in this way
	is $\binom{p}{2}$. The vector subspace of $S_{p-1}$ spanned by the coefficient of $x_3^2$ among all
	quadratic syzygies is all of $S_{p-1}$.
\end{lemma}

\begin{proof}
	Since $\phi_0,\phi_1,\phi_2$ form a regular sequence in degree $p$,
	a nonzero linear syzygy over $S_{p-1}$ must involve $x_3$ nontrivially.
	Let $V_1$ be the linear subspace of $S_{p-1}$ spanned by the
	coefficient of $x_3$ among all linear syzygies, i.e.
	\[
		V_1=\Span\{a_3:a_0x_1+\ldots+a_3x_3 \text{ is a linear syzygy}\}
	\]
	(and note that the {\em span} keyword isn't necessary).
	Since there are exactly $p$ of those, by the observation just made, $\dim_\CC V_1=p$.
	
	Any linear syzygy $L(\mathbf s, \mathbf x)$ gives rise to a quadratic syzygy of the form
	$L(\mathbf s, \mathbf x)x_j$ for each $j$.
	Let $g(\mathbf s,\mathbf x)$ be a quadratic syzygies not arising in this way.
	We know that $g$ must involve $x_3^2$ nontrivially,
	for else, it is of the form
	\[
		\sum_{j=0,1,2} (B_{j,0}x_0+B_{j,1}x_1+B_{j,2}x_2+B_{j,3}x_3) x_j
	\]
	and since $\phi_0,\phi_1,\phi_2$ form a regular sequence in degree $p$,
	by the Koszul theorem (e.g. \ref{prop:key-koszul}),
	we have
	\[
		B_{j,0}x_0+B_{j,1}x_1+B_{j,2}x_2+B_{j,3}x_3=C_{j,0}x_0+C_{j,1}x_1+C_{j,2}x_2
	\]
	In particular,
	\[
		L_j(\mathbf s,\mathbf x)=B_{j,0}x_0+B_{j,1}x_1+B_{j,2}x_2+B_{j,3}x_3-C_{j,0}x_0-C_{j,1}x_1-C_{j,2}x_2
	\]
	is a linear syzygies on the $\phi_j$.
	But then $g-\sum_j L_jx_j$ is a quadratic syzygy only involving $x_0,x_1,x_2$,
	and so must be $0$, contradicting the assumption that $g$ was not $T$-generated by linear syzygies.
	
	Let $V_2$ be the linear subspace of $S_{p-1}$ spanned by the coefficient of $x_3^2$ for quadratic
	sygyzygies $g$ as in the previous paragraph. The same argument shows that we cannot have $V_1\cap V_2\neq0$.
	
	We finish the proof by an easy dimension count.
	The discussion so far gives us that the number of {\em new} quadratic syzygies is at most
	$\dim_\CC V_2\leq\binom{p+1}{2}-p$,
	so even if all the $4p$ quadratic syzygies generated by the linear syzygies
	and the new quadratic syzygies are linearly independent altogether,
	we get at most $4p+\binom{p+1}{2}-p=p(p+7)/2$-many of them.
	On the other hand,
	the number of linearly independent quadratic syzygies must be at least $p(p+7)/2$
	by Remark \ref{rem:rel-moving-planes-quadrics-num-syz}.
	
	It follows that there are $\binom{p+1}{2}-p=\binom{p}{2}$ {\em new} quadratic syzygies,
	which along with the $4p$ pushed linear syzygies are linearly independent altogether.
	Also, the linear span of the $S_{p-1}$ coefficient of $x_3^2$ among the quadratic syzygies
	is $V_1\oplus V_2=S_{p-1}$.
\end{proof}

\begin{lemma}
	\label{lemma:rel-moving-planes-quadrics-second}
	Let $g(\mathbf s, \mathbf x)$ be a syzygy of degree $k>2$.
	Then $g$ is a $T$-combination of linear and quadratic syzygies.
\end{lemma}

\begin{proof}
	The proof uses the same arguments as in the proof of Lemma \ref{lemma:rel-moving-planes-quadrics-main}.
	Since the $S_{p-1}$-coefficients of $x_3^2$ among the quadratic syzygies span $S_{p-1}$,
	we can find among them
	\[
		g_\nu(\mathbf s,\mathbf x)=(\text{terms involving $x_3$ at most linearly})+{\mathbf s}^\nu x_3^2
	\]
	for all $|\nu|=p-1$. We can then rewrite
	\[
		g(\mathbf s,\mathbf x)=(\text{terms involving $x_3$ at most linearly})+(\sum_\nu {\mathbf s}^\nu\cdot h_\nu(\mathbf x))x_3^2
	\]
	where $h_v(\mathbf x)$ is a degree-$(k-2)$ homogeneous polynomial. But now
	\[
		g-\sum_\nu h_\nu g_\nu=\sum_{|\beta|=k-1} (A_{\beta,0}x_0+A_{\beta,1}x_1+A_{\beta,2}x_2+A_{\beta,3}x_3) x_0^{\beta_0}x_1^{\beta_1}x_2^{\beta_2}
	\]
	where $A_{\beta,j}$ is a $(p-1)$-form in $S$.
	Lemma \ref{prop:key-koszul} and the same argument as in the proof of Lemma \ref{lemma:rel-moving-planes-quadrics-main}
	show that the above must be $T$-generated by linear syzygies.
	The result now follows.
\end{proof}

\begin{proof}[\bf Proof of Theorem \ref{thm:rel-moving-planes-quadrics}]
	By Lemma \ref{lemma:rel-moving-planes-quadrics-second} we know that $N=N_1|N_2$
	and since there are $p$ linear and $\binom{p}{2}$ quadratic syzygies
	by Lemma \ref{lemma:rel-moving-planes-quadrics-main}, $N$ is square.
	By Theorem \ref{thm:gcd-minors},
	\[
		\det(N)=\det(N_1|N_2)=P^{\deg\phi}
	\]
	and comparing the degrees on both sides, we see that $\deg(\phi)=1$.
\end{proof}


%% STRIP BEGIN
\end{document}