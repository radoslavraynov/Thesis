\documentclass[fleqn,reqno]{amsart}
\usepackage{../lib/radoslav-macro}
\usepackage{../lib/radoslav-more}
\usepackage{times}
\usepackage{natbib}


\newcounter{chapter}
\setcounter{chapter}{1}
\numberwithin{first}{chapter}


\begin{document}
%% STRIP END

The implicitization problem asks for finding the equations defining the image of a rational map
into some projective space $\PP^n$.
This is an old problem in algebraic geometry which was effectively solved in the 60s
by Gr\"obner bases.
However, this method has two intrinsic shortcomings.
The first is its computational complexity.
Even relatively small examples may require, using today's computational capacity,
running time in the hundreds of years or more.
The second problem is that geometrically it represents a black box.

%% STRIP BEGIN

\end{document}
