\documentclass[fleqn,reqno]{amsart}
\usepackage{../lib/radoslav-macro}
\usepackage{../lib/radoslav-more}
\usepackage{times}
\usepackage{natbib}

\begin{document}
%% STRIP END

The main goal of this thesis is to develop a framework for both
efficiently computing the closed image of a rational map $\phi:X\To\PP^n$, and
studying the relation between algebraic properties of the coordinates $\phi_j$ on one hand,
and geometric properties of the image~$Y$, on the other.
We consider the setup where $X$ is a smooth projective toric variety
and are most interested in the case when $X$ of dimension $n-1$ and $\phi$ is generically finite.
In this case,
finding the image entails computing a single equation~$P(\mathbf x)$,
called the implicit equation.

This is a version of the implicitization problem which is classical in algebraic geometry.
It has an effective solution through elimination using Gr\"obner bases, for instance,
but those techniques have two drawbacks.
First, they become unfeasible even for examples of modest complexity.
Second, they represent a black box in terms of geometry---a tool which
takes the~$\phi_j$ and returns~$P$,
and whether the computation finishes before our Sun dies or not,
no geometric feature can be used or inferred from the process.
Motivated partially by the advance of computer aided design (CAD),
interest in new approaches to the implicitization problem
reappeared in the 1990s.
One reference often cited as the initiating work is \cite{SC-95},
a SIGGRAPH paper using moving curves and surfaces for implicitization.
One of the first works building on these idea from the standing point
of commutative algebra and algebra geometry is \citet{CGZ-00}.

Our approach follows the theme of moving surfaces but sets to remove the intrinsic
ad-hoc constructions it requires.
In this sense,
we draw inspiration by the method of the approximation complex, initiated by \citet{BJ-03}.
Our main results are a blend of the results in the two papers.

Let $S$ be the Cox ring of~$X$ graded by $\Pic(X)$,
and $J=\la\phi_0,\ldots,\phi_n\ra\se S$ be the ideal of the coordinates.
Then the Rees ideal $I$ of $J$ can be thought of
an algebraic object through the Rees algebra, but also
a geometric object through the graph of $\phi$.
We construct, for any degree~$\d$ on $S$,
a matrix~$N$ representing a generating set for the polynomial relations on the $\phi_j$
with coefficients from $S_\d$ only.
This is just bigraded piece of the Rees ideal.
The matrix $N$ generalizes the types of representation matrices
studied by the aforementioned methods.
One of the our main results in an analogue to the results for approximation complex.
We show that if $N$ is of size $r\times\mu$, then $\text{rank}(N)=r$ and
\[
	\gcd(\text{minors}(r,N))=P^{\deg(\phi)}
\]
Relaxing the requirement that $X$ be of dimension $n-1$, we find that
\[
	\rad(\text{minors}(r,N))=P
\]

Results of form:
if conditions (1)--(k) hold, then there exists a square matrix $M$ such that $\det(M)=P$,
which are the main theme of the moving surfaces method,
are now equivalent to saying the degree-$(\d,\bullet)$
ideal of the coordinates $J$ has $r$ relations over $S_d$
as described above.
This suggest a template for proofs allowing one to focus on the various conditions only.

Moreover, the matrix $N$ can be computed incrementally
using only tools from linear algebra,
allowing calculation in situations previously out of reach.

--- to the extent that some of the cases seem to work just because
two numbers turn out to be the same.

The method of the approximation complex, on the other hand, is more robust.

\newpage
Implicitization is an old problem in algebraic geometry,
which asks for the equations of the closed image in of a rational map
given by sections sections of a line bundle on the source.
A solution via elimination has been known since the 19th century.
Today, elimination can generally be carried out through a Gr\"obner basis calculation.
However, this entails has two drawbacks.
First, such an approach is a black box with respect to geometry.
Second, this method is computationally heavy and because unfeasible for even reasonably
small examples.

In the 1990s, with the advance of Computer Aided Design (CAD) software,
the problem was brought back to life.
Others have attributed a central role to a SIGGRAPH paper, \citet{SC-95},
but similar results have appeared much earlier.
Most of those papers suggested, or rather showed empirically, how to find the implicit equation 
of the parametrization of curve in $\PP^2$, or of a surface in $\PP^3$.
The first mathematical results in terms of representation matrices ---
matrices which capture syzygies on the coordinates the rational map ---
came around the early 2000s by as series of papers, most notably \citet{CGZ-00}.
Starting with \citet{BJ-03}, another approach was suggested which departed from
linear and commutative algebra and brought the picture to algebraic geometry.
Since then many iterations and strengthenings of the those have been published.

[...] the point of our work [...] general framework [...]

Chapter~\ref{ch:preliminaries}.

Say what Example 2.23 ($\mt{ex100}$) means --- that it refers to the m2 file ex/100.m2

\newpage
Drawing insight from the work published since \citet{CGZ-00},
our work had three different directions, or goals.

The parallel between the methods of moving planes and quadrics and the methods using
the approximation

Giving up the idea that only linear and quadratic syzygies must be used, and that there is
a special degree which makes

Our second goal was to bridge the gap between 

Out third and last goal, was to address the actual problem of computing the image.
This came into being when I wanted to test a hypothesis on the image of

I wanted to test a hypothesis against the image a rational map
given by 4 general bi-quartics and having a preset base locus of total multiplicity $6$.
The degree of that image was 26 and the calculation didn't seem was going to finish.
This 

\newpage

The approach we take follows the method of moving surfaces in that we 

%% STRIP BEGIN
%% BIBLIOGRAPHY
\bibliographystyle{unsrtnat}
\bibliography{../lib/refs}

\end{document}