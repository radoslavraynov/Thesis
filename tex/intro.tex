\documentclass[fleqn,reqno]{amsart}
\usepackage{../lib/radoslav-macro}
\usepackage{../lib/radoslav-more}
\usepackage{times}
\usepackage{natbib}

\begin{document}
%% STRIP END



Implicitization is an old problem in algebraic geometry,
which asks for the equations of the closed image in of a rational map
given by sections sections of a line bundle on the source.
A solution via elimination has been known since the 19th century.
Today, elimination can generally be carried out through a Gr\"obner basis calculation.
However, this entails has two drawbacks.
First, such an approach is a black box with respect to geometry.
Second, this method is computationally heavy and because unfeasible for even reasonably
small examples.

In the 1990s, with the advance of Computer Aided Design (CAD) software,
the problem was brought back to life.
Others have attributed a central role to a SIGGRAPH paper, \citet{SC-95},
but similar results have appeared much earlier.
Most of those papers suggested, or rather showed empirically, how to find the implicit equation 
of the parametrization of curve in $\PP^2$, or of a surface in $\PP^3$.
The first mathematical results in terms of representation matrices ---
matrices which capture syzygies on the coordinates the rational map ---
came around the early 2000s by as series of papers, most notably \citet{CGZ-00}.
Starting with \citet{BJ-03}, another approach was suggested which departed from
linear and commutative algebra and brought the picture to algebraic geometry.
Since then many iterations and strengthenings of the those have been published.

[...] the point of our work [...] general framework [...]

Chapter~\ref{ch:preliminaries}.

Say what Example 2.23 ($\mt{ex100}$) means --- that it refers to the m2 file ex/100.m2

\newpage
Drawing insight from the work published since \citet{CGZ-00},
our work had three different directions, or goals.

The parallel between the methods of moving planes and quadrics and the methods using
the approximation

Giving up the idea that only linear and quadratic syzygies must be used, and that there is
a special degree which makes

Our second goal was to bridge the gap between 

Out third and last goal, was to address the actual problem of computing the image.
This came into being when I wanted to test a hypothesis on the image of

I wanted to test a hypothesis against the image a rational map
given by 4 general bi-quartics and having a preset base locus of total multiplicity $6$.
The degree of that image was 26 and the calculation didn't seem was going to finish.
This 

In

%% STRIP BEGIN
%% BIBLIOGRAPHY
\bibliographystyle{unsrtnat}
\bibliography{../lib/refs}

\end{document}