\documentclass[fleqn,reqno]{amsart}
\usepackage{../lib/radoslav-macro}
\usepackage{../lib/radoslav-more}
\usepackage{times}
\usepackage{natbib}

\begin{document}
%% STRIP END

The main goal of this thesis is to develop a framework for
efficiently computing the closed image of a rational map $\phi:X\To\PP^n$ by
making use of the relation between algebraic properties of the coordinates $\phi_j$ on one hand,
and geometric properties of the image~$Y$, on the other.
We consider the setup where $X$ is a smooth projective toric variety
and are most interested in the case when $X$ is of dimension $n-1$ and $\phi$ is generically finite.
In this case,
the image is defined by a single equation~$P(\mathbf x)$,
called the implicit equation.

This is a version of the implicitization problem, which is classical in algebraic geometry.
It has an effective solution through elimination using Gr\"obner bases, for instance,
but those techniques have two drawbacks.
First, they become unfeasible even on examples of very modest complexity.
Second, they represent a black box in terms of geometry---a tool which
takes the~$\phi_j$ and returns~$P$,
and whether the computation finishes in our Sun's lifetime or not,
no geometric feature can be used for or inferred from the process.
Motivated partially by the advance of computer aided design (CAD),
interest in new approaches to the implicitization problem
reappeared in the 1990s.
One reference often cited as the initiating work is \cite{SC-95}~---
a SIGGRAPH paper using moving curves and surfaces for implicitization.
One of the first works building on this idea from the standing point
of commutative algebra and algebraic geometry is \citet{CGZ-00}.

Our approach follows the theme of moving surfaces but sets to remove the intrinsic
ad-hoc constructions it requires.
In this sense,
we draw inspiration from the method of the approximation complex,
initiated by \citet{BJ-03}.
Our main results are a blend of the results in the two papers.

Let $S$ be the Cox ring of~$X$ and
let $J=\la\phi_0,\ldots,\phi_n\ra\se S$ be the ideal of the coordinates.
Then the Rees ideal $I$ of $J$ can be thought of as
an algebraic object through the Rees algebra, but also
a geometric object through the graph of $\phi$.
We construct, for any degree~$\d$ on $S$,
a matrix~$N$ representing a generating set for the polynomial relations on the $\phi_j$
with coefficients from $S_\d$ only.
This is just a bigraded piece of the Rees ideal.
The matrix $N$ generalizes the types of representation matrices
studied by the aforementioned papers.
One of our main results, Theorem~\ref{thm:gcd-minors},
is an analogue of a result for the approximation complex.
If we let $r\times\mu$ be the size of $N$,
we show that $\text{rank}(N)=r$ and
\[
	\gcd(\text{minors}(r,N))=P(\mathbf x)^{\deg(\phi)}
\]
Relaxing the requirement that $X$ be of dimension $n-1$,
we find in Theorem~\ref{thm:rad-minors} that
\[
	\rad(\text{minors}(r,N))=P
\]

Furthermore, results of form:
if conditions (1)--($k$) hold,
then there exists a square matrix $M$ such that $\det(M)=P(\mathbf x)$~---
which are the main theme of the moving surfaces methods,
now become essentially equivalent to saying that the degree-$(\d,\bullet)$ graded piece
of the Rees ideal is minimally generated by $r$ elements.
This suggest a template for proofs allowing one to focus the specifics of the conditions.

Moreover, the matrix $N$ can be computed incrementally
using only tools from linear algebra.
This is important because entails a significant on-average speed up,
allowing us to calculate examples which are out of reach via Gr\"obner basis techniques.
More importantly, this allows us to use geometric tricks, say,
to confirm that
\[
	P(\mathbf x)=\gcd(\det(M_1),\det(M_2))
\]
for some selected matrices $M_1$ and $M_2$.
See Examples~\ref{ex604} and~\ref{ex603}.

The thesis is organized as follows.
In Chapter~\ref{ch:preliminaries}, we establish the notation and define the main objects.
Section~\ref{sec:mult} is devoted to the notion of multiplicity of a base point and
introduces less-standard notation.
We state our main results in Chapter~\ref{ch:main-results} in a rather self-contained form.
As the title suggests, examples are the heart of this thesis.
Chapter~\ref{ch:examples} presents some 18 examples describing our main results.
In fact, we believe that a careful read renders some, or all, of the proofs of the main results,
presented in Chapter~\ref{ch:main-proofs}, unnecessary.
Chapter~\ref{ch:koszul-bpf} relates our construction to the results of \citet{CGZ-00}.
We take another stand point, though, outlining a template for such a proof in
Section~\ref{sec:template-proof},
and then filling-in the details in
Sections~\ref{sec:moving-quadrics} and \ref{sec:moving-planes-quadrics}.

We consider was conceived as a work in computational algebraic geometry.
As such we field-test our approach and in Chapter~\ref{ch:fast-method}
we develop and implement, in Macaulay2, \citet{M2},
algorithms for finding $N$ and $P$.
We compare computation time against the standard techniques and
show how one can use geometric insight to leverage computation time.

%% STRIP BEGIN
%% BIBLIOGRAPHY
\bibliographystyle{unsrtnat}
\bibliography{../lib/refs}

\end{document}