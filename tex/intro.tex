\documentclass[fleqn,reqno]{amsart}
\usepackage{../lib/radoslav-macro}
\usepackage{../lib/radoslav-more}
\usepackage{times}
\usepackage{natbib}


\newcounter{chapter}
\setcounter{chapter}{1}
\numberwithin{first}{chapter}


\begin{document}
%% STRIP END

Implicitization is an old problem in algebraic geometry,
which asks for the equations of the closed image in of a rational map
given by sections sections of a line bundle on the source.
A solution via elimination has been known since the 19th century.
Today, elimination can generally be carried out through a Gr\"obner basis calculation.
However, this entails has two drawbacks.
First, such an approach is a black box with respect to geometry.
Second, this method is computationally heavy and because unfeasible for even reasonably
small examples.

In the 1990s, with the advance of Computer Aided Design (CAD) software,
the problem was brought back to life after a SIGRAPH paper by Soederber and Chen,
[SIGRAPH Sederber and Chen].
They suggested, or rather showed empirically, how to find the implicit equation 
of the parametrization of curve in $\PP^2$, or of a surface in $\PP^3$.
The first mathematical results in terms of representation matrices ---
matrices which capture syzygies on the coordinates the rational map ---
came around the early 2000s by as series of papers, most notably \citet{CGZ-00}.
Starting with \citet{BJ-03}, another approach was suggested which departed from
linear and commutative algebra and brought the picture to algebraic geometry.
Since then many iterations and strengthenings of the those have been published.





%% STRIP BEGIN

\end{document}
