\documentclass[fleqn,reqno]{amsart}
\usepackage{../lib/radoslav-macro}
\usepackage{../lib/radoslav-more}
\usepackage{times}
\usepackage{natbib}

\begin{document}
%% STRIP END



\begin{paragraf*}
The goal of this chapter is to set up the notation and recall some basic facts
which we shall need later on.
Section \ref{sec:notation} defines the main objects of interest and establishes some
standard notation and terminology.
Section \ref{sec:strands} is a short and elementary treatment on strands of maps of free modules
in the way we need them.
Section \ref{sec:mult} is devoted to the notion of multiplicity.
It comes in many flavors and some care is needed when dealing with it.
\end{paragraf*}

\begin{paragraf*}
In Section~\ref{sec:def-examples},
we present two examples with the only purpose of working out
the definitions of the first three sections.
\end{paragraf*}



\section{Notation}
\label{sec:notation}

\begin{paragraf}
\label{par:cox-ring}
Let $X$ be a non-degenerate toric variety of dimension $n-1$.
We shall not entertain our readers in defining what a toric variety is,
but rather point them to \citet{CLSh11}.

Throughout the thesis,
we work exclusively with the Cox ring $S$ of $X$, \citet{Cox-93}.
Let $\Delta$ be the fan of $X$ and
$\Delta(1)=\{\rho_1,\ldots,\rho_m\}\se\Delta$ be the set of 1-dimensional cones.
We take
\[
	S=\CC[s_1,\ldots,s_m]
\]
with the grading induce by
\[
	0\To\ZZ^{n-1}\cong\mathbf M\To\ZZ^m\To\Pic(X)\To0
\]
A monomial $\mathbf s^\alpha$ corresponds to the divisor $\sum_k \alpha_k[\rho_k]$.
While this suffice to define the grading on $S$,
we need to define an analogue to the irrelevant ideal for $\PP^n$,
in order to recover $X$.
We define
\[
	\n=\la \prod_{\rho_k^\perp}s_k\not\in\sigma~:~\sigma\in\Delta\ra
\]
\end{paragraf}

\begin{paragraf}
From now on we use the notation $\Proj(S,\mathfrak n)$ for the construction of a projective toric variety from
its Cox ring. In the cases where the irrelevant ideal $\mathfrak n$ is the usual one,
for example for products of projective spaces, or is not relevant to the discussion,
we simply write $\Proj(S)$.
This is within reason because the construction coincides with the usual construction
in case $S$ is standardly graded.
\end{paragraf}

\begin{paragraf}
\label{par:setup}
Let $X$ be a smooth projective toric variety of dimension $n-1$ ($n>1$).
Let $S$ be its Cox ring \eqref{par:cox-ring} and
let $S'$ be any homogeneous coordinate ring
furnishing an embedding of $X$ into projective space, i.e. $X=\Proj(S')$ with
\[
	S'=\CC[s_0,\ldots,s_m]/\mathfrak{p}
\]
for some homogeneous prime $\mathfrak{p}$ of height $m-n+1$.
Let $T=\CC[\mathbf x]$ be the homogeneous coordinate ring of $\PP^n$.

Let $\L=\O_X(\e)$ be a line bundle on $X$ such that $h^0(\L)> n$.
Note that $\e$ is a degree on $S$.

Under the identification above, suppose that
\[
\phi_0,\ldots,\phi_n\in H^0(X,\L)=S_{\mathbf e}
\]
are linearly independent and consider the rational map
\[
	\phi=(\phi_0,\ldots,\phi_n):X\To\PP^n=\Proj(T)
\]

We denote by $J$ the ideal of the coordinates $\phi_j$,
\[
	J=\la\phi_0,\ldots,\phi_n\ra\se S
\]

Let $Y$ denote the closed image of $\phi$.
Since $X$ is reduced and irreducible, so is $Y$ and, in particular,
$Y=V(P)$ for some homogeneous prime ideal $P\se T$,
\[
	Y=\image(\phi)=V(P)
\]

If $\dim(Y)=n-1$, then $P$ is principal.
In that case,
if we want to refer to a generator of $P$, necessarily up to a unit in $T$,
we shall dereference the ideal by writing $P(\mathbf x)$.

We sometimes refer to $S$ as the coordinate ring of \underbar{s}ource and
to $T$ as the coordinate ring of the \underbar{t}arget.
\end{paragraf}

\begin{paragraf}
Throughout this thesis, the term {\em image} of a rational map, as used in \eqref{par:setup},
means the scheme-theoretic image, also called the closed image. Formally,
\[
	Y = V(\ker\phi^\#),\quad \phi^\#:\O_{\PP^n}\To\phi_\ast\O_X
\]
In our situation, this is just the closure of the set-theoretic map on closed points.
\end{paragraf}

\begin{paragraf}
\label{par:blowup-algebras}
Let $R=S\tensor T=S[\mathbf x]$. Then $R$ is naturally bigraded by
\[
	\deg(af)=(\mathbf d,i)
\]
whenever $a\in S_{\mathbf d}\se R$ and $f\in T_i\se R$.
Let $S[t]$ be similarly graded, setting $\deg(t)=(-\mathbf e,1)$.
The blow-up algebras, $\Rees_S(J)$ and $\Sym_S(J)$, naturally become
factor rings of $R$ as follows.

The Rees algebra is the image of the bigraded map of $S$-algebras
\[
	\beta:R\To S[t]:\quad x_j\mapsto\phi_j\cdot t
\]

The Rees ideal of $J$ is the bigraded ideal $I=\ker(\beta)\se R$.
It is generated by the the polynomial relations on the generators $\phi_0,\ldots,\phi_n$ of $J$,
that is,
\[
	I=\la \textstyle\sum_{|\alpha|=i} a(\mathbf s){\mathbf x}^{\alpha}
	~\boldsymbol:~
	\textstyle\sum_{|\alpha|=i} a(\mathbf s){\mathbf \phi}^{\alpha}=0
	~\boldsymbol;~\forall i\ra\se R
\]
For the sake of brevity, we denote the Rees algebra by $B$.

While $\Rees_S(J)$ captures all polynomial relations on the $\phi_j$,
the symmetric algebra $\Sym_S(J)$ captures only the linearly-generated ones.
Specifically, we have
\[
	\Sym_S(J)=R/\la \textstyle\sum_j a(\mathbf s)x_j
	~\boldsymbol:~
	\textstyle\sum_j a(\mathbf s)\phi_j=0
	\ra
\]

We shall sometimes refer to $R$ as the ambient ring of the blow-up algebras.
\end{paragraf}

\begin{paragraf}
\label{par:the-graph}
The ring $R=S[\mathbf x]$ and the blow-up algebras just defined are in essence geometric objects.
As $\CC$-algebras, $R$ corresponds to the bihomogeneous coordinate ring of the product of
the source and the target varieties,
and $B=\Rees_S(J)$ --- to the coordinate ring of the graph of $\phi$, $\Gamma(\phi)$,
defined (e.g. \citet{Har92}) as the closure of
\[
	\{(q,\phi(q)):q\in X\}\se X\times \PP^n
\]

The natural surjective morphisms $R\To \Sym_S(J)\To \Rees_S(J)$ induce natural closed embeddings
\begin{align}
\label{eq:R-Sym-B}
\Gamma(\phi)=\Biproj(\Rees_S(J))\To \Biproj(\Sym_S(J))\To \Biproj(R)=X\times \PP^n
\end{align}
\end{paragraf}

\begin{paragraf}
\label{par:base-locus}
Denote the subscheme $V(J)\se X$ by $Z$.
Then $Z$ is the base locus of the rational map.
Geometrically, $\Biproj(B)$ is the blow-up of the variety $X$ along the closed $Z$.
See Lemma~\ref{lemma:lengthRees} for details.
\end{paragraf}

\begin{paragraf}
\label{par:ci-points}
Let $X$ be a variety and $q\in X$ be a smooth point, that is,
such that the stalk $\O_q$ is regular local.
Let $Z\se X$ be a closed subscheme containing the point $q$.
We say that $q$ is a complete intersection (c.i.) point of $Z$
if the stalk of $q$ on $Z$, $\O_{q,Z}$,
is a complete intersection factor ring of the regular local $\O_q$.
Now suppose $Z\se X$ is of codimension $d$ at $q$
and that the stalk $O_{q,Z}$ can be defined from $\O_q$ by $d+1$ elements.
In this case, we say that $q$ is an almost complete intersection (a.c.i.)
point of $Z$.
\end{paragraf}

\begin{paragraf}
\label{par:linear-type}
Suppose that the base locus of the rational map $\phi$, $V(J)$, is zero-dimensional.
The first embedding in \eqref{eq:R-Sym-B} is an isomorphism if and only if
$V(J)$ is a locally complete intersection scheme.
For a proof in our setting, see (\citet{BJ-03}, Proposition 4.14).
\end{paragraf}

\begin{paragraf}
\label{lemma:push-gens}
The ideal $I$ is bigraded in $R$, so its $S$-graded pieces are finite $T$-modules.
We denote the graded piece in degree $\d$ on $S$ by $I_{\d,\bullet}$,
and sometimes call it a ($T$-)strand of $I$.

More generally, let $\mathbf M$ be a finite bigraded $R$-module
generated by some finite set $\{h_\ell:\ell\}$.
Let $\d$ be any degree on $S$.
Setting $({\mathbf a}_\ell,i_\ell)=\deg(h_\ell)$, one has
\[
	\displaystyle\sum_{\ell,~{\mathbf b}_\ell~:~{\mathbf a}_\ell+{\mathbf b}_\ell=\d}
	(R_{{\mathbf b}_\ell,\bullet}) h_\ell={\mathbf M}_{\d,\bullet}
\]
so
\[
	\displaystyle\bigcup_{\ell}~\{\mathbf s_k\cdot h_\ell:\Span_\CC\{\mathbf s_k:k\}=
	S_{{\mathbf b}_\ell}\}
\]
is finite and a $T$-module generating set for $\mathbf M_\d$.
\end{paragraf}

\begin{paragraf}
\label{par:basis}
Throughout this thesis we work with a fixed monomial order on $S$ and $T$.
For example, this could be the graded lexicographic order but the specific choice
is not important.
With this agreement,
the statement:
let $\mathbf b$ be a row-vector corresponding to the basis of $S_{\mathbf d}$,
means a row vector having having as its coordinates the monomials of $S$,
listed in the order.

In this sense, we write
\[
\mt{basis}(S_{\mathbf d}),\quad\mt{basis}(T_i),\quad\mt{basis}(R_{\mathbf d,i})
\]
for the row vectors consisting of monomial bases in the selected order for the $\CC$-vector spaces
$S_{\mathbf d}, T_i$  and $R_{\mathbf d,i}$, respectively.
\end{paragraf}

\begin{paragraf}
\label{par:syzygies}
Let $r=\dim_\CC(S_{\mathbf d})$ and
let $\mathbf b=\mt{basis}(S_{\mathbf d})$.
We mostly use $r$ instead of $\dim_\CC(S_{\mathbf d})$ when $\mathbf d$ is fixed.

Given a form $g(\mathbf x)=g(\mathbf s;\mathbf x)\in I_{\mathbf d,i}$,
that is, a syzygy of degree $i$ over $S_{\mathbf d}$,
we can write $g(\mathbf x)$ as
\[
g(\mathbf x)=\mathbf b\cdot C=
\begin{bmatrix}
	x_0^i& x_0^{i-1}x_1& \ldots& x_n^i
\end{bmatrix}\cdot C'
\]
where $C$ is an $r\times1$ column vector with entries in $T_i$,
and $C'$ is a $\binom{n+i}{n}\times1$ column vector with entries in $S_{\mathbf d}$.

We use the term syzygy both for the column vector $C$ and the form $g$ in the
Rees ideal.
\end{paragraf}

\begin{paragraf}
\label{par:N-matrix}
We now come to the most important bit of notation.
Let us fix a degree $\mathbf d$ on the source.
As already apparent from \eqref{par:syzygies},
identifying the generators of $I_{\d,\bullet}$ with column-$r$-vectors,
$I_{\d,\bullet}$ becomes a sub-$T$-module of the free graded $T^r$.

Let $C_1,\ldots,C_\mu$ be a minimal generating set for $I_{\d,\bullet}$.
We take $N$ to be the matrix with columns $C_1,\ldots,C_\mu$.
Equivalently, let $N$ be the the representation matrix for a minimal
generating set of $I_{\d,\bullet}$ with respect to $\basis(S_\d)$.

The matrix $N$ a graded $T$-linear map
\[
N:\bigoplus_{k} T(-i_k)\To T^r
\]
whose image is just $I_{\d,i}$.

By grouping the columns corresponding to the same degree $i$ into submatrices $N_i$,
for each valid $i$ we get a single matrix whose columns are the degree $i$-syzygies over $S_{\mathbf d}$.
Clearly, $N_i$ is empty for any $i$ larger than the maximum degree $\delta$ of a minimal generator
of $I_{\mathbf d,\bullet}$.
By the assumption on the linear independence of the $\phi_j$, $N_0$ is empty too.
In any case, the $N_i$ fit together to give $N$,
\[
	N=(N_1~|~N_2~|~\ldots~|~N_\delta)
\]

In Chapter 7 we describe the close connection of $N_1$ and $N_2$ to the matrices used in
the moving planes and quadrics results.

Finally, whenever useful, we write $h_i$ for the number columns in $N_i$
and $h$ for the tuple $(h_1,\ldots,h_\delta)$.
\end{paragraf}

\begin{paragraf}
Recall that for any ideal $Q$ in a unique factorization domain $T$,
$\gcd(Q)$ is defined to be the unique minimal principal ideal which contains $Q$,
and this definition obviously generalizes the definition on elements
when $T$ is a Euclidean domain.
\end{paragraf}

\begin{paragraf}
As a final bit of notation, we mention that we use $\rad(-)$ for the radial,
and $\sat(-)$ for the saturation with respect to the irrelevant ideal.
\end{paragraf}



\section{Strands of Module Maps}
\label{sec:strands}

\begin{paragraf}
\label{par:all-phi-k}
Let us consider the coordinates of $\phi$ as a row vector over $S$. We get a graded $S$-linear map
\[
\begin{bmatrix}
	\phi_0& \phi_1& \ldots& \phi_n
\end{bmatrix}:S^{n+1}\To S^1(\e)
\]
where $S(\e)$ has the usual meaning of putting $1\in S$ in degree $-\e$.
Similarly, we can consider the graded $S$-linear map given by the quadratic monomials
of the coordinates,
\[
\begin{bmatrix}
	\phi_0^2& \phi_0\phi_1& \ldots& \phi_n^2
\end{bmatrix}:S^{(n+2)(n+1)/2}\To S^1(2\mathbf e)
\]

These two maps have a central role in the methods of moving planes and quadrics.
However, there is no reason to stop at degree $2$,
so next we describe the general situation.
\end{paragraf}

\begin{paragraf}
\label{par:strands}
Let $k$ be a positive integer and $\mathbf d$ be a fixed degree on $S$ such that $S_\d\neq0$.
Define $\phi^{(k)}$ to be the graded $S$-linear map formed by the coordinates of $\phi$,
\[
\phi^{(k)}=\begin{bmatrix}\phi_0^k& \phi_0^{k-1}\phi_1& \cdots& \phi_n^k\end{bmatrix}:
S^{\binom{n+k}{n}}\To S^1(k\mathbf e)
\]
and set $\Phi^{(k)}$ be the linearization of $\phi^{(k)}$ in degree $\d$, that is,
\[
\Phi^{(k)}:S_{\mathbf d}^{\binom{n+k}{n}}\To S_{k\mathbf e+\mathbf d}^1
\]
as a map of complex vector spaces.

Choosing bases \eqref{par:syzygies}, we can think of $\Phi^{(k)}$ as a matrix over $\CC$ of size
\[
\dim_\CC(S_{k\mathbf e+\mathbf d})\times r\binom{n+k}{n}
\]
whose columns can be indexed by the monomials in $R_{\mathbf d,k}$.
\end{paragraf}

\begin{paragraf}
\label{par:syzygies-kernel}
The advantage of the matrices $\Phi^{(k)}$ over the matrices $\phi^{(k)}$
is that the kernel of $\Phi^{(i)}$ corresponds directly to the degree-$i$ syzygies over $S_{\mathbf d}$.
That is, for a fixed $\mathbf d$,
\[
\mathbf v\mapsto\mt{basis}(R_{\mathbf d,i})\cdot \mathbf v:\ker(\Phi^{(i)})\To I_{\mathbf d, i}
\]
is an isomorphism of vector spaces.
\end{paragraf}



\section{Multiplicity}
\label{sec:mult}

\begin{paragraf}
\label{par:mult-ha}
Recall the following notation from \cite{Har77}.
For a homogeneous prime ideal $P$ in a graded ring $T$,
we set $T_{(P)}$ to be the degree-$0$ graded piece of the localization of $T$
at the homogeneous elements outside of $P$.
If $P$ is a minimal prime of a graded $T$-module $\mathbf M$,
we denote by $\mult_P(\mathbf M)$ the length of the $T_P$-module $\mathbf M_P$
(see {\em loc. cit.}, I, Proposition 7.4).
\end{paragraf}

\begin{paragraf}
\label{par:mult}
Let $Z$ be a zero-dimensional closed subscheme of a smooth projective variety $X$.
Let $\mathscr{J}\se\mathscr{O}_X$ be the ideal sheaf of $Z$
and let $q\in Z$ be any point.
Since $Z$ is zero-dimensional,
$\mathscr{J}_q$ is an ideal of definition
in the regular local ring $\mathscr{O}_{q,X}$.

Define the multiplicity of $Z$ at $q$, denoted $e_q$,
to be the Hilbert-Samuel multiplicity of $\mathscr{J}_q$,
denoted $e(\mathscr{J}_q,\O_q)$ (see \citet{Eis95} or \citet{BH98}).

Define the degree of $Z$ at $q$, denoted $d_q$, to be the length of the local ring,
\[	
	\length(\mathscr{O}_{q,Z})=\dim_{\CC}(\mathscr{O}_{q,Z})
\]

We have that $d_q\leq e_q$ with equality if and only if $q$ is a complete intersection point.
\end{paragraf}

\begin{paragraf}
We shall mostly be interested in the degree and multiplicity of points on the base locus $Z$.
Since by assumption $X$ is toric, the ideal sheaf $\mathscr{J}$ of $Z$ in \eqref{par:mult}
is the ideal sheaf $\widetilde J$.
We stick to the latter notation for the rest of the paper.
\end{paragraf}

\begin{paragraf}
Let $\L$ be a line bundle on $X$.
We denote by $[\L]$ the class of $\L$ in the Chow ring.
Since $\dim(X)=n-1$, we can identify $[\L]^{n-1}$ with an integer --- its degree.
Suppose that the base locus $Z$ is zero-dimensional.
Then by (\citet{Ful84}, Proposition 4.4), see \citet{Cox-01} for details,
we have the formula
\begin{align}
	\label{eq:degree-formula}
	\deg(\phi)\deg(Y)=[\mathscr L]^{n-1}-\sum_{q\in Z}e(\widetilde{J}_q,\O_q)
\end{align}
\end{paragraf}

\begin{paragraf}
\label{par:self-intersection}
The self-intersection $[\mathscr L]^{n-1}$ is obvious when $X=\PP^{n-1}$.
Then $\mathscr L=\O(d)$ for some integer $d$, and
\[
	[\O(d)]^{n-1}=d^{n-1}
\]
Similarly, let $X=(\PP^1)^{n-1}$. Then $\L=\O(e_1,\ldots,e_{n-1})$ for integers $e_k$,
and
\begin{align}
	\label{eq:formula-prod-P1}
	[\O(e_1,\ldots,e_{n-1})]^{n-1}=(n-1)!\cdot e_1\cdots e_{n-1}
\end{align}
The formula above can be easily proved by remembering that the rulings of $X$ have
self-intersection zero,
so the only nonzero term in the power of $[\L]$ in the Chow ring, is the multiplication
of all rulings.
\end{paragraf}

\begin{paragraf*}
See Example~\ref{ex201} and Example~\ref{ex311} for more information.
\end{paragraf*}


\section{Examples}
\label{sec:def-examples}

\begin{example}[$\mt{ex201}$]
\label{ex201}
Let $X=\PP^1_{s,u}\times\PP^1_{t,v}$ be the product of two projective lines,
and let $S=\CC[s,u;t,v]$ be its Cox ring --- a bihomogeneous ring graded by $\Pic(X)$
such that $\deg(s)=\deg(u)=(1,0)$ and $\deg(t)=\deg(v)=(0,1)$.
The irrelevant ideal $\n$ is the product of the irrelevant ideals of the factors,
\[
	\n=\la s,u\ra\cap\la t,v\ra=\la st,sv,ut,uv\ra
\]
We write $X=\Proj(S,\n)$ or simply $X=\Proj(S)$ for the construction in the toric setting.

Consider the rational map $\phi:X\To\PP^3$ given by
\[
	\phi=(s^{2} v^{2},s u v^{2},u^{2} t^{2}+u^{2} t v,s u t v-101 u^{2}t v)
\]
In this case, the coordinates $\phi_0,\ldots,\phi_3$ are global sections
of the line bundle $\O_X(2,2)$, and are linearly independent over the base field $\CC$.
Note that
\[
	h^0(\O_X(2,2))=\dim_\CC(S_{(2,2)})=9
\]

The ideal of the coordinates,
\[
	J=\la s^{2} v^{2},s u v^{2},u^{2} t^{2}+u^{2} t v,s u t v-101 u^{2}t v\ra\se S
\]
defines the base locus, $Z=V(J)$.
In this case, $Z$ is supported on the points
\[
	q_1=(0,1)\times(0,1) \text{ and } q_2=(1,0)\times(1,0)
\]
so the base locus is zero-dimensional and the map is generically finite.

Near $q_1$ the scheme $Z$ looks like $V(s,t)$ in $\AA^2_{s,t}$,
while near $q_2$, $Z$ looks like $V(u^2,uv,v^2)$ in $\AA^2_{u,v}$.
This allows us to compute their degrees and multiplicities.
Let $\O_{q_i,Z}$ denote the stalk at $q_i$ on $Z\se X$ ($i=1,2$),
and let $\widetilde J$ denote the ideal sheaf arising form $J$.
The the degrees are
\[
	d_{q_1}=\length(\O_{q_1,Z})=1 \text{ and } d_{q_2}=\length(\O_{q_2,Z})=3
\]
while the multiplicities are
\[
	e_{q_1}=e(\widetilde{J}_{q_1},\O_{q_1,X})=1 \text{ and } e_{q_2}=e(\widetilde{J}_{q_2},\O_{q_2,X})=4
\]
where $e(\mathbf I,\mathbf R)$,
for a Noetherian local $\mathbf R$ and an ideal of definition $\mathbf I$,
denotes the Hilbert-Samuel multiplicity.

Note further that $d_{q_1}=e_{q_1}$ reflects the fact that $q_1$ is a complete intersection
(c.i.) point,
while $d_{q_2}<e_{q_2}$ so $q_2$ is not.
It is, however, an almost complete intersection (a.c.i.) point,
since it is of codimension-$2$ and is (stalk-, affine-)locally defined by $3$ equations.

The closed image $Y\se\PP^3$ of $\phi$ is given by a single equation,
\[
	P(\mathbf x)={x}_{0}^{2} {x}_{2}-202 {x}_{0} {x}_{1} {x}_{2}+10201
  {x}_{1}^{2} {x}_{2}-{x}_{0} {x}_{1} {x}_{3}+101 {x}_{1}^{2} {x}_{3}-{x}_{0}
  {x}_{3}^{2}
\]
The equation $P(\mathbf x)$, called the implicit equation, show up in the Rees ideal.

Let $B=\Rees_S(J)$ be realized as the quotient ring of $R=S[\mathbf x]$
by the aforementioned Rees ideal $I$.
In our case,
\[
	I=\Bigg\la
	\begin{cases}
	u {x}_{0}-s {x}_{1},
	(s v-101 u v) {x}_{2}+(-u t-u v){x}_{3},\\
	(t^{2}+t v) {x}_{1}-101 v^{2} {x}_{2}+(-t v-v^{2}) {x}_{3},\\
	(s t-101 ut) {x}_{1}-s v {x}_{3},v {x}_{0} {x}_{2}-101 v {x}_{1} {x}_{2}+(-t-v) {x}_{1}{x}_{3},\\
  	t {x}_{0} {x}_{2}-101 t {x}_{1} {x}_{2}-101 v {x}_{2} {x}_{3}+(-t-v){x}_{3}^{2},\\
  	s {x}_{0} {x}_{2}+(-202 s+10201 u) {x}_{1} {x}_{2}+(-s+101 u){x}_{1} {x}_{3}-s {x}_{3}^{2},\\
	t {x}_{0} {x}_{1}-101 t {x}_{1}^{2}-v {x}_{0}{x}_{3},\\P(x_0,x_1,x_2,x_3)
	\end{cases}\Bigg\ra
\]
and $P(\mathbf x)$ can be seen as the unique linear generator in bidegree $((0,0),3)$.
In fact, it is the unique $\CC[\mathbf x]$-generator of $I_{(0,0),\bullet}$.

Generalizing slightly, let $\d=(1,1)$ be a degree on $S$.
Then $I_{\d,\bullet}$ is minimally generated by 5 elements as a $\CC[\mathbf x]$-module.
Let those be $g_1(\mathbf s;\mathbf x),\ldots,g_5(\mathbf s;\mathbf x)$.
Writing
\[
	\basis(S_\d)=\begin{bmatrix}st& sv& ut& uv\end{bmatrix}
\]
to be a row vector over $R$ consisting of a linear basis for $S_\d$,
we can define the matrix $N$ over $\CC[\mathbf x]$ by the identity
\[
	\basis(S_\d)\cdot N=\begin{bmatrix}g_1&\ldots&g_5\end{bmatrix}
\]
In this case, $N$ is a $4\times5$-matrix each of whose columns contains forms of the same degree.
Specifically,
\[
	N=\begin{bmatrix}0&
	       {x}_{1}&
	       0&
	       0&
	       {x}_{0} {x}_{2}-{x}_{3}^{2}\\
	       {x}_{2}&
	       {-{x}_{3}}&
	       {-{x}_{1}}&
	       {-{x}_{3}}&
	       {-{x}_{3}^{2}}\\
	       {-{x}_{3}}&
	       {-101 {x}_{1}}&
	       0&
	       {x}_{0}-101 {x}_{1}&
	       -10201 {x}_{1} {x}_{2}-202 {x}_{3}^{2}\\
	       -101 {x}_{2}-{x}_{3}&
	       0&
	       {x}_{0}&
	       0&
	       20402 {x}_{2} {x}_{3}-202 {x}_{3}^{2}\\
	       \end{bmatrix}
\]

In this sense, $N$ is a representation matrix as well as a matrix of syzygies.
It represents a generating set for the graded syzygies on the $\phi_j$ with coefficients
in $S_{(1,1)}$.
\end{example}

\begin{example}[$\mt{ex202}$]
\label{ex202}
Let $X=\PP^2$.
Its Cox ring is just its standard homogeneous coordinate ring, $\CC[s,t,u]$,
with the irrelevant ideal $\n=\la s,t,u\ra$.

Let $J\se S$ be a graded ideal generated by $4$ linearly independent forms of the same degree $\e$.
For concreteness we can take
\[
	J=\la s^3,t^2u,s^2t+u^3,stu\ra
\]
so that its generators are all $3$-forms.
We can think of $J$ as defining a rational map to $\PP^3$,
\[
	\phi=(s^3,t^2u,s^2t+u^3,stu):X\To\PP^3
\]

Then $\phi$ is a morphism on the open set away from $Z=V(J)$.
Note that $J$ is not saturated, so there is a better representative for the base locus scheme,
i.e.
\[
\sat(J)=J:\n^\infty=\la s^2,u\ra
\]
From this we know that $Z$ is supported on $q=(0,1,0)$ only,
and that $q$ is a complete intersection point, so its degree and multiplicity coincide,
\[
	e_q=d_q=\length(\O_q/\widetilde{J}_q)=2
\]

Let $P$ be the principal ideal generated by the implicit equation for $\phi$.
By the formulas in Section~\ref{sec:mult}, we have
\[
	\deg(\phi)\deg(P)=3^2-e_q=7
\]
Since the coordinates $\phi_j$ are not linearly dependent,
and they never will be in this thesis,
$\deg(P)\neq1$, so $\deg(\phi)=1$ and the map is generically 1-1.
We then have that the implicit equation is a septic.
Indeed,
\[
	P(\mathbf x)={x}_{0}^{3} {x}_{1}^{4}-{x}_{0}^{2} {x}_{1}^{3} {x}_{2} {x}_{3}+{x}_{3}^{7}
\]

Besides a rational map, or a ring map in the form of $\phi^\#$,
$\phi$ can be made into a map of free $S$-modules.
For example, we have that
\[
	\phi^{(1)}=\begin{bmatrix}\phi_0&\ldots&\phi_3\end{bmatrix}:S^4\To S^1(3)
\]
is a graded map of $S$-modules.

We can linearize this map in this map in the following way.
Take a degree on $S$, for concreteness, take $\d=1$.
We take a the strand of the map $\phi^{(1)}$ in degree $\d=1$.
To this end,
note that the basis on the source of $\phi^{(1)}$ can be indexed by the monomials in $R_{1,1}$,
for example, by putting
\[
	\begin{bmatrix}0\\s\\0\\0\end{bmatrix}=sx_1
\]
and the basis on the target can be indexed by the monomials in $S_4$.
For the linearization, we get
\[
	\Phi^{(1)}=
	\bgroup\makeatletter\c@MaxMatrixCols=12\makeatother\begin{bmatrix}1&
      0&
      0&
      0&
      0&
      0&
      0&
      0&
      0&
      0&
      0&
      0\\
      0&
      0&
      1&
      0&
      1&
      0&
      0&
      0&
      0&
      0&
      0&
      0\\
      0&
      0&
      0&
      0&
      0&
      0&
      0&
      0&
      1&
      0&
      0&
      0\\
      0&
      0&
      0&
      0&
      0&
      0&
      1&
      0&
      0&
      0&
      0&
      0\\
      0&
      0&
      0&
      1&
      0&
      0&
      0&
      0&
      0&
      0&
      1&
      0\\
      0&
      0&
      0&
      0&
      0&
      0&
      0&
      0&
      0&
      0&
      0&
      0\\
      0&
      0&
      0&
      0&
      0&
      0&
      0&
      0&
      0&
      0&
      0&
      0\\
      0&
      1&
      0&
      0&
      0&
      0&
      0&
      1&
      0&
      0&
      0&
      0\\
      0&
      0&
      0&
      0&
      0&
      0&
      0&
      0&
      0&
      0&
      0&
      1\\
      0&
      0&
      1&
      0&
      0&
      0&
      0&
      0&
      0&
      0&
      0&
      0\\
      0&
      0&
      0&
      0&
      0&
      0&
      0&
      0&
      0&
      0&
      0&
      0\\
      0&
      0&
      0&
      0&
      0&
      1&
      0&
      0&
      0&
      0&
      0&
      0\\
      0&
      0&
      0&
      0&
      0&
      0&
      0&
      0&
      0&
      1&
      0&
      0\\
      0&
      0&
      0&
      0&
      0&
      0&
      1&
      0&
      0&
      0&
      0&
      0\\
      0&
      0&
      0&
      0&
      0&
      0&
      0&
      0&
      0&
      0&
      1&
      0\\
      \end{bmatrix}\egroup
\]

We end this example noting that $\ker(\Phi^{(1)})$ is isomorphic to $I_{1,1}$.
In our situation this is easy to check.
The kernel is spanned by
\[
	\mathbf v=\bgroup\makeatletter\c@MaxMatrixCols=12\makeatother\begin{bmatrix}0&
	      {-1}&
	      0&
	      0&
	      0&
	      0&
	      0&
	      1&
	      0&
	      0&
	      0&
	      0\\
	      \end{bmatrix}\egroup\T
\]
which corresponds to $I_{1,1}=\Span_\CC(sx_1-tx_3)$
after multiplication with the indexing set for the columns,
\[
	\basis(R_{1,1})=\bgroup\makeatletter\c@MaxMatrixCols=12\makeatother\begin{bmatrix}s {x}_{0}&
      s {x}_{1}&
      s {x}_{2}&
      s {x}_{3}&
      t {x}_{0}&
      t {x}_{1}&
      t {x}_{2}&
      t {x}_{3}&
      u {x}_{0}&
      u {x}_{1}&
      u {x}_{2}&
      u {x}_{3}\\
      \end{bmatrix}\egroup
\]
\end{example}



%% STRIP BEGIN
%% BIBLIOGRAPHY
\bibliographystyle{unsrtnat}
\bibliography{../lib/refs}

\end{document}