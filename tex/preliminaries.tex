\documentclass[fleqn,reqno]{amsart}
\usepackage{../lib/radoslav-macro}
\usepackage{../lib/radoslav-more}
\usepackage{times}
\usepackage{natbib}



\newcounter{chapter}
\setcounter{chapter}{1}
\numberwithin{first}{chapter}



\begin{document}
%% STRIP END

\begin{paragraf}
\label{par:preliminaries-intro}
The goal of this chapter is to set up the notation and recall some basic facts
which we shall need later on.
The first section defines the main objects of interest and establishes some
standard notation and terminology.
Section 2 is a short and elementary treatment on strands of maps of free modules
in the way we shall need them.
Section 3 is solely devoted to the notion of multiplicity.
It comes in many flavors and some care is needed when dealing with it.

Because notation is somewhat heavy and appears rather unmotivated at first,
we present the gist of the first three sections through few examples in the last fourth section.
\end{paragraf}



\section{Notation}

\begin{paragraf}
\label{par:toric-variety}
Define what a projective toric variety is.
\end{paragraf}

\begin{paragraf}
\label{par:cox-ring}
Define the Cox ring of $X$ as in \citet{93-Cox-ring}.
\end{paragraf}

\begin{paragraf}
From now on we use the notation $\Proj(S,\mathfrak n)$ for the construction of a projective toric variety from
its Cox ring. In the cases where the irrelevant ideal $\mathfrak n$ is the usual one,
for example for products of projective spaces, or is not relevant for the discussion,
we shall simply write $\Proj(S)$.
This is feasible because the construction coincides with the usual construction
in case $S$ is a quotient of a standard graded polynomial ring.
\end{paragraf}

\begin{paragraf}
\label{par:setup}
Let $X$ be a smooth projective toric variety of dimension $n-1$ ($n>1$).
Let $S$ be its Cox ring \eqref{par:cox-ring} and
let $S'$ be any homogeneous coordinate ring
furnishing an embedding of $X$ into projective space, i.e. $X=\Proj(S')$ with
\[
	S'=\CC[s_0,\ldots,s_m]/\mathfrak{p}
\]
for some homogeneous prime $\mathfrak{p}$ of height $m-n+1$.
Let $T=\CC[\mathbf x]$ be the homogeneous coordinate ring of $\PP^n$.

Let $\L$ be a line bundle on $X$ such that $h^0(\L)> n$.
Because $X$ is toric,
there is a degree $\mathbf e$ on $S$ such that $\L\cong\O_X(\mathbf e)$.
We shall sometimes write $\mathbf e\in\Pic(X)$.

Under the identification above, suppose that
\[
\phi_0,\ldots,\phi_n\in H^0(X,\L)=S_{\mathbf e}
\]
are linearly independent and consider the rational map
\[
	\phi=(\phi_0,\ldots,\phi_n):X\To\PP^n=\Proj(T)
\]

We denote by $J$ the ideal of the coordinates $\phi_j$,
\[
	J=\la\phi_0,\ldots,\phi_n\ra\se S
\]

We assume from now on that the closed image of $\phi$, which we denote by $Y$,
is of dimension $n-1$.
Since $X$ is reduced and irreducible, so is $Y$ and, in particular,
$Y=V(P)$ for some principal homogeneous prime ideal $P\se T$,
\[
	Y=\image(\phi)=V(P)
\]

If we want to refer to a generator of $P$, necessarily up to a unit in $T$,
we shall dereference the ideal by writing $P(\mathbf x)$.

We refer to $S$ as the Cox ring of \underbar{s}ource, to $S'$ as its homogeneous coordinate ring,
and to $T$ as the homogeneous coordinate ring of the \underbar{t}arget.
\end{paragraf}

\begin{remark}
Throughout this thesis, the term {\em image of a rational map}, as used in \eqref{par:setup},
means the scheme-theoretic image, also called the closed image. Formally,
\[
	Y = V(\ker\phi^\#),\quad \phi^\#:\O_{\PP^n}\To\phi_\ast\O_X
\]
In our situation,  is just the closure of the set-theoretic map on closed points of $X$.
\end{remark}

\begin{paragraf}
\label{par:blowup-algebras}
Let $R=S\tensor T=S[\mathbf x]$. Then $R$ is naturally bigraded by
\[
	\deg(af)=(\mathbf d,i)
\]
whenever $a\in S_{\mathbf d}\se R$ and $f\in T_i\se R$.
Let $S[t]$ be similarly graded, setting $\deg(t)=(-\mathbf e,1)$.
The blow-up algebras, $\Rees_S(J)$ and $\Sym_S(J)$, naturally become
factor rings of $R$ as follows.

The Rees algebra is the image of the bigraded map of $S$-algebras
\[
	\beta:R\To S[t]:\quad x_j\mapsto\phi_j\cdot t
\]

We set $I=\ker(\beta)\se R$ to be the  Rees ideal.
It is the bigraded ideal of $R$ describing the polynomial relations between the
selected set of generators $\phi_0,\ldots,\phi_n$ for $J$,
that is,
\[
	I=\la \textstyle\sum_{|\alpha|=i} a(\mathbf s){\mathbf x}^{\alpha}
	~\boldsymbol:~
	\textstyle\sum_{|\alpha|=i} a(\mathbf s){\mathbf \phi}^{\alpha}=0
	~\boldsymbol;~\forall i\ra\se R
\]
For the sake of brevity, we set $B=\Rees_S(J)$ as an $S$-algebra.

While $\Rees_S(J)$ captures all polynomial relations on the $\phi_j$,
the symmetric algebra $\Sym_S(J)$ captures only the linear ones.
Specifically, we have
\[
	\Sym_S(J)=R/\la \textstyle\sum_j a(\mathbf s)x_j
	~\boldsymbol:~
	\textstyle\sum_j a(\mathbf s)\phi_j=0
	\ra
\]

We shall sometimes refer to $R$ as the {\em ambient} ring of the blow-up algebras.
\end{paragraf}

\begin{paragraf}
\label{par:the-graph}
The ambient ring $R=S[\mathbf x]$ and the blow-up algebras just defined are in essence geometric objects.
As $\CC$-algebras, $R$ corresponds to the bihomogeneous coordinate ring of the product of
the source and the target varieties,
and $B=\Rees_S(J)$ --- to the coordinate ring of the graph of $\phi$, $\Gamma(\phi)$ \ref{Harris},
defined as the closure of
\[
	\{(q,\phi(q)):q\in X\}\se X\times \PP^n
\]

The natural surjective morphisms $R\To \Sym_S(J)\To \Rees_S(J)$ induce natural closed embeddings
\begin{align}
\label{eq:R-Sym-B}
\Gamma(\phi)=\Biproj(\Rees_S(J))\To \Biproj(\Sym_S(J))\To \Biproj(R)=X\times \PP^n
\end{align}
\end{paragraf}

\begin{paragraf}
\label{par:ci-points}
Let $X$ be a variety and $q\in X$ be a smooth point, that is,
such that the stalk $\O_q$ is regular local.
Let $Z\se X$ be a closed subscheme containing the point $q$.
We say that $q$ is a complete intersection (c.i.) point of $Z$
if the stalk of $q$ on $Z$, $\O_{q,Z}$,
is a complete intersection factor ring of the regular local $\O_q$.
Now suppose $Z\se X$ is of codimension $d$ at $q$
and that the stalk $O_{q,Z}$ can be defined out of $\O_q$ by $d+1$ elements.
In this case, we say that $q$ is an almost complete intersection (a.c.i.)
point of $Z$.

Note that the notion of a complete intersection ring does not require
an ambient regular local ring but we shall not need this fact here.
\end{paragraf}

\begin{paragraf}
\label{par:linear-type}
Suppose that the base locus of the rational map $\phi$, $V(J)$, is zero-dimensional.
Then it is known that the first embedding in \eqref{eq:R-Sym-B} is an isomorphism
if and only if $V(J)$ is a locally complete intersection scheme (e.g. \citet{BJ-JA-03}, Proposition 4.14),
that is, all its points are c.i. points.

The two equivalent conditions on $J$ are sometimes expressed by saying that
$J$ is of linear type. We shall not use this notation.
\end{paragraf}

\begin{paragraf}
\label{par:finite-strand}
Before we take the interpretation of the ideal $I$ as
the homogeneous polynomial syzygies on the $\phi_j$ any further,
we note that its graded pieces for a fixed degree $\mathbf d$ on the source
are finite as $T$-modules.

More generally, let $M$ be any finite graded $R$-module.
We denote by
\[
M_{\mathbf d,\bullet}=\bigoplus_{i\in\ZZ} M_{\mathbf d,i}
\]
the graded piece of $M$ in degree $\mathbf d$ as a $T$-module, and note that
\[
M=\bigoplus_{\mathbf d} M_{\mathbf d,\bullet}
\]
is a decomposition of $M$ as a $T$-module.
If $\{g_k:k\}$ is a finite generating set for $M$ over $R$ and
$\{\mathbf s_\alpha:\alpha\}$ are the monomials in $S$, then
\[
\{\mathbf s_\alpha\cdot g_k: \deg(\mathbf s_\alpha)+\deg_S(g_i)={\mathbf d};\quad\forall k, \alpha\}
\]
is a finite generating set for $M_{\mathbf d,\bullet}$ over $T$.

Since $I$ is homogeneous in the bigraded noetherian $R$,
its graded pieces $I_{\mathbf d,\bullet}$ are finite $T$-modules.
We shall sometimes call these graded pieces strands, akin to the equivalence between graded modules
and complexes.
\end{paragraf}

\begin{paragraf}
\label{par:moving-planes-notation}
For a fixed degree $\mathbf d$ on the source,
the elements of $I_{\mathbf d,i}$ are degree-$i$ syzygies of the $\phi_j$ with (module-)coefficients
in $S_{\mathbf d}$.
For specific degrees $\mathbf d$ and $i=1,2$, these have been called moving planes and quadrics,
respectively [references].
We shall only use this notation in Chapter \ref{ch:koszul-bpf} when we present a sort of a "template proof"
and use prove two results relating our work to what is already known in the case of
basepoint-free maps over $\PPP$ and $\PP^2$,
as well as a couple of new results along those lines.
\end{paragraf}

\begin{paragraf}
\label{par:basis}
Throughout this thesis we work with a fixed monomial order on $S$ (or rather $S'$).
For example, this could be the graded lexicographic order but the specific choice
is immaterial. With this convention, the statement
\begin{itemize}
\item[]
\text{let $\mathbf b$ be a row-vector corresponding to the basis of $S_{\mathbf d}$}
\end{itemize}
would mean a row vector having having as its coordinates the monomials of $S$,
listed in the selected order.

In this sense, and fixing some basis on $T$ too, we shall write
\[
\mt{basis}(S_{\mathbf d}),\quad\mt{basis}(T_i),\quad\mt{basis}(R_{\mathbf d,i})
\]
for the row vectors consisting of monomial bases in the chosen order for the $\CC$-vector spaces
$S_{\mathbf d}, T_i$  and $R_{\mathbf d,i}$, respectively.
\end{paragraf}

\begin{paragraf}
\label{par:syzygies}
Let $r=\dim_\CC(S_{\mathbf d})$ and
let $\mathbf b=\mt{basis}(S_{\mathbf d})$ be the row-$r$-vector of the monomial basis of $S$ in degree $\mathbf d$.
We shall mostly use $r$ instead of $\dim_\CC(S_{\mathbf d})$ when $\mathbf d$ has been fixed.

Given a form $g(\mathbf x)=g(\mathbf s;\mathbf x)\in I_{\mathbf d,i}$,
that is, a syzygy of degree $i$ over $S_{\mathbf d}$,
we can write $g(\mathbf x)$ as
\[
g(\mathbf x)=\mathbf b\cdot C=
\begin{bmatrix}
	x_0^i& x_0^{i-1}x_1& \ldots& x_n^i
\end{bmatrix}\cdot C'
\]
where $C$ is an $r\times1$ column vector with entries in $T_i$,
and $C'$ is a $\binom{n+i}{n}\times1$ column vector with entries in $S_{\mathbf d}$.

We shall use the term syzygy both for the column vector $C$ and the form $g$ in the
Rees ideal.
\end{paragraf}

\begin{paragraf}
\label{par:N-matrix}
We now come to the most important bit of notation.
Let us fix a degree $\mathbf d$ on the source.
As already apperent from \eqref{par:syzygies},
identifying the generators of $I_{\mathbf d}$ with column-$r$-vectors,
$I_{\mathbf d}$ becomes a sub-$T$-module of the free graded $T^r$.

Let $N$ be the matrix built from those column-generators.
Setting $\mu=\mu(I_{\mathbf d,\bullet})$ to be the number of columns,
$N$ becomes an $r\times\mu$-matrix over $T$
and more importantly, a graded $T$-linear map
\[
N:\bigoplus_{k} T(-i_k)\To T^r
\]
whose image is just $I_{\mathbf d,i}$.

By grouping the columns corresponding to the same degree $i$ into submatrices $N_i$,
for each valid $i$ we get a single matrix whose columns are the degree $i$-syzygies over $S_{\mathbf d}$.
Clearly, $N_i$ is empty for any $i$ larger than the maximum degree $\delta$ of a minimal generator
of $I_{\mathbf d,\bullet}$.
By the assumption on the linear independence of the $\phi_j$, $N_0$ is empty too.
In any case, the $N_i$ fit together to give $N$,
\[
	N=(N_1~|~N_2~|~\ldots~|~N_\delta)
\]

In Chapter 7 we describe the close connection of $N_1$ and $N_2$ to the matrices used in
the moving planes and quadrics results.

Finally, whenever useful, we write $h_i$ for $\dim_\CC \Span (N_i)=\dim_\CC(I_{\mathbf d,i})$
and $h$ for the tuple $(h_1,\ldots,h_\delta)$.
\end{paragraf}

\begin{paragraf}
Recall that for any ideal $Q$ in a unique factorization domain $T$,
$\gcd(Q)$ is defined to be the unique minimal principal ideal which contains $Q$,
and this definition obviously generalizes the definition on elements
when $T$ is a Euclidean domain.
\end{paragraf}

\begin{paragraf}
As a final bit of notation, we mention that we use $\rad(-)$ for the radial,
and $\sat(-)$ for the saturation with respect to the irrelevant ideal.
\end{paragraf}



\section{Strands of Module Maps}

\begin{paragraf}
\label{par:all-phi-k}
Let us consider the coordinates of $\phi$ as a row vector over $S$. We get a graded $S$-linear map
\[
\begin{bmatrix}
	\phi_0& \phi_1& \ldots& \phi_n
\end{bmatrix}:S^{n+1}\To S^1(\e)
\]
where $S(\e)$ has the usual meaning of putting $1\in S$ in degree $-\e$.
Equally well, we can consider the graded $S$-linear map introduced by the quadratic monomials
of the coordinates,
\[
\begin{bmatrix}
	\phi_0^2& \phi_0\phi_1& \ldots& \phi_n^2
\end{bmatrix}:S^{(n+2)(n+1)/2}\To S^1(2\mathbf e)
\]

As already mentioned in \eqref{par:N-matrix}, in the case $n=3$,
these two maps have been studied in relation to the methods of moving planes and quadrics.
In this section we consider all such $S$-linear maps coming from $\phi$,
that is, not just those coming from monomials of degree $1$ and $2$.
\end{paragraf}

\begin{paragraf}
\label{par:strands}
Let $k$ be a positive integer and $\mathbf d$ be a fixed degree on $S$ such that $S_\d\neq0$.
Define $\phi^{(k)}$ to be the graded $S$-linear map formed by the coordinates of $\phi$,
\[
\phi^{(k)}=\begin{bmatrix}\phi_0^k& \phi_0^{k-1}\phi_1& \cdots& \phi_n^k\end{bmatrix}:
S^{\binom{n+k}{n}}\To S^1(k\mathbf e)
\]
and set $\Phi^{(k)}$ be the strand of $\phi^{(k)}$ in degree $\d$, that is,
\[
\Phi^{(k)}:S_{\mathbf d}^{\binom{n+k}{n}}\To S_{k\mathbf e+\mathbf d}^1
\]
is a map of complex vector spaces.

Choosing bases \eqref{par:syzygies}, we can think of $\Phi^{(k)}$ as a matrix over $\CC$ of size
\[
\dim_\CC(S_{k\mathbf e+\mathbf d})\times r\binom{n+k}{n}
\]
whose columns can be indexed by the monomials in $R_{\mathbf d,k}$.
\end{paragraf}

\begin{example}
Give a small example to clarify this.
\end{example}

\begin{paragraf}
\label{par:syzygies-kernel}
The advantage of the matrices $\Phi^{(k)}$ over the matrices $\phi^{(k)}$
is that the kernel of $\Phi^{(i)}$ corresponds directly to the degree-$i$ syzygies over $S_{\mathbf d}$.
That is, for a fixed $\mathbf d$,
\[
\mathbf v\mapsto\mt{basis}(R_{\mathbf d,i})\cdot \mathbf v:\ker(\Phi^{(i)})\To I_{\mathbf d, i}
\]
is an isomorphism of vector spaces.
\end{paragraf}



\section{Multiplicity}

\begin{paragraf}
\label{par:mult-ha}
We recall some notation from \cite{Har-book-77}. For a homogeneous
prime ideal $P$ in a graded ring $T$, we set $T_{(P)}$ to be the degree-0 graded piece of the localization
of $T$ at the homogeneous elements outside of $P$. If $P$ is a minimal prime of a graded $T$-module
$M$, we denote by $\mult_P(M)$ the length of the $T_P$-module $M_P$ (see {\em loc. cit.}, I, Proposition 7.4).
\end{paragraf}

\begin{paragraf}
\label{par:mult}
Let $Z$ be a zero-dimensional closed subscheme of a smooth projective variety $X$.
Let $\mathscr{J}\se\mathscr{O}_X$ be the ideal sheaf of $Z$
and let $q\in Z$ be any point.
Since $Z$ is zero-dimensional, $\mathscr{J}_q\se\mathscr{O}_{q,X}$ is an ideal of definition
in the regular local $\mathscr{O}_{q,X}$.

Define the multiplicity of $Z$ at $q$, denoted $e_q$,
to be the Hilbert-Samuel multiplicity of $\mathscr{J}_q$.
Define the degree of $Z$ at $q$, denoted $d_q$, to be the length of the local ring,
\[	
	\length(\mathscr{O}_{q,Z})=\dim_{\CC}(\mathscr{O}_{q,Z})
\]

We have that $d_q\leq e_q$ with equality if and only if $q$ is a local complete intersection
point (\citet{BH-98-CMrings}, Theorem 4.7.4).
\end{paragraf}



\section{Examples}

In this section we try to summarize the notation and definitions introduced so far
by detailing them in a couple of explicit examples.

\begin{example}[$\mt{ex203}$]
\label{ex203}
Let $X=\PP^1_{s,u}\times\PP^1_{t,v}$ be the product of projective lines and let $S=\CC[s,u;t,v]$
be its Cox ring---a homogeneous coordinate ring graded by $\Pic(X)$ such that $\deg(s)=\deg(u)=(1,0)$
and $\deg(t)=\deg(v)=(0,1)$.
We write $X=\Proj(S)$ for the construction in the toric setting.

Let $\mathscr{L}=\O_X(2,2)$ be the line bundle on $X$ given by $\mathbf p=(2,2)$,
either as an element in $\Pic(X)=\ZZ^2$ or as a degree on $S$.
Since $h^0(\mathscr{L})=5$, we can choose $n+1=\dim(X)+2=4$ linearly independent over $\CC$
global section $\phi_0,\ldots,\phi_3$ of $\mathscr{L}$ to get a rational map
\[
	\phi=(\phi_0,\ldots,\phi_3):X\To\PP^3=\Proj(T)
\]
where $T=\CC[x_0,\ldots,x_3]$.

Since the sections are just $(2,2)$-forms, we can form the graded ideal
\[
	J=\la \phi_0,\ldots,\phi_3\ra\se S
\]

and since we are only be interested of maps whose image is of full dimension,
we can assume that the base locus of $\phi$, $Z=V(J)$, is zero-dimensional.
In particular, $\phi$ is a morphism of schemes away from $Z=V(J)$ as a set,
but we remember the scheme structure on $Z$,
the saturation of $J$ with respect to the irrelevant ideal $\m\se S$ of $X$.

For an explicit example, we take
\[
	\phi=(s^{2} v^{2},s u v^{2},u^{2} t^{2}+u^{2} t v,s u t v-101 u^{2}t v)
\]
in which case
\[
	J=\la s^{2} v^{2},s u v^{2},u^{2} t^{2}+u^{2} t v,s u t v-101 u^{2}t v\ra
\]
is saturated and $Z$ is zero-dimensional supported on the closed points
\[
	q_1=(0,1)\times(0,1) \text{ and } q_2=(1,0)\times(1,0)
\]

The base locus $Z$ looks quantatively different at $q_1$ and $q_2$, i.e.
in the former case it its defining ideal looks like $\la s,t\ra$ in $\AA^2_{s,t}$
while in the latter case like $\la u^2,uv,v^2\ra$.
This is formalized in in the following way.
Let $\mathscr{J}$ be the ideal sheaf defining $Z$,
then the geometric multiplicities are given by the length of the artinean locals,
\[
	d_{q_1}=\length(\O_{q_1,Z})=1 \text{ and } d_{q_2}=\length(\O_{q_2,Z})=3
\]
while the algebraic multiplicities are given by the Samuel multiplicities
of the ideals of finite colength $\mathscr{J}_{q_1}$ and $\mathscr{J}_{q_2}$,
\[
	e_{q_1}=e(\mathscr{J}_{q_1},\O_{q_1,X})=1 \text{ and } e_{q_2}=e(\mathscr{J}_{q_2},\O_{q_2,X})=4
\]
which in turn reflects the fact that $q_1$ is a complete intersection (c.i.) point of $Z$
while $q_2$ is~not.

The closed image $Y\se\PP^3$ of $\phi$ is given by a single {\em implicit} equation
\[
	P(\mathbf x)={x}_{0}^{2} {x}_{2}-202 {x}_{0} {x}_{1} {x}_{2}+10201
  {x}_{1}^{2} {x}_{2}-{x}_{0} {x}_{1} {x}_{3}+101 {x}_{1}^{2} {x}_{3}-{x}_{0}
  {x}_{3}^{2}
\]
and we use $P$ to also denote its principal prime ideal codimension-$1$ ideal in $T=\CC[\mathbf x]$.

The Rees algebra $B=\Rees_S(J)$ is the quotient of $R=S[\mathbf x]$ by the ideal $I\se R$
of syzygies on the $\phi_j$,
% \[
% 	I=\text{the kernel of the morphism of $S$-algebras }(x_j\mapsto \phi_j\cdot t :R\to S[t])
% \]
and for any fixed degree $\mathbf d$,
the graded piece $I_{\mathbf d,\bullet}$ is a finite sub-$T$-module of $T^r$ where
\[
	r=\dim_\CC(S_{\mathbf d})=h^0(\O_X(\mathbf d))
\]

As such $I_{\mathbf d,\bullet}=\image(N)$ for an $r\times\mu$ graded matrix $N$, where $\mu$
is the size of a minimal homogeneous generating set. Specifically, in our example
\[
	I=\Bigg\la
	\begin{cases}
	u {x}_{0}-s {x}_{1},
	(s v-101 u v) {x}_{2}+(-u t-u v){x}_{3},\\
	(t^{2}+t v) {x}_{1}-101 v^{2} {x}_{2}+(-t v-v^{2}) {x}_{3},\\
	(s t-101 ut) {x}_{1}-s v {x}_{3},v {x}_{0} {x}_{2}-101 v {x}_{1} {x}_{2}+(-t-v) {x}_{1}{x}_{3},\\
  	t {x}_{0} {x}_{2}-101 t {x}_{1} {x}_{2}-101 v {x}_{2} {x}_{3}+(-t-v){x}_{3}^{2},\\
  	s {x}_{0} {x}_{2}+(-202 s+10201 u) {x}_{1} {x}_{2}+(-s+101 u){x}_{1} {x}_{3}-s {x}_{3}^{2},\\
	t {x}_{0} {x}_{1}-101 t {x}_{1}^{2}-v {x}_{0}{x}_{3},\\P(x_0,x_1,x_2,x_3)
	\end{cases}\Bigg\ra
\]
and for $\mathbf d=(1,1)$ we get $r=4$, $\mu=5$ and 
\[
	N=\begin{bmatrix}0&
	       {x}_{1}&
	       0&
	       0&
	       {x}_{0} {x}_{2}-{x}_{3}^{2}\\
	       {x}_{2}&
	       {-{x}_{3}}&
	       {-{x}_{1}}&
	       {-{x}_{3}}&
	       {-{x}_{3}^{2}}\\
	       {-{x}_{3}}&
	       {-101 {x}_{1}}&
	       0&
	       {x}_{0}-101 {x}_{1}&
	       -10201 {x}_{1} {x}_{2}-202 {x}_{3}^{2}\\
	       -101 {x}_{2}-{x}_{3}&
	       0&
	       {x}_{0}&
	       0&
	       20402 {x}_{2} {x}_{3}-202 {x}_{3}^{2}\\
	       \end{bmatrix}
\]

Each column $C$ of $N$ is a syzygy on the $\phi_j$ of some degree $i$ on the $x_j$
and degree $\mathbf d=(1,1)$ in the sense that
\[
	\begin{bmatrix}s t&
	       s v&
	       u t&
	       u v\\
	       \end{bmatrix}
		   \cdot C(\mathbf x)
\]
is a homogeneous degree $(\mathbf d,i)$-element of $I$, i.e.
\[
	\begin{bmatrix}s t&
	       s v&
	       u t&
	       u v\\
	       \end{bmatrix}
		   \cdot C(\mathbf\phi)=0
\]
identically in $S$.

Finally, we note that $\phi$ is of degree $1$, $\mathbf d\in\reg(J)$ and we have the equality of ideals
\[
	P^{\deg(\phi)}=\gcd(\mt{minors}(4,N))\demo
\]
\end{example}

\begin{example}[$\mt{ex204}$]
\label{ex204}
add
\end{example}



%% STRIP BEGIN
%% BIBLIOGRAPHY
\bibliographystyle{unsrtnat}
\bibliography{../lib/refs}

\end{document}
