\documentclass[fleqn,reqno]{amsart}
\usepackage{../lib/radoslav-macro}
\usepackage{../lib/radoslav-more}
\usepackage{times}

\begin{document}
%% STRIP END



\begin{paragraf*}
We are now ready to state our main results.
While this chapter is supposed to be self-contained and most of the relevant notation
and definitions are listed in \eqref{par:again} below,
one should consult Chapter \ref{ch:preliminaries} for a more relaxed exposition
and further definitions, for instance,
for the notions of degree and multiplicity of a basepoint.

Examples \ref{ex203} and \ref{ex204} should serve as quick reference points.
\end{paragraf*}

\begin{paragraf}
\label{par:again}
\label{par:setup}
Let $X$ be a smooth projective toric variety of dimension $n-1$ ($n>1$)
with Cox ring $S$ and irrelevant ideal $\n\se S$.
Let $T=\CC[x_0,\ldots,x_n]$ and let
\[
\phi=(\phi_0,\ldots,\phi_n):X\To\PP^n=\Proj(T)
\]
be a rational map given by linearly independent sections of the line bundle $\O_X(\e)$
for some degree $\e$ on $S$.
Let $J=\la \phi_0,\ldots,\phi_n\ra\se S$ be the ideal of the coordinates $\phi_j$.

Let $\d$ be a degree on $S$ such that $r=\dim_\CC(S_\d)>0$ and let $I\se R=S\tensor T$ be the
Rees ideal of $J$. Let $I_{\d,\bullet}$ be its degree-$\d$ bigraded piece,
considered as a finite graded $T$-module.
We denote by $B=R/I$ the Rees algebra of $J$.

Let $N$ be the $r\times\mu$ coefficient matrix of
a minimal set of homogeneous generators for~$I_{\d,\bullet}$,
and let $N_i$ be the submatrix of $N$ corresponding to generators of degree $i$,
so that
\[
N=(~N_1~|\ldots|~N_\delta~)
\]
\end{paragraf}

\begin{paragraf}
\label{par:conds}
Let $P\se T$ be the prime ideal corresponding to the closed image of $\phi$ in $\PP^n$.
We denote this image by $Y=V(P)$.
Let $Z=V(J)\se X$ be the base locus of $\phi$.

We are going to be interested the following three conditions:
\begin{enumerate}
\item
\label{itm:conds:1}
The map $\phi$ is generically finite onto its image, that is,
$Y\se\PP^n$ is of codimension~$1$ and in particular, the ideal $P$ is principal.
In this case, we denote a generator of~$P$ by~$P(\mathbf x)$.

\item
\label{itm:conds:2} The base locus $Z$ is zero-dimensional, that is,
consists of finitely many points.
Note that those are necessarily closed over $\CC$.

\item
\label{itm:conds:3} The map $\phi$ is birational.
\end{enumerate}

Clearly, either of \eqref{itm:conds:2} and \eqref{itm:conds:3} implies \eqref{itm:conds:1}.
\end{paragraf}

\begin{paragraf*}
We begin by making a few easy but important observations about the Rees ideal.
These are needed later.
\end{paragraf*}

\begin{proposition}
\label{prop:BP}~
\begin{enumerate}[label=\normalfont(\alph*)]
\item
The ideal $I\se R$ is prime, and so is $I_P$ in the $T$-module localization $R_P$.

\item
The quotient $B_P$ is naturally a finite-type graded $K(T/P)$-algebra with grading induced by $S$.

\item\label{itm:BPKTP-alg}
The $K(T/P)$-algebra $B_P$ is a homogeneous coordinate ring of a projective variety.
\end{enumerate}
\end{proposition}

\begin{paragraf}
\label{par:reg-IP}
Let $V(I_P)$ be the closed subset of the projective toric variety $\Proj(R_P)$.
Note that the latter is a variety over $K(T/P)$ with the grading of $S$.
Following \citet{MS-04} and using Proposition~\ref{prop:BP},
we consider the regularity of the defining ideal $I_P$,
and denote it from now on by $\reg(I_P)$.

We recall that $\reg(I_P)$ is a finitely generated additively-closed subset
of the semigroup of degrees on $S$, and that
for any $\d\in\reg(I_P)$, we have $\la(I_P)_\d\ra=I$.
This parallels the usual Castelnuovo-Mumford regularity for $\PP^n$
and is the content of Theorem 1.3 in the referenced paper.
\end{paragraf}

\begin{paragraf*}
In light of Proposition \ref{prop:BP}, our first result becomes an easy exercise.
However, it a step toward the goal of this paper --- to exhibit a general relation between
the algebra of the coordinates $\phi_j$ and the geometry of the image $Y$.
\end{paragraf*}

\begin{theorem}
\label{thm:rad-minors}
In the setup of \eqref{par:again}, one has
\[
	\rad(\mt{minors}(r,N))=P
\]
\end{theorem}

\begin{paragraf*}
The geometric interpretation of the theorem is clear ---
the nonzero minors of $N$ define hypersurfaces in $\PP^n$ whose intersection,
at least set-theoretically, is the image $Y$.

Example~\ref{ex307}, for instance, shows that the radical is necessary.
\end{paragraf*}

\begin{paragraf*}
Our next result is the main theorem of this thesis, unifying two currently
popular non-Gr\"{o}bner bases approaches to implicitization and setting the stage for both
the ad-hoc template proofs in Chapter~\ref{ch:koszul-bpf} and
the fast implicitization method described in Chapter~\ref{ch:fast-method}.

From now on, we shall only be interested in maps which are generically finite.
\end{paragraf*}

\begin{theorem}
\label{thm:gcd-minors}
Consider the setup of \eqref{par:again} and assume \itmref{itm:conds:1}{par:conds}.
Fix a degree $\mathbf d\in\reg(I_P)$ as described in \eqref{par:reg-IP}.
One has
\[
	\gcd(\mt{minors}(r, N))=P^{\deg\phi}
\]
In particular, if there is a degree $\mathbf d$ in the regularity for which $\mu=r$,
then $N$ is square and, up to a unit,
\[
	\det(N)=P(\mathbf x)^{\deg\phi}
\]
\end{theorem}

\begin{corollary}
\label{thm:detM}
In the setup of Theorem~\ref{thm:gcd-minors},
let $M$ be any $r\times r$~matrix of syzygies over~$S_\d$. One has
\[
	\det(M)=P(\mathbf x)^{\deg\phi}\cdot H(\mathbf x)
\]
for a homogeneous $H(\mathbf x)$ of degree
\begin{align}
\label{eq:degree-offset}
\deg(\det(M))-\deg(\phi)\cdot\deg(Y)
\end{align}

Furthermore, there exist a list of such matrices $\{M_k\}$ whose
corresponding $H_k(\mathbf x)$ are nonzero and without a common factor.
\end{corollary}

\begin{paragraf*}
Geometrically, the former is a refinement of Theorem \ref{thm:rad-minors}.
Each of the maximal minors of $N$, in fact,
the determinant of any $r\times r$ matrix $M$ of syzygies over~$S_\d$, is either zero
or describes in $\PP^n$ the union of a $\deg(\phi)$-fold $Y$ and
a hypersurface of degree \eqref{eq:degree-offset}.
While an arbitrary collection $M_k$ of such matrices may introduce hypersurfaces
with an intersection that is strictly larger than $Y$,
the maximal minors suffice to shave off any extraneous components.

The theme of extraneous factors is already apparent in
\citet{BCD-03}, \citet{BCJ-09} and \citet{BDD-09}.
In our notation, they used the approximation complex to show that for a toric $X$,
certain $\d$ and empty or zero-dimensional almost complete intersection base locus $Z$,
\[
	\gcd(\minors(r,N_1))=P^{\deg\phi}\cdot\prod_{q\in Z} L_q(\mathbf x)^{e_q-d_q}
\]
where each $L_q(\mathbf x)$ is a linear form, and
$e_q$ and $d_q$ are the multiplicity and degree of $q$.

In the case of complete intersection base locus, the proof of Theorem~\ref{thm:gcd-minors}
gives a special case of the above.
\end{paragraf*}

\begin{corollary}[\citet{BCD-03}]
\label{cor:approx-complex}
In the setup of Theorem~\ref{thm:gcd-minors}, let $\d$ be {\em large enough} in $\reg(I_P)$.
Let $\Proj(B)$ and $\Proj(\Sym(J))$ be naturally isomorphic in the sense of~\eqref{eq:R-Sym-B}.
One has
\[
	\gcd(\mt{minors}(r, N_1))=P^{\deg\phi}
\]

In particular, the result holds if the base locus is empty or
zero-dimensional and locally a complete intersection.
\end{corollary}

\begin{paragraf*}
It is known that if $M$ is a square matrix over $T$ of size $r$,
then the singular locus of $V(\det(M))$ is contained in the closed subset defined by the
comaximal minors, that is, the $(r-1)$-minors.
Although we failed to find a reference,
we believe that this relation is more intrinsic and holds for all matrices $N$.
However, what this ought to correspond to is the multiple-point locus of the image.
See Example~\ref{ex315} for details.
We conjecture the following
\end{paragraf*}

\begin{conjecture}
\label{conj:sing-locus}
Consider the setup of \eqref{par:again} and assume \itmref{itm:conds:3}{par:conds}.
On the level of closed points, one has
\[
	V(\minors(r-1,N))\se\mathrm{Sing}(Y)
\]
\end{conjecture}



\begin{paragraf*}
In the simplest cases of interest, when $X=\PP^2$ or $X=\PPP$ and $\phi$ is basepoint-free,
we can chose $\d$ so that the matrix $N$ becomes square.
Next two theorems are slight generalizations of the results in \citet{CGZ-00}.
More importantly, they show that our methods directly generalize the methods of
moving planes and quadrics in the setting in which they are most useful.
\end{paragraf*}

\begin{theorem}
\label{thm:rel-moving-planes-quadrics}
Let $X=\PP^2$, $\phi$ be basepoint-free,
and suppose that there are exactly $p=\e$ linear syzygies over degree $\d=p-1$,
that is, the minimal possible number.
One has that $N=(N_1~|~N_2)$, $N$ is square and $\phi$ is birational.
In particular,
\[
	\det(N)=\det(N_1~|~N_2)=P(\mathbf x)
\]
\end{theorem}

\begin{theorem}
\label{thm:rel-moving-quadrics}
Let $X=\PP^1\times\PP^1$, $\phi$ be basepoint-free with coordinates in degree $\mathbf e=(p,q)$,
and suppose there are no linear syzygies over degree $\d=(p-1,q-1)$.
Then one has that $N_2$ is square, $N=N_2$ and $\phi$ is birational. In particular,
\[
	\det(N)=\det(N_2)=P(\mathbf x)
\]
\end{theorem}

\begin{paragraf*}
Both of these theorems are examples of a template proof described in Chapter~\ref{ch:koszul-bpf}.
While applying it in general requires elaborate choses of the degree $\d$ and
regularity computations, for example \citet{AHW-05},
in the case of Theorems \ref{thm:rel-moving-planes-quadrics} and \ref{thm:rel-moving-quadrics},
we only use a type of Koszul-ness on the syzygies of low degree.
This is the content of Section~\ref{sec:bpf-low-koszul}.
\end{paragraf*}



\begin{paragraf*}
We conclude this list by a method to compute the degree of a rational map using Gr\"{o}bner bases.
While we are only going to use this in our examples,
it helps expand our understanding about the object $B_P$.

The author wants to thank Mike Stillman for suggesting the following
\end{paragraf*}

\begin{proposition}
\label{prop:deg-GB}
Let $\CC[s_0,\ldots,s_m]$ be the fixed ambient polynomial ring of $S'$ as described in
\eqref{par:setup}.
Define the ideal $I_B$ of $\CC[s_0,\ldots,s_m;x_0,\ldots,x_n]$ by the equality
\[
	B=\CC[\mathbf s;\mathbf x]/I_B
\]
Let $>'$ be any product order in which the $\mathbf s$ variables
come before the $\mathbf x$ variables.
Then a reduced Gr\"obner basis for $I_B$ with respect to $>'$ has the form
\begin{align*}
	&g_1(\mathbf s;\mathbf x)=p_1(\mathbf x){\mathbf s}^{\alpha_1}+\text{lower order terms}\\
	&\qquad\cdots\\
	&g_r(\mathbf s;\mathbf x)=p_r(\mathbf x){\mathbf s}^{\alpha_r}+\text{lower order terms}\\
	&g_{r+1}(\mathbf s;\mathbf x)=P(\mathbf x)
\end{align*}
and
\[
	\deg(\phi)=
	\deg \big(\la{\mathbf s}^{\alpha_1},\ldots,{\mathbf s}^{\alpha_r}\ra\se\CC[\mathbf s]\big)
\]
\end{proposition}

% \begin{paragraf*}
% Next, we present an example with two surprising features.
% First, it shows what the general map of some fixed degree and base locus looks like,
% and how this is related to the geometry of its image.
% Second, this will be an example of a map whose image is the determinant of a matrix
% of linear forms, yet not one build from syzygies.
% It foreshadows the results in Chapter \ref{ch:bicubics} and highlights the close
% relation of $N$ to $S$, not just $Y$.
% See Example \ref{ex311} for a concrete calculation and the examples in Chapter \ref{ch:examples}
% for a further discussion.
% \end{paragraf*}
%
% \begin{conjecture}
% \label{conj:nice-example}
% Let $X=\PPP$ and $\phi$ be a general map among the maps be given by four bicubics with
% a common zero locus the single point $(0,1)\times(0,1)$ of degree $3$ and multiplicity $4$.
% Then $\phi$ is generically 1-1, the image $Y$ is of degree $4$ and is singular along
% three non-degenerate concurent lines.
% Further, $Y$ is defined by the determinant of a $4\times4$-matrix of linear forms.
% \end{conjecture}



%% STRIP BEGIN
%% BIBLIOGRAPHY
\bibliographystyle{unsrtnat}
\bibliography{../lib/refs}

\end{document}