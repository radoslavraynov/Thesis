\documentclass[fleqn,reqno]{amsart}
\usepackage{../lib/radoslav-macro}
\usepackage{../lib/radoslav-more}
\usepackage{times}
\usepackage{natbib}


\newcounter{chapter}
\setcounter{chapter}{2}
\numberwithin{first}{chapter}


\begin{document}
%% STRIP END



\begin{paragraf}
In what follows, let $S,T,\phi,J,P,I,R$ and $B$ be as defined in Chapter \ref{ch:preliminaries}.
Once a degree $\d$ on the source has been fixed, explicitly or implicitly,
$N,r,\mu$ and $h_i$ are used to denote the corresponding objects in that degree.
% \hyperref[ex101]{Example \ref*{ex101}} should serve as a quick reference.
Examples \ref{ex203} and \ref{ex204} should serve as quick reference points.
\end{paragraf}

\begin{paragraf*}
We begin by making a few easy but important observations about the Rees ideal $I$.
These are needed for the setup of some of the later results.
\end{paragraf*}

\begin{proposition}
\label{prop:BP}
\begin{enumerate}[label=\normalfont(\alph*)]
\item
The ideal $I\se R$ is prime, and so is $I_P$ in the $T$-module localization $R_P$.
\item
The quotient $B_P$ is naturally a finite-type graded $K(T/P)$-algebra
with grading induced by $S$.
\item
The ideal $I$ is saturated with respect to the irrelevant ideal of $S\se R$.
\item
Finally, the $K(T/P)$-algebra $B_P$ is the Cox ring of a projective toric variety.
\end{enumerate}
\end{proposition}

\begin{paragraf}
\label{par:describe-d}
From now on, we shall only be interested in degrees $\d$ is such that the line bundle
$\O_X(\d)$ has global sections, that is, $h^0(\O_X(\d))>0$.
Following \citet{03-Maclagan-Smith}, by part (d) of Proposition \ref{prop:BP}
there is a notion of regularity on $B_P$, or rather on
\[
\Proj_{K(T/P)}(B_P)
\]
We denote this order ideal in the poset of degrees by $\reg(B_P)$.

We shall see later that the latter is $0$-dimensional, so one can choose to
stick to the classical $\Proj(-)$ construction by reembedding the set of points
in projective space over $K(T/P)$ and thinking of the regularity
of the toric variety as those $\d\in\Pic(X)$ which correspond to degrees in the
reembedding that are greater or equal to the Castelnuovo-Mumford regularity.
\end{paragraf}

\begin{paragraf*}
In light of Proposition \ref{prop:BP}, our first result becomes an easy exercise.
It establishes for the first time a general relation between the minors of matrix~$N$
and the implicit equation~$P$.
\end{paragraf*}

\begin{theorem}
\label{thm:rad-minors}
Let $\mathbf d$ be any degree on $S$. One has
\[
\rad(\mt{minors}(r,N))=P
\]
\end{theorem}

\begin{paragraf*}
The geometric interpetation of this theorem is clear ---
the nonzero minors of $N$ define hypersurfaces in $\PP^n$ whose intersection,
at least set-theoretically, is the image $Y$.
\end{paragraf*}

\begin{paragraf*}
Our next result is the main theorem of this thesis, unifying the two currently
popular non-Gr\"{o}bner bases approaches and setting the stage for both
the ad-hoc template proofs in Chapter \ref{ch:koszul-bpf} and
the fast implicitization method described in Chapter \ref{ch:fast-method}.
\end{paragraf*}

\begin{theorem}
\label{thm:gcd-minors}
Fix a degree $\mathbf d\in\reg(B_P)$. One has
\[
	\gcd(\mt{minors}(r, N))=P^{\deg\phi}
\]
In particular, if there is a $\mathbf d$ for which $\mu=r$,
then one has that $N$ is square and, up to a unit,
\[
	\det(N)=P(\mathbf x)^{\deg\phi}
\]
\end{theorem}

\begin{corollary}
\label{thm:detM}
Fix a degree $\d\in\reg(B_P)$ and let $M$ be any $r\times r$ matrix of syzygies over $S_\d$. One has
\[
	\det(M)=P(\mathbf x)^{\deg\phi}\cdot H(\mathbf x)
\]
for a homogeneous $H(\mathbf x)$ of degree $\deg\det(M)-\deg\phi\cdot\deg(Y)$.

Furthermore, there exist such matrices $M_k$ such that the corresponding $H_k(\mathbf x)$
are nonzero and without a common factor.
\end{corollary}

\begin{paragraf*}
Geometrically, the former is a refinement of Theorem \ref{thm:rad-minors}.
Each of the maximal minors of $N$, in fact
the determinant of any $r\times r$ matrix $M$ of syzygies over $S_\d$, is either $0$
or describes in $\PP^n$ the union of a $\deg\phi$-fold $Y$ and
a hypersurface of degree $\deg\det(M)-\deg(\phi)\deg(Y)$.
The maximal minors suffice to shave off the extraneous 

The theme of extraneous factors is already apparent in \citet{BCJ09} and \citet{Botbol2011}
who use the approximation complex to show that for a toric $X$,
certain $\d$ and almost complete intersection baselocus $Z$, in our running notation, one has
\[
\gcd(\minors(r,N_1))=P^{\deg\phi}\cdot\prod_{p\in Z} L_p(\mathbf x)^{e_p-d_p}
\]
where each $L_p(\mathbf x)$ is a linear form.

In the case of complete intersection baselocus, the proof of Theorem \ref{thm:gcd-minors}
gives for free a special case of the above,
\end{paragraf*}

\begin{corollary}
\label{cor:approx-complex}
Let $\mathbf d$ be large enough in the poset of degrees on $S$.
Let $N_1\se N$ be the matrix of linear syzygies.
If the base locus of $\phi$ is empty or locally a complete intersection, then
\[
	\gcd(\mt{minors}(r, N_1))=P^{\deg\phi}
\]
\end{corollary}

\begin{paragraf*}
Say how this is relates the singular locus, or rather the double-point locus,
to the co-maximal minors.
Say that there is a similar result in the literature relating the comaximal minors
of a square matrix to the singular locus of the hypersurface defined by the determinant
but that no referece is given.
Remember the engineering linear algebra book where you read it.
Reference examples where this is addressed and examples showing that this is not
an equality.
\end{paragraf*}

\begin{conjecture}
\label{conj:sing-locus}
Let $\phi$ be birational and $\d$ be any degree on $S$. One has
\[
V(\minors(r-1,N))\se\mathrm{Sing}(Y)
\]
\end{conjecture}



\begin{paragraf*}
In the simplest cases of interest, when $X=\PP^2$ or $X=\PPP$ and $\phi$ is basepoint-free,
we can chose $\d$ so that the matrix $N$ becomes square.
Next two theorems are slight generalizations of the results in \citet{00-CGZ-JSC}.
More importantly, they show that our methods directly generalize the methods of
moving planes and quadrics in the setting in which they are most useful.
\end{paragraf*}

\begin{theorem}
\label{thm:rel-moving-planes-quadrics}
Let $X=\PP^2$, $\phi$ be basepoint-free,
and suppose that there are exactly $p=\mathbf e$ linear syzyges over degree $\d=p-1$,
that is, the minimal possible number.
Then one has that $N=(N_1~|~N_2)$, $N$ is square and $\phi$ is birational.
In particular,
\[
\det(N)=\det(N_1~|~N_2)=P(\mathbf x)
\]
\end{theorem}

\begin{theorem}
\label{thm:rel-moving-quadrics}
Let $X=\PP^1\times\PP^1$, $\phi$ be basepoint-free with coordinates in degree $\mathbf e=(p,q)$,
and suppose there are no linear syzyges over degree $\d=(p-1,q-1)$.
Then one has that $N_2$ is square, $N=N_2$ and $\phi$ is birational. In particular,
\[
	\det(N)=\det(N_2)=P(\mathbf x)
\]
\end{theorem}

\begin{paragraf*}
Both of these theorems are examples of a template proof decribed in Chapter \ref{ch:koszul-bpf}.
While applying it in general requires elaborate choses of the degree $\d$ and
regularity computations, in the case of basepoint-free maps we only used a type
of Koszul-ness on the syzygies of low degree. For details see Chapter \ref{ch:koszul-bpf}.
\end{paragraf*}



\begin{paragraf*}
We close this list by a method to compute the degree of a rational map using Gr\"{o}bner bases
and an interesting example of a typical relation between the geometry of the image and the
algebra of the coordinates.

The method below describes what a Gr\"{o}bner basis of $I$ looks like. While we are not
going to use it in any way in this thesis, it helps expand our understanding about the object $B_P$.

The author wants to thank Mike Stillman for suggesting it.
\end{paragraf*}

\begin{proposition}
\label{prop:degviaGB}
Let $\CC[s_0,\ldots,s_m]$ be the fixed ambient polynomial ring of $S'$ as described in \ref{par:setup}.
Define the ideal $I_B$ of $\CC[s_0,\ldots,s_m;x_0,\ldots,x_n]$ by the equality
\[
	B=\CC[\mathbf s;\mathbf x]/I_B
\]
Let $>'$ be any product order in which the $\mathbf s$ variables
come before the $\mathbf x$ variables.
Then a reduced Gr\"obner basis for $I_B$ with respect to $>'$ has the form
\begin{align*}
	&g_1(\mathbf s;\mathbf x)=p_1(\mathbf x){\mathbf s}^{\alpha_1}+\text{lower order terms}\\
	&\qquad\cdots\\
	&g_r(\mathbf s;\mathbf x)=p_r(\mathbf x){\mathbf s}^{\alpha_r}+\text{lower order terms}\\
	&g_{r+1}(\mathbf s;\mathbf x)=P(\mathbf x)
\end{align*}
and
\[
	\deg(\phi)=\deg \big(\la{\mathbf s}^{\alpha_1},\ldots,{\mathbf s}^{\alpha_r}\ra\se\CC[\mathbf s]\big)
\]
\end{proposition}

\begin{paragraf*}
Next, we present an example with two surprising features.
First, it shows what the general map of some fixed degree and base locus looks like,
and how this is related to the geometry of its image.
Second, this will be an example of a map whose image is the determinant of a matrix
of linear forms, yet not one build from syzygies.
It foreshadows the results in Chapter \ref{ch:bicubics} and highlights the close
relation of $N$ to $S$, not just $Y$.
See Example \ref{ex311} for a concrete calculation and the examples in Chapter \ref{ch:examples}
for a further discussion.
\end{paragraf*}

\begin{conjecture}
\label{conj:nice-example}
Let $X=\PPP$ and $\phi$ be a general map among the maps be given by four bicubics with
a common zero locus the single point $(0,1)\times(0,1)$ of degree $3$ and multiplicity $4$.
Then $\phi$ is generically 1-1, the image $Y$ is of degree $4$ and is singular along
three non-degenerate concurent lines.
Further, $Y$ is defined by the determinant of a $4\times4$-matrix of linear forms.
\end{conjecture}



%% STRIP BEGIN
%% BIBLIOGRAPHY
\bibliographystyle{unsrtnat}
\bibliography{../lib/refs}

\end{document}
