\documentclass[fleqn,reqno]{amsart}
\usepackage{../lib/radoslav-macro}
\usepackage{../lib/radoslav-more}
\usepackage{times}
\usepackage{natbib}

\newcounter{chapter}
\setcounter{chapter}{6}
\numberwithin{first}{chapter}

\begin{document}
%% STRIP END

\begin{paragraf*}
This section describes a method for fast implicitization.
\end{paragraf*}

\begin{paragraf*}
This chapter describes a method for fast implicitization using
the theory developed in Chapters \ref{ch:main-results} and \ref{ch:main-proofs}.
We present two algorithms in Section \ref{sec:algorithm}.
We then discuss the problems arising from the naive version.

In Section \ref{sec:example1} we carry out a calculation which 
and \ref{sec:example2},
we employ the algorithm and its implementation to compute examples which are unfeasible
to attack using the standard methods.
\end{paragraf*}

\begin{paragraf*}
The Macaulay2 code presented in this chapter is available at
\begin{center}
	\href{http://www.math.cornell.edu/~rzlatev/phd-thesis/algos}
	{\ttten http://www.math.cornell.edu/$\sim$rzlatev/phd-thesis/algos}
\end{center}
\end{paragraf*}

\begin{paragraf*}
We continue to assume the notation of Chapter \ref{ch:preliminaries}
and the setup of Chapter \ref{ch:main-results}.
However, to avoid distraction, we assume that the map $\phi$ is generically 1-1.
\end{paragraf*}



\section{Algorithm}
\label{sec:algorithm}

\begin{paragraf*}
At first glance, an algorithm for finding the implicit equation is contained in the proof
of the our main theorem, Theorem \ref{th:gcd-minors}.
In its simplest form, the algorithm is as follows
\end{paragraf*}

% \begin{absolutelynopagebreak}
\begin{algorithm} {\sc Naive Algorithm\bf.}
\label{algo:naive}
\begin{algorithmic}
  \State {\bf input:} the ideal $J$ and a degree $\d$ on $S$
  \State {\bf output:} the implicit equation $P^{\deg(\phi)}$
  \State Set $r=\dim_\CC(S_{\mathbf d})$
  \State Compute an $R$-generating set $\{h_l:l\}$ for the Rees ideal $I$
  \State Compute a $T$-generating set $\{g_k:k\}$ for $I_{\d,\bullet}$ from the $h_l$
  	using Lemma \ref{lemma:basis-Id}
  \State Set $N$ to be the coefficient matrix of the $g_k$ with respect to $\mt{basis}(S_\d)$
  \State Compute $P=\gcd(\mt{minors}(r,N))$
\end{algorithmic}
\end{algorithm}
% \end{absolutelynopagebreak}

\begin{implementation}
We first need an implementation of Lemma \ref{lemma:basis-Id}.
\label{code:lemma:basis-Id}
\begin{verbatim}
procPushGensHigher = (d,r,I) -> (
  G := for g in I_* list (
    if all(d-((degree g)_r), Z->Z>=0) then
      basis((d-((degree g)_r))|{0},ring I)**g
    else continue);
  trim image fold(G,matrix {{0_(ring I)}},(a, b)->a|b)
)
\end{verbatim}
\end{implementation}

\begin{implementation}
\label{code:algo:naive}
The implementation of Algorithm \ref{algo:naive} is now straight-forward.
We only note that the ambient ring $R$ is required as input for encapsulation purposes.
\begin{verbatim}
ComputeNRees = method ()
ComputeNRees (Ideal, List, Ring) := Matrix => (J, d, R) -> (
  x := symbol x;
  I := reesIdeal(J, Variable=>x);
  AI := ring I;
  zm := 0*d;
  g := map(R,AI,first entries super basis(zm|{1},R));
  I = g(I);
  r := toList(0..#(degree J_0)-1);
  V := procPushGensHigher (d,r,I);
  matrix entries ( (gens V) // basis(d|{0}, R) )
)
\end{verbatim}
\end{implementation}

\begin{paragraf}
\label{par:flaws}
The short length and robustness of Algorithm \ref{algo:naive} made it our preferred way
for testing the theory.
In fact, all concrete calculations presented so far,
including all examples of Chapter \ref{ch:examples},
were carried out using this algorithm.

At the same time, its simplicity allows us to easily spot some of its drawbacks.
We discuss four of them.
\begin{enumerate}
\item
\label{itm:flaws:GB}
Line 3 of the algorithm refers to Lemma~\ref{lemma:basis-Id} but unless we were given
an $R$-generating set for the Rees ideal,
we have to compute it from scratch.
This is at least as hard a computation as the computation of the implicit equation
in a very concrete sense ---
the Rees ideal contains the implicit equation as its only generator in bidegree $(0,\bullet)$.
This follows from Proposition~\ref{prop:deg-GB} and shows up in Example~\ref{ex314}.

On the other hand, if we had a way to efficiently compute a generating set for $I$,
we would only need an algorithm for finding the unique generator in the aforementioned bidegree,
thus producing an equally efficient solution to the implicitization problem.

\item
\label{itm:flaws:many-minors}
Line 5 computes the $\gcd$ of all maximal minors of $N$.
While computing the $\gcd$ of polynomials is fast,
the number of minors can be very large.
This happens even for reasonably small examples.
For instance, the smallest nonzero matrix $N$ for the example of Section \ref{sec:example1}
is of size $18\times50$.
The number of maximal minors is
\[
	\binom{50}{18}=18053528883775
\]
so even if it took the unrealistic $0.001$ seconds to compute each minor,
a single machine would require 572 years to compute them all.

\item
\label{itm:flaws:large-det}
Continuing with the example of Section \ref{sec:example1},
we note that each maximal minor is the determinant of a matrix of order $18$
whose entries are quartic forms on 5 variables.
The total degree of each minor is then $72$.
Computing large determinants symbolically is time-consuming.
The calculation of the determinant of a $13\times13$-matrix of quartic forms
over 5 variables took about an hour to finish.
Scaling this, we can expect this calculation to take around
\[
	14\times15\times16\times17\times18=1028160
\]
hours or about $714$ days.

\item
\label{itm:flaws:large-poly}
Finally, suppose we have found the polynomial in question --- by whatever means.
It is a form of degree $48$ over 5 variables, and very likely dense in those monomials.
This suggests that the polynomial will be represented by
\[
	\binom{53}{5}=2869685
\]
coefficients.
\end{enumerate}
\end{paragraf}

\begin{paragraf*}
Regrettably,
\itmref{itm:flaws:large-det}{par:flaws} would be an issue for any algorithm
relying on computing determinants of matrices of syzygies,
while \itmref{itm:flaws:large-poly}{par:flaws}
would be an issue for any implicitization algorithm whatsoever.
Rather than seeing these as obstacles,
we point them out as an argument {\em for} the idea of using such
representation matrices in place of the implicit equation altogether.
We explores this theme further in Section \ref{sec:long-examples}.
\end{paragraf*}

\begin{paragraf}
\label{par:proposed}
Fix a degree $\d$ and recall that
\[
	N=(~N_1~|~\cdots~|~N_\delta~)
\]
where $N_i$ is the submatrix of $N$ consisting of the degree-$i$ columns,
and $\delta$ is the maximal degree in a minimal system of homogeneous generators
for~$I_{\d,\bullet}$.

Loosely speaking, we propose an method which employs the following two observations:
\begin{enumerate}
\item
\label{itm:N'}
instead of computing the whole matrix $N$, one can compute the $N_i$'s separately,
keeping track of a partial syzygy matrix $N'$;
\item
\label{itm:gcd}
instead of computing all the minors, one only needs to compute enough to determine
the $\gcd$ correctly;
\end{enumerate}
\end{paragraf}

\begin{paragraf*}
These two simple observations produce an immense speed up on average.
The advantage of \itmref{itm:N'}{par:proposed} over computing an $R$-generating set for the
Rees ideal is that the former uses only linear algebraic routines.
The advantage of computing only the needed, rather than all minors, is clear.
\end{paragraf*}

\begin{lemma}
\label{lemma:C1}
Let
\[
	N'=N'_i=(~N_1~|~\cdots~|~N_i~)
\]
be the partial matrix of syzygies up to degree $i$.
If $N'$ satisfies the following condition
\begin{align}
	\label{C:OK-matrix}
	\tag{C1}
	\minors(r,N')\neq0 \text{ and } \deg(\gcd(\minors(r,N')))=\deg(P)
\end{align}
then
\[
\gcd(\minors(r,N'))=P
\]
\end{lemma}

\begin{proof}
By Lemma \ref{check}, we have that $\det(M)\in P$ for any square matrix of syzygies,
so for every minors of $N'$ in particular.
It follows that the $\gcd$ of the minors is in $P$.
The sufficiency follows since $P$ is principal of the same degree as the $\gcd$.
\end{proof}

% \begin{absolutelynopagebreak}
\begin{algorithm}{\sc Proposed Algorithm}{\bf.}
\label{algo:proposed}
\begin{algorithmic}
\State{{\bf input:} none}
\State{{\bf output:} the implicit equation $P$}
\State{Set $r=h^0(S_\mathbf d)$}
\State{Set $N'=r\times0$ matrix over $T$}
\While{\ref{C:OK-matrix} is not satisfied for $N'$}
	\State{Given $N_1,\ldots,N_{i-1}$, use Algorithm \ref{algo:compute-Ni} to compute $N_i$}
	% \Comment{Call Recursion again}
	\State{Set $N'=N'~|~N_i$}
\EndWhile
\State{Report $P=\gcd(\minors(r, N'))$}
\end{algorithmic}
\end{algorithm}
% \end{absolutelynopagebreak}

% \begin{absolutelynopagebreak}
\begin{algorithm}{\sc ComputePartialSyzyzies}{\bf.}
\label{algo:compute-Ni}
\begin{algorithmic}
\State{{\bf input:} a list of matrices of sygygy-generators $N_1,\ldots,N_{i-1}$}
\State{{\bf output:} the syzygy-generators matrix $N_i$}
\For{$0<j<i$}
%	\State{Let $\mt{basis}(T)$ be the basis for $T_{i-j}$ as a row vector}
	\State{Set $N_{ji}=\mt{basis}(T_{i-j})\tensor N_{j}$}
	\State{Set $K_{ji}$ to be the linearization of $N_{ji}$}
\EndFor
\State{Set $K_i=\ker(\Phi^{(i)})$}
\State{Let $K_i'$ be such that $\Span(K_i)=\Span(K_i')\oplus(\sum_j\Span(K_{ji}))$}
\State{Let $N_i$ be such that $\mt{basis}(R_{\mathbf d,i})\cdot K_i'=\mt{basis}(S_{\mathbf d})\cdot N_i$}
\State{Report $N_i$}
\end{algorithmic}
\end{algorithm}
% \end{absolutelynopagebreak}



\section{Examples}

\begin{example}[$\mt{ex601}$]
\label{ex601}
We use the ideas of the previous section to compute the implicit equation
of a rational map given by 5 generic $(2,1,1)$-forms on
\[
	X=\PP^1_{s_0,s_1}\times\PP^1_{t_0,t_1}\times\PP^1_{u_0,u_1}
\]
The base locus is empty is empty,
so by formulas \eqref{eq:self-intersection} and \eqref{eq:degree-formula},
the degree of the image is
\[
	3!\times2\times1\times1=12
\]

Using $\d=(1,1,1)$,
our method computed the partial matrix $N'$ in a little more than a 1 second.
The standard Gr\"obner basis computation took more than 6 minutes.
The details follow.

\begin{verbatim}
i1 : loadPackage "ImplicitizationAlgos"

o1 = ImplicitizationAlgos

o1 : Package

i2 : KK=ZZ/32009;

i3 : S=KK[s_0,s_1,t_0,t_1,u_0,u_1,
          Degrees=>{2:{1,0,0},2:{0,1,0},2:{0,0,1}}];

i4 : T=KK[x_0..x_4];

i5 : B=super basis({2,1,1},S);

             1       12
o5 : Matrix S  <--- S

i6 : J=ideal(B*random(S^12,S^5));

o6 : Ideal of S

i7 : R=KK[s_0,s_1,t_0,t_1,u_0,u_1,x_0..x_4,
          Degrees=>{2:{1,0,0,0},2:{0,1,0,0},2:{0,0,1,0},5:{0,0,0,1}}];

i8 : F=sub(gens J,R);

             1       5
o8 : Matrix R  <--- R

i9 : d={1,1,1};

i10 : N1=ComputeNi(F,1,d,{});

              8       4
o10 : Matrix R  <--- R

i11 : time N2=ComputeNi(F,2,d,{N1});
     -- used 0.155756 seconds

              8       4
o11 : Matrix R  <--- R
\end{verbatim}

At this point, the size of $N'=(~N_1~|~N_2~)$ is equal or $r=12$,
and \eqref{C:OK-matrix} holds.
We will see in Chapter~\ref{ch:koszul-bpf} that in this setting $N$ is square,
so in fact $N=N'$.

\begin{verbatim}
i12 : time ComputeNi(F,2,d,{});
     -- used 0.155017 seconds

              8       24
o12 : Matrix R  <--- R

i13 : time N3=ComputeNi(F,3,d,{N1,N2});
     -- used 1.94713 seconds

              8
o13 : Matrix R  <--- 0

i14 : time ComputeNi(F,3,d,{});
     -- used 2.00069 seconds

              8       80
o14 : Matrix R  <--- R
\end{verbatim}

While not relevant to our computation, the last 3 commands give us more information
about the syzygies over $(1,1,1)$.
There are 24 linearly independent quadratic syzygies,
4 of which arise from quadratic $R$-generators, so 20 must be coming from linear generators.
Since there are only 4 linear generators, all the quadratic syzygies they give rise to
remain linearly independent to among themselves and the new.

\begin{verbatim}
i15 : N'=sub(N1|N2,T);

              8       8
o15 : Matrix T  <--- T

i16 : time N''=ComputeNConj(F,12,d);
     -- used 0.301989 seconds

              8       8
o16 : Matrix R  <--- R

i17 : N''=sub(N'',T);

              8       8
o17 : Matrix T  <--- T

i18 : image N'==image N''

o18 = true
\end{verbatim}

Finally, we compute the the implicit equation using $N'$ and directly.

\begin{verbatim}
i19 : time P'=ideal det N';
     -- used 6.53631 seconds

o19 : Ideal of T

i20 : time P=ker map(S,T,J_*);
     -- used 367.429 seconds

i21 : P==P'

o21 = true
\end{verbatim}
\end{example}

\begin{example}[$\mt{ex602}$]
\label{ex602}
We compute the implicit equation of five general $(2,2,1)$-forms over $(\PP^1)^3$.
The base locus is empty, so the degree of the equation is $24$.
We find in the form of a determinant of an $18\times18$-matrix of quadratic forms.

\begin{verbatim}
i1 : loadPackage "ImplicitizationAlgos"

o1 = ImplicitizationAlgos

o1 : Package

i2 : KK=ZZ/32009;

i3 : S=KK[s_0,s_1,t_0,t_1,u_0,u_1,
          Degrees=>{2:{1,0,0},2:{0,1,0},2:{0,0,1}}];

i4 : T=KK[x_0..x_4];

i5 : B=super basis({2,2,1},S);

             1       18
o5 : Matrix S  <--- S

i6 : J=ideal(B*random(S^18,S^5));

o6 : Ideal of S

i7 : R=KK[s_0,s_1,t_0,t_1,u_0,u_1,x_0..x_4,
          Degrees=>{2:{1,0,0,0},2:{0,1,0,0},2:{0,0,1,0},5:{0,0,0,1}}];

i8 : F=sub(gens J,R);

             1       5
o8 : Matrix R  <--- R

i9 : d={2,1,1};
\end{verbatim}
We pick $\d=(2,1,1)$.

\begin{verbatim}
i10 : time N1=ComputeNi(F,1,d,{});
     -- used 0.024571 seconds

              12
o10 : Matrix R   <--- 0

i11 : time N2=ComputeNi(F,2,d,{});
     -- used 0.607232 seconds

              12       12
o11 : Matrix R   <--- R

i12 : time N3=ComputeNi(F,3,d,{N2});
     -- used 10.7996 seconds

              12       60
o12 : Matrix R   <--- R

i13 : time N'=ComputeNConj(F,24,d);
     -- used 1.49427 seconds

              12       12
o13 : Matrix R   <--- R
\end{verbatim}
We have that $N_1=0$ and $N_2$ is square.
Note the similarity to the situation of Corollary \ref{thm:rel-moving-quadrics}.
\end{example}

% \section{First Example}
% \label{sec:example1}
%
% As a first showcase example, we carry out the computation of a general degree-$26$
% implicit equation with a preset baselocus.
%
% Let $X=\PP^1_{s,u}\times\PP^1_{t,v}$ and $Q=\la s^3,s^2t,t^2\ra$.
% Let $\m=\la st,sv,ut,uv\ra$ be the irrelevant ideal of $X$.
% Take $J$ to be an ideal generated by $4$ general biquartics ${\phi_0,\ldots,\phi_3}$ in $Q$
% and set
% \[
% \phi=(\phi_0,\ldots,\phi_3):\PPP\To\PP^3
% \]
%
% Note that here {\em general} refers to a choice of $\Span_\CC\{\phi_0,\ldots,\phi_3\}$ as an
% element in
% \[
% \text{Gr}(4,\mt{basis}(Q_{4,4}))\cong\mathbb{G}(3,19)
% \]
% with respect to the resulting structure of $I$---see Theorem \ref{1} for a precise statement.
% Over a finite field, however, this is equivalent to a
% {\em uniformly random} choice of the $80$ coefficients.
%
% A possible realization in Macaulay2 follows.
%
% \begin{verbatim}
% i1 : KK=ZZ/32009;
%
% i2 : S=KK[s,u,t,v,Degrees=>{2:{1,0},2:{0,1}}];
%
% i3 : T=KK[x_0..x_3];
%
% i4 : Q=ideal(s^3,t^2,s^2*t);
%
% o4 : Ideal of S
%
% i5 : mm=ideal(s,u)*ideal(t,v);
%
% o5 : Ideal of S
%
% i6 : super basis({4,4},Q);
%
%              1       20
% o6 : Matrix S  <--- S
%
% i7 : J=ideal(o6*random(S^20,S^4));
%
% o7 : Ideal of S
%
% \end{verbatim}
%
% Since $Q$ is saturated with respect to $\m$ and the saturation of $J$ is $Q$,
% scheme-theoretically, $\phi$'s baselocus is just $V(Q)$.
% The latter is supported on a single point $q=(0,1)\times(0,1)$
% of degree $5$ and multiplicity $6$.
%
% \begin{verbatim}
% i8 : decompose Q
%
% o8 = {ideal (s, t)}
%
% o8 : List
%
% i9 : saturate(Q,mm)==Q
%
% o9 = true
%
% i10 : saturate(J,mm)==Q
%
% o10 = true
%
% i11 : multiplicity Q
%
% o11 = 6
%
% i12 : degree Q
%
% o12 = 5
%
% \end{verbatim}
%
% We refer the reader to Section \ref{sec:multiplicity} for an explanation why
% Macaulay2's routines compute the correct values (in biprojective space) and
% how to compute them by hand in the case of monomial baselocus.
%
% We continue by setting up the ambient ring $R$.
%
% \begin{verbatim}
% i13 : R=KK[s,u,t,v,x_0..x_3,Degrees=>{2:{1,0,0},2:{0,1,0},4:{0,0,1}}];
%
% i14 : J=sub(J, R);
%
% o14 : Ideal of R
%
% \end{verbatim}
%
% Fix $\mathbf d=(3,3)$.
% Because $\delta=2$ in this case, we can compute the $N_{ji}$ matrices directly.
% Here is an alternative calculation to the one described in the previous section.
%
% We start by computing $N_1$.
%
% \begin{verbatim}
% i15 : subs=toList((0..3)/(i->x_i=>J_i));
%
% i16 : time G1=sub(super basis({3,3,1},R),subs)//(super basis({7,7,0}, R));
%      -- used 0.013008 seconds
%
%               64       64
% o16 : Matrix R   <--- R
%
% i17 : time K1=matrix entries gens ker G1;
%      -- used 0.014648 seconds
%
%               64       5
% o17 : Matrix R   <--- R
%
% i18 : time N1=(super basis({3,3,1},R))*K1//(super basis({3,3,0},R));
%      -- used 0.000439 seconds
%
%               16       5
% o18 : Matrix R   <--- R
%
% \end{verbatim}
%
% At this point $N'=N_1$.
% Because there are only $5$ linear syzygies, $N'$ cannot satisfy \ref{C:OK-matrix},
% so we continue.
%
% \begin{verbatim}
% i19 : time G2=sub(super basis({3,3,2},R),subs)//(super basis({11,11,0}, R));
%      -- used 0.19233 seconds
%
%               144       160
% o19 : Matrix R    <--- R
%
% i20 : time K22=matrix entries gens ker G2;
%      -- used 0.177021 seconds
%
%               160       32
% o20 : Matrix R    <--- R
%
% i21 : time K12=(((super basis({3,3,1},R))*K1)**(super basis({0,0,1}, R)))//(super basis({3,3,2},R));
%      -- used 0.009606 seconds
%
%               160       20
% o21 : Matrix R    <--- R
%
% i22 : time K2=gens trim image (K22%K12);
%      -- used 0.009801 seconds
%
%               160       12
% o22 : Matrix R    <--- R
%
% i23 : time N2=(super basis({3,3,2},R))*K2//(super basis({3,3,0},R));
%      -- used 0.007547 seconds
%
%               16       12
% o23 : Matrix R   <--- R
%
% i24 : time N=N1|N2;
%      -- used 0.000043 seconds
%
%               16       17
% o24 : Matrix R   <--- R
%
% \end{verbatim}
%
% We now have a partial matrix $N'=N_1~|~N_2$ with $5$ linear and $12$ quadratic columns.
% This gives more columns than rows and, furthermore, any maximal minor would have degree
% 27 or 28, depending on whether it involves all the linear columns or not.
%
% Because we actually have $N=N'$ in this case, $N'$ would certainly satisfy \ref{C:OK-matrix}.
% A probabilistic test for $N'$ follows.
%
% \begin{verbatim}
% i25 : time Nf=sub(N, subs);
%      -- used 0.447607 seconds
%
%               16       17
% o25 : Matrix R   <--- R
%
% i26 : time rank N
%      -- used 0.007021 seconds
%
% o26 = 16
%
% i27 : time rank Nf
%      -- used 0.910507 seconds
%
% o27 = 15
%
% \end{verbatim}
%
% This shows us that the ideal of maximal minors of $N$ is nonzero
% and that every minor is a multiple of the implicit equation.
% We randomly pick two minors of degree 27 and check if \ref{C:OK-matrix}
% is satisfied.
%
% \begin{verbatim}
% i28 : cols1=sort join({0,1,2,3,4},RandPerm(5,16,11))
%
% o28 = {0, 1, 2, 3, 4, 5, 6, 7, 8, 9, 10, 11, 12, 13, 14, 15}
%
% o28 : List
%
% i29 : time rank (N_cols1)
%      -- used 0.00428 seconds
%
% o29 = 16
%
% i30 : time rank (Nf_cols1)
%      -- used 5.32849 seconds
%
% o30 = 15
%
% i31 : cols2=sort join({0,1,2,3,4},RandPerm(5,16,11))
%
% o31 = {0, 1, 2, 3, 4, 5, 6, 7, 8, 10, 11, 12, 13, 14, 15, 16}
%
% o31 : List
%
% i32 : cols1==cols2
%
% o32 = false
%
% i33 : time rank (N_cols2)
%      -- used 0.005311 seconds
%
% o33 = 16
%
% i34 : time rank (Nf_cols2)
%      -- used 5.15664 seconds
%
% o34 = 15
%
% \end{verbatim}
%
% We selected 11 random columns out of the 12 quadratic syzygies and added them
% to the list of linear syzygies.
% The (0-indexed) columns represented by the lists $cols_1$ and $cols_2$ correspond
% to nonzero minors vanishing on the implicit equation.
% Because both are of degree 27 and the equation of degree 26,
% those are going to be linear multiples of $P(\mathbf x)$.
%
% We have two possibilities for finding the equation. The first is direct computation.
%
% \begin{verbatim}
% i35 : time D1=det (N_cols1);
%      -- used 475.552 seconds
%
% i36 : time D2=det (N_cols2);
%      -- used 485.142 seconds
%
% i37 : time P=gcd (D1,D2);
%      -- used 0.19819 seconds
%
% \end{verbatim}
%
% Note that each of the determinants took about 8 minutes to get
% while the $\gcd$ --- under 20 seconds.
% Computing determinants symbolically is another routine that is
% readily parallelizibe but very time consuming in general.
%
% For a comparison the following -- the machine, the time, etc, etc
%
% [Say why this is needed and that after a general change of coordinates...
% the latter is part of the algorithm.]
%
% [Sometimes inspecting $N_{cols_1}$ we are able to find a $15\times15$ minor
% which would then be the implicit equation.]
%
% \begin{paragraf*}
% Sometimes we can only give the answer as the gcd of two minors. Next example
% explores this situation.
% \end{paragraf*}

% \begin{paragraf}
% A few remarks are in order.
% We prove the correctness of Algorithm \ref{algo:compute-N'} below and provide a complete realization
% in Macaulay2 in Implementation \ref{algo:fast-method-impl-M2}.
% For example, finding a matrix ``such that'' is really finding a coefficient matrix for some
% vectors given some basis.
% \end{paragraf}
%
% \begin{theorem}
% \label{thm:algo-proof-of-correctness}
% Let $\phi:X\To\PP^n$ be as before. Suppose that $\mathbf d$ is in the regularity of $\Proj(B_P)$.
% Then Algorithm \ref{algo:compute-N'} correctly computes $P^{\deg\phi}$
% in at most $\delta$ steps, where $\delta$ is the maximal degree showing up in a minimal set of homogeneous
% generators for $I_{\mathbf d,\bullet}$.
% \end{theorem}
%
% \begin{proof}
% Given that \ref{C:OK-matrix} is suffient, we only need to show $N'$ is going to
% satisfy it after finitely many steps.
% But $N'=N$ at step $\delta$ and $N$ satisfies the condition by Theorem \ref{thm:gcd-minors}.
% It follows that the matrix $N'$ is going to satisfy \ref{C:OK-matrix} after at most $\delta$ iterations,
% in which case the $\gcd$ of the minors is $P$.
%
% Remark: any nonzero minor is divisible by $P^{\deg\phi}$ by the main theorem.
% \end{proof}
%
% \begin{paragraf}
% As presented, the algorithm has an obvious bottle neck---the number of minors to be computed is {\em very}
% large. For example, [show the example of the $\binom{72}{36}$ and use Stirling numbers to say how many matrices there are].
% A probabilistic remedy is the following.
% Make a generic change coordinates on the column-space of $N$.
% Now pick minors as along as they are nonzero (test rank) but vanish on the $\phi_j$ (test rank after substitute).
% Now you expect every next minor to knock the degree of the $\gcd$ down by at least one.
% Stop when the degree become $\deg(\phi)\deg(Y)$.
% This method obviously terminates---eventually all the minors are taken---but generally {\em much} quicker than
% computing all the minors.
% It is worth pointing out that the $\gcd$ computation is fast, computing determinants symbolically isn't.
% \end{paragraf}
%
% \begin{paragraf}
% We now present the final version of our algorithm using the probabilistic speed-ups described above.
% Because the algorithm's pseudo code closely follows typical computer algebra system's routines,
% we close the section with a realization in Macaulay2.
% In the next two sections, we explore the positives of even further, showing how it is sometimes
% possible to avoid even the determinant computations.
% \end{paragraf}


%% STRIP BEGIN
\end{document}