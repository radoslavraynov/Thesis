\documentclass[fleqn,reqno]{amsart}
\usepackage{../lib/radoslav-macro}
\usepackage{../lib/radoslav-more}
\usepackage{times}
\usepackage{natbib}

\newcounter{chapter}
\setcounter{chapter}{6}
\numberwithin{first}{chapter}

\begin{document}
%% STRIP END

\begin{paragraf*}
This chapter is devoted to a description of a method for fast implicitization using
the theory developed in Chapters \ref{ch:main-results} and \ref{ch:main-proofs}.
We present our main algorithm in Section \ref{sec:algorithm} after we discuss
the problems arising from the naive algorithm
which we used for the examples of Chapter \ref{ch:examples}.
In Section \ref{sec:example1} and \ref{sec:example2},
we employ the algorithm and its implementation to compute examples which are unfeasible
to attack using the standard methods.

While some details on the Macaulay2 implementations follow the pseudo code presented,
the code is available at
\begin{center}
	\href{http://www.math.cornell.edu/~rzlatev/phd-thesis/algos}{\ttten http://www.math.cornell.edu/$\sim$rzlatev/phd-thesis/algos}
\end{center}
\end{paragraf*}

\begin{paragraf*}
We continue to assume the notation of Chapter \ref{ch:preliminaries}
and the setup of Chapter \ref{ch:main-results}.
\end{paragraf*}



\section{Algorithm}
\label{sec:algorithm}

\begin{paragraf*}
At first glance, an algorithm for finding the implicit equation is contained in the proof
of the our main theorem, Theorem \ref{th:gcd-minors}.
It consist of finding a $T$-module generating set for~$I_{\d,\bullet}$,
which can be done using Lemma \ref{lemma:basis-Id}.
In fact, this is exactly what used to construct a naive algorithm for finding $N$
in all explicit calculations so far.
The pseudo code for the algorithm and its implementation follow.
\end{paragraf*}

% \begin{absolutelynopagebreak}
\begin{algorithm} {\sc Naive Algorithm\bf.}
\label{algo:naive-algorithm}
\begin{algorithmic}
  \State Set $r=\dim_\CC(S_{\mathbf d})$
  % \State Compute a generating set $\{h_l:l\}$ for the Rees ideal $I$
  \State Compute a generating set $\{g_k:k\}$ for $I_{\d,\bullet}$
  	using Lemma \ref{lemma:basis-Id}
  \State Set $N$ to be the coefficient matrix of the $g_k$ with respect to $\mt{basis}(S_\d)$
  \State Compute $P=\gcd(\mt{minors}(r,N))$
\end{algorithmic}
\end{algorithm}
% \end{absolutelynopagebreak}

\begin{implementation}
We first need an implementation of Lemma \ref{lemma:basis-Id}.
It takes an $R$-module $M$ and degree $\mathbf{a}$ on $S$,
and tries to produces the linearly independent elements in $M_{\mathbf a,\bullet}$
each generator gives rise to.
\label{code:naive-algorithm}
\begin{verbatim}
procPushGensHigher = (a,r,M) -> (
  G := for g in M_* list (
    if all(a-((degree g)_r), Z->Z>=0) then
      basis((a-((degree g)_r))|{0},ring M)**g
    else continue);
  trim image fold(G,matrix {{0_(ring M)}},(a, b)->a|b)
)
\end{verbatim}
The actual implementation of Algorithm \ref{algo:naive-algorithm} is pretty straight-forward.
We only note that the ambient ring $R$ is required as input for encapsulation.
\begin{verbatim}
ComputeNRees = method ()
ComputeNRees (Ideal, List, Ring) := Matrix => (J, d, R) -> (
  x := symbol x;
  I := reesIdeal(J, Variable=>x);
  AI := ring I;
  zm := 0*d;
  g := map(R,AI,first entries super basis(zm|{1},R));
  I = g(I);
  r := toList(0..#(degree J_0)-1);
  V := procPushGensHigher (d,r,I);
  matrix entries ( (gens V) // basis(d|{0}, R) )
)
\end{verbatim}
\end{implementation}

\begin{paragraf}
\label{par:flaws}
The flaws of this algorithm as a method for fast implicitization are fatal.
\begin{enumerate}
\item
\label{itm:flaws:GB}
Line 3 of the algorithm refers to Lemma \ref{lemma:basis-Id} but unless we were given
an $R$-generating set for the Rees ideal,
we have to compute it from scratch.
This is at least as hard a computation as the computation of the implicit equation
in a very strict sense ---
the Rees ideal contains the implicit equation as its only generator in bidegree $(0,\bullet)$.
This follows from Proposition \ref{prop:deg-GB} and is shown in Example \ref{ex314}.
If we had a way to efficiently compute this, however,
then we would only need an algorithm find the unique generator in the aforementioned degree.

\item
\label{itm:flaws:many-minors}
In line 5, while computing the $\gcd$ of polynomials is fast,
the number of minors to be computed can be very large.
This happens even for reasonably small examples ---
the one described in Section \ref{sec:example2} finds a matrix $N$ of size $18\times50$.
The number of minors to compute is
\[
	\binom{50}{18}=18053528883775
\]
so even if it took the unrealistic $0.001$ seconds to compute each minor,
a single machine would require 572 years to compute them all.

\item
\label{itm:flaws:large-det}
It gets worse.
Coming back to the example referenced in \ref{itm:flaws:many-minors},
we note that each minor is a determinant of order $18$, over 5 variables,
and of total degree $72$.
Computing large determinants symbolically is time-consuming.
The calculation of the determinant of a $13\times13$-matrix of quartic forms
over 5 variables took about an hour to finish.
Scaling this, we can expect this calculation to take around
\[
	14\times15\times16\times17\times18=1028160
\]
hours or about $714$ days.

\item
\label{itm:flaws:large-poly}
Finally, suppose we have found the polynomial in question --- by whatever means.
It is a form of degree $48$ over 5 variables, and very likely dense in those monomials.
This suggests that the polynomial will be represented by
\[
	\binom{53}{5}=2869685
\]
coefficients.
\end{enumerate}
\end{paragraf}

\begin{paragraf*}
Regrettably,
\itmref{itm:flaws:large-det}{par:flaws} would be an issue for any algorithm
relying on computing determinants of matrices of syzygies,
while \itmref{itm:flaws:large-poly}{par:flaws}
would an issue for any implicitization algorithm whatsoever.
However, this can be used as an argument {\em for}
using syzygy matrices in place of the implicit equation altogether.
The examples of Sections \ref{sec:example1} and \ref{sec:example2} further explore this theme.

In addition, we note that issue \itmref{itm:flaws:many-minors}{par:flaws}
could and should be attacked by parallel computing in addition to whatever method is used.
\end{paragraf*}


FINISH THIS TODAY!


\newpage
[also mention that when $d=e$ this is just solving a system of linear equations for each of the coefficients
of $P(\mathbf x)$. See \citet{Wang2004} for an article devoted to this idea.]

\begin{paragraf*}
This chapter describes an algorithm for computing the closed image in the running setup.
While somewhat similar algorithms have been proposed along the lines of
the approximation complex (see [ref]),
our algorithm utilizes the theory developed in Charpter \ref{chap:main-results}
and allows for a general treatement regardless of the structure of the basepoints.
Compared to the Gr\"obner basis calculations and even more naive approaches
such as direct linear algebra calculations of the coefficients [],
the method presented in Section \ref{sec:algorithm} has two advantages:
\begin{enumerate}
\item
its bottleneck parallelizes directly and, in fact, can be tackled separately;
\item
it is not a blackbox in terms of the geometry, which allows for a probabilistic speedups;
\end{enumerate}

In Section \ref{sec:first-example} we study a concrete example in which the second
property manifests.
This allows us to compute an image of degree 26 in 20 minutes as oppesed to 5 hours
using Gr\"obner bases computations.
In Section \ref{sec:second-example} we come back the failing example from
Section \ref{sec:failing-numerics-P4} and compute its degree-48 image.
\end{paragraf*}

\begin{paragraf}
We continue to assume the setup of \ref{ch:preliminaries}.
We work over a fixed degree $\d$ in the regularity of $B_P$.
To avoid distraction, we assume throughout this chapter that $\phi$ is birational.
Most of what is said here holds for maps of degree greater than 1 essentially by
substituting $P^{\deg\phi}$ for $P$.
\end{paragraf}

\begin{paragraf}
At first glance, it seems that an algorithm for the computation of the image of $\phi$ can be
extracted from our main result, Theorem \ref{thm:gcd-minors}, and in fact,
is already outlined in the examples of Chapter \ref{ch:examples}.
In its simplest form, it becomes
\end{paragraf}


\begin{paragraf}
\label{par:problem-computing-rees-gb}
In reality, unless we figure out how to compute $N$ efficiently,
the algorithm boils down to computing a Gr\"obner basis again.
In fact, one that is much more complex ---
we need to find a Gro\"{o}bner basis for the Rees ideal $I$,
which already contains $P(\mathbf x)$ as the unique generator in degree $(0,\deg(Y))$.
So while Algorithm \ref{algo:simplest} may be useful when studing
the relation between the baselocus and the geometry of the image,
it is rather useless as means of implicitization.
\end{paragraf}

\begin{paragraf}
\label{par:problem-too-many-minors}
Another problem with Algorithm \ref{algo:simplest} is the computation of the minors.
While computing the $\gcd$ of polynomials is fast,
symbolicly computing determinants of large order is very time consuming even for a small number of variables.
\end{paragraf}

\begin{paragraf}
% Loosely speaking, we shall modify the algorithm in such a way that
Recall that
\[
	N=N_1~|~\cdots~|~N_\delta
\]
where $N_i$ is the submatrix of $N$ whose columns are degree-$i$ syzygies.
Loosely speaking, our plan is the following:
\begin{enumerate}
\item
instead of computing the whole matrix $N$, we compute the $N_i$'s and keep track
of a partial syzygy matrix $N'$;
\item
instead of computing all the minors, we only compute as many as needed
to determine the $\gcd$ correctly;
\end{enumerate}

These two simple observations produce an immense speed up on average.
In the case of the example described in the previous paragraph,
we only need to compute 2 minors of $N_4$ [are you sure this is not all of $N$?]

Another feature of the algorithm that we are not going to pursue is

\begin{enumerate}
\setcounter{enumi}{2}
\item
\label{superhui}
the setup should allow for an easy parallelization
\end{enumerate}
\end{paragraf}

\begin{lemma}
\label{lemma:C1}
Let
\[
	N'=N'_i=N_1~|~\cdots~|~N_i
\]
be the partial matrix of syzygies up to degree $i$.
If $N'$ satisfies the following condition
\begin{align}
	\label{C:OK-matrix}
	\tag{C1}
	\minors(r,N')\neq0 \text{ and } \deg(\gcd(\minors(r,N')))=\deg(Y)
\end{align}
then
\[
\gcd(\minors(r,N'))=P
\]
\end{lemma}

\begin{proof}
By Lemma \ref{check}, we have that $\det(M)\in P$ for any square matrix of syzygies,
so for every minors of $N'$ in particular.
It follows that the $\gcd$ of the minors is in $P$.
The sufficiency follows since $P$ is principal of the same degree as the $\gcd$.
\end{proof}

\begin{paragraf}
After we detail the outline above, we prove correctness of the algorithm and
then discuss some of its strong and weak points.
\end{paragraf}

% \begin{absolutelynopagebreak}
\begin{algorithm}{\sc Main -- First Version}{\bf.}
\label{algo:compute-N'}
\begin{algorithmic}
\State{{\bf input:} none}
\State{{\bf output:} the implicit equation $P$}
\State{Set $r=h^0(S_\mathbf d)$}
\State{Set $N'=r\times0$ matrix over $T$}
\While{\ref{C:OK-matrix} is not satisfied for $N'$}
	\State{Given $N_1,\ldots,N_{i-1}$, use Algorithm \ref{algo:compute-Ni} to compute $N_i$}
	% \Comment{Call Recursion again}
	\State{Set $N'=N'~|~N_i$}
\EndWhile
\State{Report $P=\gcd(\minors(r, N'))$}
\end{algorithmic}
\end{algorithm}
% \end{absolutelynopagebreak}

% \begin{paragraf}
% To present the subroutine we introduce the following notation.
%
% For any valid $i$, let $G_i$ be a generating set of the syzygies of degree $i$ over $S_{\mathbf d}$.
% In our notation, this is just the list of coordinates of ${\mathbf b}\cdot N_i$ where $\mathbf b$
% is the row-$r$-vector of a vector space basis for $S_{\mathbf d}$.
% For example, $X=\PP^2_{s,t,u}$, $\mathbf d=1$ and
% \[
% N_2=\begin{bmatrix}
% x^2 & z^2\\
% y^2 & 0 \\
% z^2 & xy
% \end{bmatrix}
% \]
% then
% \[
% G_2=\{sx^2+ty^2+uz^2,~sz^2+uxy\}
% \]
%
% Now for any $0< i\leq j$ we set $G_{ij}$ to be the list of $j$-syzygies coming from $i$-syzygies
% in the obvious way. Formally,
% \[
% 	G_{ij}=\bigcup_{|\alpha|=j-i}\bigcup_{g\in G_i}\{{\mathbf x}^\alpha\cdot g(\mathbf s,\mathbf x)\}
% \]
% In the example, above
% \[
% G_{23}=\{
% 	sx^3+txy^2+uxz^2,sxz^2+ux^2y,
% 	sx^2y+ty^3+uyz^2,syz^2+uxy^2,
% 	sx^2z+ty^2z+uz^3,sz^3+uxyz
% \}
% \]
% \end{paragraf}

% \begin{absolutelynopagebreak}
\begin{algorithm}{\sc ComputePartialSyzyzies}{\bf.}
\label{algo:compute-Ni}
\begin{algorithmic}
\State{{\bf input:} a list of matrices of sygygy-generators $N_1,\ldots,N_{i-1}$}
\State{{\bf output:} the syzygy-generators matrix $N_i$}
\For{$0<j<i$}
%	\State{Let $\mt{basis}(T)$ be the basis for $T_{i-j}$ as a row vector}
	\State{Set $N_{ji}=\mt{basis}(T_{i-j})\tensor N_{j}$}
	\State{Set $K_{ji}$ to be the linearization of $N_{ji}$}
\EndFor
\State{Set $K_i=\ker(\Phi^{(i)})$}
\State{Let $K_i'$ be such that $\Span(K_i)=\Span(K_i')\oplus(\sum_j\Span(K_{ji}))$}
\State{Let $N_i$ be such that $\mt{basis}(R_{\mathbf d,i})\cdot K_i'=\mt{basis}(S_{\mathbf d})\cdot N_i$}
\State{Report $N_i$}
\end{algorithmic}
\end{algorithm}
% \end{absolutelynopagebreak}

\begin{paragraf}
A few remarks are in order.
We prove the correctness of Algorithm \ref{algo:compute-N'} below and provide a complete realization
in Macaulay2 in Implementation \ref{algo:fast-method-impl-M2}.
For example, finding a matrix ``such that'' is really finding a coefficient matrix for some
vectors given some basis.
\end{paragraf}

\begin{theorem}
\label{thm:algo-proof-of-correctness}
Let $\phi:X\To\PP^n$ be as before. Suppose that $\mathbf d$ is in the regularity of $\Proj(B_P)$.
Then Algorithm \ref{algo:compute-N'} correctly computes $P^{\deg\phi}$
in at most $\delta$ steps, where $\delta$ is the maximal degree showing up in a minimal set of homogeneous
generators for $I_{\mathbf d,\bullet}$.
\end{theorem}

\begin{proof}
Given that \ref{C:OK-matrix} is suffient, we only need to show $N'$ is going to
satisfy it after finitely many steps.
But $N'=N$ at step $\delta$ and $N$ satisfies the condition by Theorem \ref{thm:gcd-minors}.
It follows that the matrix $N'$ is going to satisfy \ref{C:OK-matrix} after at most $\delta$ iterations,
in which case the $\gcd$ of the minors is $P$.

Remark: any nonzero minor is divisible by $P^{\deg\phi}$ by the main theorem.
\end{proof}

\begin{paragraf}
As presented, the algorithm has an obvious bottle neck---the number of minors to be computed is {\em very}
large. For example, [show the example of the $\binom{72}{36}$ and use Stirling numbers to say how many matrices there are].
A probabilistic remedy is the following.
Make a generic change coordinates on the column-space of $N$.
Now pick minors as along as they are nonzero (test rank) but vanish on the $\phi_j$ (test rank after substitute).
Now you expect every next minor to knock the degree of the $\gcd$ down by at least one.
Stop when the degree become $\deg(\phi)\deg(Y)$.
This method obviously terminates---eventually all the minors are taken---but generally {\em much} quicker than
computing all the minors.
It is worth pointing out that the $\gcd$ computation is fast, computing determinants symbolically isn't.
\end{paragraf}

\begin{paragraf}
We now present the final version of our algorithm using the probabilistic speed-ups described above.
Because the algorithm's pseudo code closely follows typical computer algebra system's routines,
we close the section with a realization in Macaulay2.
In the next two sections, we explore the positives of even further, showing how it is sometimes
possible to avoid even the determinant computations.
\end{paragraf}

\section{Implementation in Macaulay2}

\begin{paragraf}
\label{par:algo-impl-1}
See \href{http://www.math.cornell.edu/~rzlatev/phd-thesis/}{\tt http://www.math.cornell.edu/$\sim$rzlatev/phd-thesis/}.
Should include it here in one way or another or at least mention what I really do computationally.
\end{paragraf}


\section{First Example}
\label{sec:first-example}

As a first showcase example, we carry out the computation of a general degree-$26$
implicit equation with a preset baselocus.

Let $X=\PP^1_{s,u}\times\PP^1_{t,v}$ and $Q=\la s^3,s^2t,t^2\ra$.
Let $\m=\la st,sv,ut,uv\ra$ be the irrelevant ideal of $X$.
Take $J$ to be an ideal generated by $4$ general biquartics ${\phi_0,\ldots,\phi_3}$ in $Q$
and set
\[
\phi=(\phi_0,\ldots,\phi_3):\PPP\To\PP^3
\]

Note that here {\em general} refers to a choice of $\Span_\CC\{\phi_0,\ldots,\phi_3\}$ as an
element in
\[
\text{Gr}(4,\mt{basis}(Q_{4,4}))\cong\mathbb{G}(3,19)
\]
with respect to the resulting structure of $I$---see Theorem \ref{1} for a precise statement.
Over a finite field, however, this is equivalent to a
{\em uniformly random} choice of the $80$ coefficients.

A possible realization in Macaulay2 follows.

\begin{verbatim}
i1 : KK=ZZ/32009;

i2 : S=KK[s,u,t,v,Degrees=>{2:{1,0},2:{0,1}}];

i3 : T=KK[x_0..x_3];

i4 : Q=ideal(s^3,t^2,s^2*t);

o4 : Ideal of S

i5 : mm=ideal(s,u)*ideal(t,v);

o5 : Ideal of S

i6 : super basis({4,4},Q);

             1       20
o6 : Matrix S  <--- S

i7 : J=ideal(o6*random(S^20,S^4));

o7 : Ideal of S

\end{verbatim}

Since $Q$ is saturated with respect to $\m$ and the saturation of $J$ is $Q$,
scheme-theoretically, $\phi$'s baselocus is just $V(Q)$.
The latter is supported on a single point $q=(0,1)\times(0,1)$
of degree $5$ and multiplicity $6$.

\begin{verbatim}
i8 : decompose Q

o8 = {ideal (s, t)}

o8 : List

i9 : saturate(Q,mm)==Q

o9 = true

i10 : saturate(J,mm)==Q

o10 = true

i11 : multiplicity Q

o11 = 6

i12 : degree Q

o12 = 5

\end{verbatim}

We refer the reader to Section \ref{sec:multiplicity} for an explanation why
Macaulay2's routines compute the correct values (in biprojective space) and
how to compute them by hand in the case of monomial baselocus.

We continue by setting up the ambient ring $R$.

\begin{verbatim}
i13 : R=KK[s,u,t,v,x_0..x_3,Degrees=>{2:{1,0,0},2:{0,1,0},4:{0,0,1}}];

i14 : J=sub(J, R);

o14 : Ideal of R

\end{verbatim}

Fix $\mathbf d=(3,3)$.
Because $\delta=2$ in this case, we can compute the $N_{ji}$ matrices directly.
Here is an alternative calculation to the one described in the previous section.

We start by computing $N_1$.

\begin{verbatim}
i15 : subs=toList((0..3)/(i->x_i=>J_i));

i16 : time G1=sub(super basis({3,3,1},R),subs)//(super basis({7,7,0}, R));
     -- used 0.013008 seconds

              64       64
o16 : Matrix R   <--- R

i17 : time K1=matrix entries gens ker G1;
     -- used 0.014648 seconds

              64       5
o17 : Matrix R   <--- R

i18 : time N1=(super basis({3,3,1},R))*K1//(super basis({3,3,0},R));
     -- used 0.000439 seconds

              16       5
o18 : Matrix R   <--- R

\end{verbatim}

At this point $N'=N_1$.
Because there are only $5$ linear syzygies, $N'$ cannot satisfy \ref{C:OK-matrix},
so we continue.

\begin{verbatim}
i19 : time G2=sub(super basis({3,3,2},R),subs)//(super basis({11,11,0}, R));
     -- used 0.19233 seconds

              144       160
o19 : Matrix R    <--- R

i20 : time K22=matrix entries gens ker G2;
     -- used 0.177021 seconds

              160       32
o20 : Matrix R    <--- R

i21 : time K12=(((super basis({3,3,1},R))*K1)**(super basis({0,0,1}, R)))//(super basis({3,3,2},R));
     -- used 0.009606 seconds

              160       20
o21 : Matrix R    <--- R

i22 : time K2=gens trim image (K22%K12);
     -- used 0.009801 seconds

              160       12
o22 : Matrix R    <--- R

i23 : time N2=(super basis({3,3,2},R))*K2//(super basis({3,3,0},R));
     -- used 0.007547 seconds

              16       12
o23 : Matrix R   <--- R

i24 : time N=N1|N2;
     -- used 0.000043 seconds

              16       17
o24 : Matrix R   <--- R

\end{verbatim}

We now have a partial matrix $N'=N_1~|~N_2$ with $5$ linear and $12$ quadratic columns.
This gives more columns than rows and, furthermore, any maximal minor would have degree
27 or 28, depending on whether it involves all the linear columns or not.

Because we actually have $N=N'$ in this case, $N'$ would certainly satisfy \ref{C:OK-matrix}.
A probabilistic test for $N'$ follows.

\begin{verbatim}
i25 : time Nf=sub(N, subs);
     -- used 0.447607 seconds

              16       17
o25 : Matrix R   <--- R

i26 : time rank N
     -- used 0.007021 seconds

o26 = 16

i27 : time rank Nf
     -- used 0.910507 seconds

o27 = 15

\end{verbatim}

This shows us that the ideal of maximal minors of $N$ is nonzero
and that every minor is a multiple of the implicit equation.
We randomly pick two minors of degree 27 and check if \ref{C:OK-matrix}
is satisfied.

\begin{verbatim}
i28 : cols1=sort join({0,1,2,3,4},RandPerm(5,16,11))

o28 = {0, 1, 2, 3, 4, 5, 6, 7, 8, 9, 10, 11, 12, 13, 14, 15}

o28 : List

i29 : time rank (N_cols1)
     -- used 0.00428 seconds

o29 = 16

i30 : time rank (Nf_cols1)
     -- used 5.32849 seconds

o30 = 15

i31 : cols2=sort join({0,1,2,3,4},RandPerm(5,16,11))

o31 = {0, 1, 2, 3, 4, 5, 6, 7, 8, 10, 11, 12, 13, 14, 15, 16}

o31 : List

i32 : cols1==cols2

o32 = false

i33 : time rank (N_cols2)
     -- used 0.005311 seconds

o33 = 16

i34 : time rank (Nf_cols2)
     -- used 5.15664 seconds

o34 = 15

\end{verbatim}

We selected 11 random columns out of the 12 quadratic syzygies and added them
to the list of linear syzygies.
The (0-indexed) columns represented by the lists $cols_1$ and $cols_2$ correspond
to nonzero minors vanishing on the implicit equation.
Because both are of degree 27 and the equation of degree 26,
those are going to be linear multiples of $P(\mathbf x)$.

We have two possibilities for finding the equation. The first is direct computation.

\begin{verbatim}
i35 : time D1=det (N_cols1);
     -- used 475.552 seconds

i36 : time D2=det (N_cols2);
     -- used 485.142 seconds

i37 : time P=gcd (D1,D2);
     -- used 0.19819 seconds

\end{verbatim}

Note that each of the determinants took about 8 minutes to get
while the $\gcd$ --- under 20 seconds.
Computing determinants symbolically is another routine that is
readily parallelizibe but very time consuming in general.

For a comparison the following -- the machine, the time, etc, etc

[Say why this is needed and that after a general change of coordinates...
the latter is part of the algorithm.]

[Sometimes inspecting $N_{cols_1}$ we are able to find a $15\times15$ minor
which would then be the implicit equation.]

\begin{paragraf*}
Sometimes we can only give the answer as the gcd of two minors. Next example
explores this situation.
\end{paragraf*}



\section{Second Example}
\label{sec:second-example}

\begin{paragraf*}
I have a few more examples in this spirit. Specifically, examples 604 and 605 in {\tt ./test}
at \href{https://github.com/radoslavraynov/ImplAlgos}{https://github.com/radoslavraynov/ImplAlgos}.
Those show that wheather or not the numerics workout nicely, one should not study the
syzygies separately, far less so, should one restric to linear and quadratic syzygies.
Some of the matrices turn out to be square but also, what is apparent, is that there is no
single recipe for what $\d$ to be used for a given $\mathbf e$ and some adjustments have to be made
in order to balance the size of the matrices and the degrees of the entries (the degrees of the syzygies).
\end{paragraf*}

\begin{paragraf*}
Include the example where we compute the image of 5 general $(2,2,2)$-forms on $\PPP$.
This example is hard. Here we have a basepoint-free birational map whose image is of degree $48$.
Over $\d=(1,1,1)$ we get $r=8$, so one can hope that something along the lines of the result
over $\PP^2$ and $\PPP$ would work. Admittedly, I hoped that the image would be the determinant
of an $8\times8$ matrix of sextics, and for this to be true, there has to be no syzygies in degree $5$.
Regrettably, things could not be further from the truth --- there are no syzygies in degree up to 8,
and computation for larger degrees would take even longer than the few hours or so I needed on my
machine.

However, in degree $\d=(2,2,1)$, I get, easily or rather, quickly,
an $18\times50$ matrix $N_4$ and $N_1=N_2=N_3=0$. Each of the maximal minors has degree 72 and has
as a factor our degree-48 implicit equation.

Computing even a single determinant is very very hard---it took me an hour to compute a determinant
of a $12\times12$ matrix in 5 variables, so it would probably take something in the magnitude of
$13*14*15*16*17*18$ hours to compute a single $18\times18$ minor. The good news is that we only need two.
In fact, by intersecting with a line in $\PP^4$ we can detect if two minors share more than just
the implicit equation---simply find the multiplicity of the zero-dimensional intersection. If it is
48, the gcd of those 2 is our guy.
This is easy and quick---takes a minute or so.
\end{paragraf*}



%% STRIP BEGIN
\end{document}