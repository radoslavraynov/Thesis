\documentclass[phd,tocprelim]{cornell}
\let\ifpdf\relax

\usepackage{../lib/radoslav-macro}
\usepackage{../lib/radoslav-more}

%if you're having problems with overfull boxes, you may need to increase
%the tolerance to 9999
\tolerance=9999

\renewcommand{\caption}[1]{\singlespacing\hangcaption{#1}\normalspacing}
\renewcommand{\topfraction}{0.85}
\renewcommand{\textfraction}{0.1}
\renewcommand{\floatpagefraction}{0.75}

\title {Examples of Implicitization of Hypersurfaces through Syzygies}
\author {Radoslav Zlatev}
\conferraldate {August}{2015}
\degreefield {Ph.D.}
\copyrightholder {Radoslav Zlatev}
\copyrightyear {2015}

\numberwithin{first}{chapter}
\setenumerate{label=\normalfont(\arabic*),ref=\normalfont\arabic*}

\usepackage{times}
\usepackage{natbib}

\begin{document}

\maketitle
\makecopyright

\pagenumbering{Alph}
\begin{abstract}
Let $X$ be a smooth projective toric variety of dimension~$n-1$
and let $\phi:X\To\PP^n$ be a generically finite rational map.
The closed image~$Y$ can be defined by a single equation $P(\mathbf x)$,
called the implicit equation.
The implicitization problem asks for techniques for finding the implicit equation.

This is an old problem in algebraic geometry, and can be solved effectively
through elimination using Gr\"obner bases.
However, this solution represents
a black box in relation to the geometry on the base locus and the closed image,
and is unfeasible even for reasonably small examples.

In this thesis, we use ideas from the two most popular non-Gr\"obner basis approaches,
the method of the approximation complex and the method of moving surfaces,
to construct a family of matrices $N$, one for each element in $\Pic(X)$,
capturing determinantal representations for $P(\mathbf x)$.
An algorithm for this calculation is described and implemented in the Macaulay2 system.
Example calculations in previously intractable situations are presented.
\end{abstract}

\begin{biosketch}
Radoslav was born in Burgas, on the Bulgarian Black Sea cost.
He attended the High School of Mathematics and the Sciences in his hometown, graduating in May~2005.
He then studied mathematics and computer science at Jacobs University Bremen,
earning a Bachelor of Science degree in June~2008.
He joined the Department of Mathematics at Cornell in August~2010.
\end{biosketch}

\begin{dedication}
To my parents.
\end{dedication}

\begin{acknowledgements}
I want to thank my advisor, Mike Stillman ---
for his guidance, his patience, and his encouragement.
I also want to thank
Irena Peeva for her support and recommendations, and
Allen Knutson and Ed Swartz for valuable comments on the thesis.
Lastly, I want to thank
Hal Schenck at UIUC  for introducing me to the problem
at the Syzygies in Berlin summer school in 2013.
\end{acknowledgements}

\contentspage
% \tablelistpage
% \figurelistpage

\normalspacing \setcounter{page}{1} \pagenumbering{arabic}
\pagestyle{cornell} \addtolength{\parskip}{0.5\baselineskip}

\pagenumbering{arabic}

%% INTRODUCTION
\chapter{Introduction}
\label{ch:intro}
\input{intro-incl}

%% PRELIMINARIES
\chapter{Preliminaries}
\label{ch:preliminaries}
\input{preliminaries-incl}

%% MAIN RESULTS
\chapter{Main Results}
\label{ch:main-results}
\input{main-results-incl}

%% EXAMPLES
\chapter{Examples}
\label{ch:examples}
\input{examples-incl}

%% PROOFS OF THE MAIN RESULTS
\chapter{Proofs of the Main Results}
\label{ch:main-proofs}
\input{main-proofs-incl}

%% A METHOD FOR FAST IMPLICITIZATION
\chapter{A Method for Fast Implicitization}
\label{ch:fast-method}
\input{fast-method-incl}

%% KOSZUL SYZYGIES AND BASEPOINT-FREE MAPS
\chapter{Koszul Syzygies and Basepoint-free Maps}
\label{ch:koszul-bpf}
\input{koszul-bpf-incl}

%% BICUBICS
% \chapter{Bicubics with Basepoints in General Position}
% \label{ch:bicubics}
% \include{bicubics-incl}

%% BIBLIOGRAPHY
\bibliographystyle{unsrtnat}
\bibliography{../lib/refs}

\end{document}