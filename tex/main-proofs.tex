\documentclass[fleqn,reqno]{amsart}
\usepackage{../lib/radoslav-macro}
\usepackage{../lib/radoslav-more}
\usepackage{times}
\usepackage{natbib}


\newcounter{chapter}
\setcounter{chapter}{2}
\numberwithin{first}{chapter}

\begin{document}
%% STRIP END



\begin{paragraf*}
While the ultimate goad of this chapter is to prove the results of Chapter~\ref{ch:main-results},
it is written in a way to help develop intuition about the interplay between matrices of syzygies and
the geometry of $Y$, specifically, of course, its implicit equation.

For this reason we start with a few elementary resuls which are either new or else their proofs are new.
For example, Lemma~\ref{lemma:detM-P} is known [Soderberg and Chen prove it ad-hoc for linear and quadratic forms;
find a good reference]
while Lemma~\ref{lemma:nonzero-detM} is new.

We follow the notation of Chapter~\ref{ch:preliminaries} and adopt setup of Chapter~\ref{ch:main-results}.
\end{paragraf*}

\begin{lemma}
\label{lemma:detM-P}
For any fixed $\d$, let $M$ be a square $r\times r$ matrix of syzygies.
Then
\[
\det(M)\in \la P\ra
\]
In particular, if $M$ is any (not necessarily square) matrix of syzygies, then
\[
\mt{minors}(r,M)\se\la P\ra
\]
\end{lemma}

\begin{proof}
The second statement clearly follows from the first, setting an empty minor to 0.

Let $\text{adj}(M)$ be adjugate matrix and $\mathbf b=\basis(S_\d)$ be the row vector
of the fixed basis for $S_{\d}$.
Then
\[
(\mathbf b\cdot M)\cdot\text{adj}(M)=\mathbf b\cdot\det(M)\mathbf 1_r=\det(M)\mathbf b
\]
Since $\mathbf b\cdot M$ is a row vector of syzygies, the LHS vanishes identically in $S$ under
the substituion $x_0=\phi_0(\mathbf s),\ldots,x_n=\phi_n(\mathbf s)$.
But then
\[
	\text{RHS}|_{x_0=\phi_0,\ldots,x_n=\phi_n}=\det(M)(\phi_0,\ldots,\phi_n)b=
	\begin{bmatrix}0& 0& \ldots& 0\end{bmatrix}
\]
over $S$, and since $\la \mathbf b\ra\se S$ is the irrelevant ideal of a toric variety,
truncated appropriately, it contains a nonzerodivisor on $S$, so
$\det(M)(\phi_0,\ldots,\phi_n)=0$.
If follows that that $\det(M)$ is in the kernel $\la P\ra$ of $\phi^\#$,
proving the first statement.
\end{proof}

\begin{lemma}
\label{lemma:nonzero-detM}
For any fixed $\d$, $N$ has at least as many columns as rows, i.e.
\[
\mu\geq r
\]
and its ideal of maximal minors is nontrivial.
\end{lemma}

\begin{proof}
For each standard basis column vector $e_k\in\CC^r$, $P(\mathbf x)e_k$ is a graded syzygy.
Let $F$ be the $\mu\times r$ matrix of coefficients for getting the $P(\mathbf x)e_k$
out of the generators of the syzygies over $S_\d$, i.e. the columns of $N$.
Putting these together, we find
\[
	N\cdot F=P(\mathbf x)\mathbf 1_r
\]
The sizes of the matrices on the LHS are $r\times\mu$ and $\mu\times r$.
Since the rank of the RHS as a $T$-matrix is $r$, we must have $\mu\geq r$.

The maximal minors are then of size $r\times r$. Since $\rank(N\cdot F)=r$,
also $\rank(N)=r$ and so not all maximal minors vanish.
\end{proof}

\begin{lemma}
\label{lemma:cokerN}	    
For any fixed $\d$, there is an isomorphism of graded $T$-modules
\[
\coker N\cong B_{\d,\bullet}
\]
In particular, if $\mathcal{C}_\nbul$ is any graded resolution of $\coker(N)$ over $T$,
then $H_0\mathcal{C}_\nbul=B_{\d,\bullet}$.
\end{lemma}

\begin{proof}
	This is obvious. We just note that the gradings are off by $\d$.
	%
	% Let $C_k$ be the columns of $N$ and $i_k$ be the degree of the entries of $C_k$,
	% i.e. the degree of the homogeneous $C_k$ in the free $T$-module $T^r$.
	% Let $\mathbf b$ be the $1\times\mu$ vector of the basis of $S_d$,
	% regarded as a graded map $T^r\To R_{d,\bullet}$ by multiplying on the left,
	% and let $\bar{\mathbf b}$ be the natural extension to $B_{d,\bullet}$.
	% Note that $\mathbf b$ and so $\bar{\mathbf b}$ are surjective.
	%
	% To establish the claim we only need to prove that the sequence of graded maps
	% \begin{align}
	% 	\label{exact-seq-coker}
	% 	\bigoplus_k T(-i_k)\xrightarrow{~\mathbf N~} T^r \xrightarrow{~\bar{\mathbf b}~} B_{d,\bullet}\To 0
	% \end{align}
	% is exact. In turn we only need to prove exactness at $T^r$.
	%
	% For any standard basis element $e_k$ of $\CC^r$ thought of as a basis element of the leftmost module above,
	% \[
	% 	(\mathbf b\cdot N\cdot e_k)(\mathbf\phi)=(\mathbf b\cdot C_k)(\mathbf\phi)=0
	% \]
	% by the construction of $N$. This shows that the image is in $I_{d,\bullet}$
	% so the sequence is a complex.
	%
	% Let $G\in\ker \bar{\mathbf b}$ be homogeneous of degree $i$, i.e. $\mathbf b\cdot G(\mathbf\phi)=0$
	% and the entries of $G$ are of degree $i$. Set $g=g(\mathbf s;\mathbf x)=\mathbf b\cdot G\in I_{(d,i)}$.
	% Because $I_{d,\bullet}$ is finite over $T$, generated by $g_k=\mathbf b\cdot C_k$ by construction of $N$,
	% there are forms $f_k$ in $T$
	% of degree $i-i_k$ (setting $f_k=0$ for $i<i_k$) such that
	% \[
	% 	g=\textstyle\sum f_kg_k
	% \]
	% or equivalently,
	% \[
	% 	\mathbf b\cdot G=\textstyle\sum f_k(\mathbf b\cdot C_k)=\mathbf b\cdot(\textstyle\sum f_kC_k)
	% \]
	% in $R_{(d,i)}$. In particular, $G=\sum f_kC_k=N\cdot\begin{bmatrix}f_1\ldots f_\mu\end{bmatrix}\T$
	% is in the image of $N$, proving exactness at $T^r$.
\end{proof}

\begin{remarkhint}
The proof is actually commented out. Once I wrote it down it seemed tautology.
\end{remarkhint}

\begin{lemma}
\label{lemma:anncokerN}
For any fixed $\d$,
\[
\Supp_T(B_{\d,\bullet})=V(P)
\]
\end{lemma}

\begin{proof}
By identifying $T=1\tensor T\se R$, we can think of $T$ as a subring of $R$. If $1\tensor Q(\mathbf x)\in I$ for
any homogeneous $Q(\mathbf x)\in T$, then $Q(\mathbf \phi)=0$, so $Q(\mathbf x)\in\ker\phi^\#=P$. 
Specifically,
\[
	T\cap I=P
\]
as subsets of $R$.
If $Q\notin V(P)$ then $Q^c\cap I\neq\emptyset$, so $B_Q=0$ as a $T$-module
because some element of $P\se I$ is inverted. This shows that $(B_{\d,\bullet})_Q=0$.

Conversely, let $N'$ be the localization of $N$ at $P$.
Then 
\[
	(B_{\d,\bullet})_P=\coker(N)_P=\coker(N')
\]
Since $\Fitt_0\coker(N')=\mt{minors}(r,N')\se PT_P\neq T_P$ by Lemma~\ref{lemma:detM-P},
the cokernel is nonzero, for example, by (\citet{Eis-book-95}, Proposition 20.6).
\end{proof}

\begin{remarkhint}
For a geometric argument, see the proof of Lemma \ref{lemma:lengthRees}.
\end{remarkhint}

\begin{proof}[\bf Proof of Proposition \ref{prop:BP}]
The ideal $I$ is prime in $R$ as the kernel of a ring map into the domain $S[t]$.
By Lemma \ref{lemma:anncokerN},
\[
P=I\cap T\se R
\]
so $B=R/I$ is naturally a $T/P$-algebra generated by the $\mathbf{s}$.
The $T$- and $T/P$-module structure of $B$ conicide and the $T$-module localization
at $P$ coincides with the localization of the ring $R/I$ at the multiplicative set $(T-P)$.
In particular, the localization $B_P$ remains a domain and becomes a $K(T/P)$-algebra
in a natural way.
The remarks about its finite-type and the grading are obvious.
This proves parts (a) and (b).

Let $\mathfrak{n}$ be the irrelevant ideal of $S$
and let $g=g(\mathbf s,\mathbf x)\in I:\mathfrak{n}^k\se R=S[\mathbf x]$ for some $k>0$.
Then $\mathbf s^\alpha g\in I$ for all $|\alpha|=k$ and so
\[
(\mathbf s^\alpha g)(\mathbf\phi)=0
\]
in $S$. Since the latter is a domain, and $X$ is nonempty,
not all the $\mathbf s^\alpha$ are zero and so $g(\mathbf\phi)=0$.
It follows, by the definition of $I$, that $g\in I$.

Noting that saturation commutes with localization finishes the proof of part (c).
Part (d) follow immediately.
\end{proof}

\begin{proof}[\bf Proof of Theorem \ref{thm:rad-minors}]
By Lemma \ref{lemma:anncokerN} and (\citet{Eis-book-95}, Proposition 20.7), we have
\[
\rad(\mt{minors}(r,N))=\rad(\Fitt_0(\coker N))=\rad(\ann (B_{\d,\bullet}))=P\qedhere
\]
\end{proof}

\begin{lemma}
\label{lemma:divdetC}
For any fixed $\d$, let $\mathcal{C}_\nbul$ be a finite graded free resolution of $\coker N$. One has
\[
	\thediv(\det(\mathcal{C}_\nbul))=\length_{T_P}(B_{\d,\bullet})_P\cdot[Y]
\]
as a Weil divisor on $\Proj(T)$.
\end{lemma}

\begin{proof}
By (\citet{GKZ-book-94}, A, Theorem 30), applied to the factorial $T$, a principal prime $Q=\la Q(\mathbf x)\ra$,
and the generically exact $\mathcal{C}_\nbul$,
\[
\ord_{Q(\mathbf x)}(\det(\mathcal{C}_\nbul))=\sum_i (-1)^i \mult_Q(H_i\mathcal{C}_\nbul)
\]
Since $\mathcal{C}_\nbul$ is exact, all the higher homology vanishes, so
\[
\ord_{Q(\mathbf x)}(\det(\mathcal{C}_\nbul))=\mult_Q(H_0\mathcal{C}_\nbul)
\]
which is $0$ outside of $\ann(H_0\mathcal{C}_\nbul)$.
By Lemma \ref{lemma:cokerN} and \ref{lemma:anncokerN} the latter is exactly $P$.
Taking the sum over all non-associate irreducible homogeneous polynomials,
\begin{align*}
\thediv(\det(\mathcal{C}_\nbul))&=\sum_{Q(\mathbf x)} \ord_{Q(\mathbf x)}(\det(\mathcal{C}_\nbul))\cdot[V(Q)]\\
&=\sum_{Q(\mathbf x)} \mult_Q (H_0\mathcal{C}_\nbul)\cdot [V(Q)]\\
&=\mult_P (H_0\mathcal{C}_\nbul)\cdot[Y]\\
&=\length_{T_P}(B_{\d,\bullet})_P\cdot[Y]
\end{align*}
establishing the claim.
\end{proof}

\begin{lemma}
\label{lemma:lengthBdimB}
For any fixed $\d$,
\[
\length_{T_P}(B_{\d,\bullet})_P=\dim_{K(T/P)}(B_{\d,\bullet})_P
\]
\end{lemma}

\begin{proof}
Note that $P(\mathbf x)\in I$, so $P(\mathbf x)$ annihilates $B$ as a $T$-module.
Setting $\m_P=PT_P\se T_P$,
\begin{align*}
\length_{T_P}(B_{\d,\bullet})_P&=
\sum_{k}\dim_{T_P/\m_P} \m_P^{k}(B_{\d,\bullet})_P/\m_P^{k+1}(B_{\d,\bullet})_P\\
&=\dim_{T_P/\m_P}(B_{\d,\bullet})_P
\end{align*}
since $\m_P$ in turn annihilates $(B_{\d,\bullet})_P$.
[A one-line argument is: the $T$- and $T/P$-module structure of $B$ are the same.]
\end{proof}

\begin{lemma}
\label{lemma:lengthRees}
Let $\d\in \reg(B_P)$ in the sense of \eqref{par:describe-d}. One has
\[
\dim_{K(T/P)} (B_{d,\bullet})_P=\deg\phi
\]
\end{lemma}

\begin{remarkhint}
This result requires neither $\codim(Y\se\PP^n)=1$ nor $\dim V(J)=0$.
\end{remarkhint}

\begin{proof}
Let $\Gamma=\Biproj(B)$ be the graph of the rational map $\phi$, or equivalently
the blow-up of $X$ along the basepoints $V(J)$. Let $E\se\Gamma$ be the exceptional
locus.
We point the reader to (\citet{Har-book-77}, II, Example 7.17.3) for the details,
summarized in the following commutative diagram

\begin{tikzcd}[column sep=large]
	\Biproj(B)=\Gamma \arrow[hookleftarrow]{r}\arrow[]{d}{\pi_1}\arrow[two heads, bend left=35]{rrd}[description]{\pi_2} &
		\Gamma-E\arrow[]{d}{\cong}[swap]{\pi_1|_{\Gamma-E}}\arrow[two heads]{rd}[description]{\pi_2|_{\Gamma-E}} \\
	X \arrow[hookleftarrow]{r}\arrow[bend right=17, two heads, dashed]{rr}[description]{\phi} &
		X-V(J) \arrow[two heads]{r}{\phi|_{X-V{J}}} &
		~ Y\se\PP^n
\end{tikzcd}

Since $\pi_2|_{\Gamma-E}=\phi|_{X-V(J)}\circ\pi_1|_{\Gamma-E}$ and $\pi_1|_{\Gamma-E}$ is an isomorphism,
the morphism $\pi_2$ is generically finite onto its image and $\deg(\phi)=\deg(\pi_2)$.
If $\gamma\in\PP^n$ is the generic point of $Y$, then the scheme-theoretic fiber
\[
\pi_2\inv(\gamma)=\Spec(\O_{\gamma,Y})\times_Y \Gamma
\]
is a closed zero-dimensional subscheme of $\Gamma$ consisting of
$\deg(\pi_2)$ points, counted with multiplicity.

The morphism $\pi_2:\Biproj(B)\to\Proj T$ is induced by the graded map of $\CC$-algebras
\[
\pi_2^\sharp=(x_j\mapsto \overline{x_j\tensor1}):T\To{(T\tensor S)}/{I}=B
\]
and the fiber of $\gamma=[P]\in\Proj T$ corresponds to the set of bihomogeneous prime ideals of $B$
which pull back to $P\se T$ via $\pi_2^\sharp$.
By an easy reduction,
% for example (\citet{Vakil-book-10}, Exercise 7.3.H, 7.3.K and 9.3.A),
this set corresponds to the set of homogeneous in the $S$-grading prime ideals
in the finite-type $K(T/P)$-algebra
\[
K(T/P)\tensor_{T/P}B\cong B_P
\]

The identification above presents the fiber as a $\Proj(-)$ over a field,
i.e.
\[
\Proj_{K(T/P)} B_P\xrightarrow{~\sim~}\pi_2\inv(\gamma)\se\Biproj(B)
\]

Since this is a finite projective scheme over the field $K(T/P)$,
its degree is given by its constant Hilbert polynomial, i.e. by
\[
\dim_{K(T/P)} (B_P)_\d=\dim_{K(T/P)} (B_{\d,\bullet})_P
\]
for all $\d$ in the regularity of the projective toric scheme above.

The result follows since by Proposition \ref{prop:BP},
$B_P$ is the Cox ring of the said scheme.
\end{proof}

\begin{example}
The construction of Lemma \ref{lemma:lengthRees} is already apparent in Example \ref{ex207}.
Since $P\se I$ in $R$, the bihomogeneous primes in $R/I=B$
which pull back to $P$ are those in
\[
(\CC[x_0,x_1,x_2]/\la x_0+x_1-x_2\ra)[s,t]/\la x_1s^2-x_0t^2\ra
\]
which intersect the coefficient ring trivially, so in turn, those in
\[
(\CC[x_0,x_1,x_2]\setminus\la x_0+x_1-x_2\ra)\inv(\CC[x_0,x_1,x_2]/\la x_0+x_1-x_2\ra)[s,t]/\la x_1s^2-x_0t^2\ra
\]
\end{example}

\begin{proof}[\bf Proof of Theorem \ref{thm:gcd-minors}]
Fix any such $\d$ and let $\mathcal{C}_\nbul$ be a minimal graded free resolution of $\coker N$.
By (\citet{GKZ-book-94}, A, Theorem 34) which applies since $\mathcal{C}_\nbul$ is exact,
\[
\det(\mathcal{C}_\nbul)=\gcd(\mt{minors}(r,N))
\]
up to a unit of $T$. But then by Lemma \ref{lemma:divdetC}, \ref{lemma:lengthBdimB} and \ref{lemma:lengthRees},
\begin{align*}
\thediv(\gcd(\mt{minors}(r,N)))&=\thediv(\det(\mathcal{C}_\nbul))\\
&=\length_{T_P} (B_{d,\bullet})_P\cdot[Y]\\
&=\dim_{K(T/P)} (B_{d,\bullet})_P\cdot[Y]\\
&=\deg\phi\cdot[Y]
\end{align*}

Because this is just an equality of Weil divisors,
\[
\gcd(\mt{minors}(r,N))=P^{\deg\phi}\qedhere
\]
\end{proof}

\begin{proof}[\bf Proof of Corollary \ref{thm:detM}]
This follows directly from Theorem \ref{thm:gcd-minors}.
\end{proof}

\begin{proof}[\bf Proof of Corollary \ref{cor:approx-complex}]
Recall that the baselocus of $\phi$ is at worst locally complete intersection if and only if
the natural morphism of schemes
\[
\Biproj(\Rees_S(J))\To\Biproj(\Sym_S(J))
\]
is an isomorphism.

We can now follow the proof of Theorem \ref{thm:gcd-minors} verbatim replacing
$N$ by $N_1$ and $B$ by $\Sym_S(J)$. Note that in this case $\d$ must be taken
from the regularity of
\[
(T-P)\inv(\Sym_S(J)/P\Sym_S(J))
\]
as a $K(T/P)$-algebra instead.
\end{proof}



\begin{paragraf*}
We recall the statement of Proposition \ref{prop:degviaGB} [change the number below].
\end{paragraf*}

\begin{proposition}
Let $\CC[s_0,\ldots,s_m]$ be the fixed ambient polynomial ring of $S'$ as described in \ref{par:setup}.
Define the ideal $I_B$ of $\CC[s_0,\ldots,s_m;x_0,\ldots,x_n]$ by the equality
\[
	B=\CC[\mathbf s;\mathbf x]/I_B
\]
Let $>'$ be any product order in which the $\mathbf s$ variables
come before the $\mathbf x$ variables.
Then a reduced Gr\"obner basis for $I_B$ with respect to $>'$ has the form
\begin{align*}
	&g_1(\mathbf s;\mathbf x)=p_1(\mathbf x){\mathbf s}^{\alpha_1}+\text{lower order terms}\\
	&\qquad\cdots\\
	&g_r(\mathbf s;\mathbf x)=p_r(\mathbf x){\mathbf s}^{\alpha_r}+\text{lower order terms}\\
	&g_{r+1}(\mathbf s;\mathbf x)=P(\mathbf x)
\end{align*}
and
\[
	\deg(\phi)=\deg \big(\la{\mathbf s}^{\alpha_1},\ldots,{\mathbf s}^{\alpha_r}\ra\se\CC[\mathbf s]\big)
\]
\end{proposition}

\begin{proof}
For any choice of $>'$, a reduced Gr\"oebner basis for $I_B$ will have the outlined general form
except possibly for the term $g_{r+1}$. By Lemma \ref{SuppRees} and its proof, $I\cap T=P$, so also
$I'\cap T=P$. This establishes $g_{r+1}$ and shows that $\alpha_k\neq0$.

By reduceness, $P(\mathbf x)\nmid p_k(\mathbf x)$, so the initial ideal of the preimage $I'_B$ of $I_P$
in $K(T/P)[\mathbf s]$---formed after factoring
$P$ and inverting $(T\setminus P)$---is generated by the $\mathbf s^{\alpha_k}$.

By Lemma \ref{lemma:lengthRees}, $\deg(\phi)$ is given by the constant Hilbert polynomial of $B_P$,
so we only need to show that the latter coincides with the degree of $I'_B$.
But this is clear since $I'_B$ can be formed by taking any generators of $I_P$
and adding generators for the extension of $J$ to $K(T/P)[\mathbf s]$.
%
% Recall the notation from Lemma \ref{lengthRees}.
% Let $U\se Y$ be the open dense set of closed points whose fibers have cardinality $\deg(\pi_2)$.
% Since the Gr\"obner basis is
% reduced and $P(\mathbf x)$ is irreducible, $P(\mathbf x)\nmid p_k(\mathbf x)~\forall k$.
% It follows that $Y\not\se V(p_1\cdots p_r)$ so we can find a closed point
% \[
% 	\mathbf a\in U-(Y\cap V(p_1\cdots p_r))\se Y
% \]
%
% The fiber $\pi_2\inv(\mathbf a)$, as a subscheme of the ambient $\PP^m_{\mathbf s}\times\PP^n_{\mathbf x}$,
% is definied by the ideal
% \[
% 	\la g_1(\mathbf s;\mathbf a),\ldots,g_r(\mathbf s;\mathbf a),0,x_0-a_0,\ldots,x_n-a_n\ra
% \]
% which in turn corresponds to the subscheme of $\PP^m_{\mathbf s}$ defined by the homogeneous ideal
% \[
% 	\la g_1(\mathbf s;\mathbf a),\ldots,g_r(\mathbf s;\mathbf a)\ra
% \]
%
% By assumption this defines a finite subscheme whose cardinality is $\deg(\pi_2)$.
% As such, $\deg(\pi_2)$ is just the degree of the ideal. Since the initial ideal has the same degree and
% since $\mathbf a\notin V(p_1\cdots p_r)$, we can rescale the $g_k$ by the $p_k({\mathbf a})\in\CC^\times$
% to get that the initial ideal is $\la {\mathbf s}^{\alpha_1},\ldots,{\mathbf s}^{\alpha_r}\ra$.
% Recalling that $\deg(\pi_2)=\deg(\phi)$ establishes the claim.
\end{proof}

\begin{example}
... Give a good example.
\end{example}



\begin{conjecture}
\label{conj:nice-example}
Let $X=\PPP$ and $J\se S=\CC[s,u;t,v]$ be given by 4 general $(2,2)$ forms in $\la s^2,st,t^2\ra$.
Then $\phi$ is generically 1-1 with
the unique basepoint $(0,1)\times(0,1)$ of degree $3$ and multiplicity $4$,
and the image is singular along 3 non-degenerate concurent lines.
\end{conjecture}

\begin{lemma}
Let $L_1,L_2,L_3$ be general...
\end{lemma}

\begin{lemma}[$\mt{ex10}$]
\label{lucky-comp}
Let $P(\mathbf x)$ be a form of degree $4$ on $\PP^3$ and set $Y=V(P(\mathbf x))$.
Suppose that $\mathrm{Sing}(Y)$ contains 3 non-degenerate concurent lines.
Then there is a square $4\times4$ matrix of linear forms $M$ such that $\det(M)=P(\mathbf x)$.
\end{lemma}

\begin{proof}
After a linear change of coordinates on $\PP^3$ we can assume that the lines are the
3 coordinate axes in the distinguished $\{x_3\neq0\}$, i.e. $V(x_0,x_1)$, $V(x_1,x_2)$ and $V(x_0,x_2)$.

On the level of ideals, using Euler's identity, the assumption translates to
\[
\la P_{x_0}(\mathbf x),\ldots, P_{x_3}(\mathbf x) \ra\se\la x_0x_1,x_0x_2,x_1x_2\ra
\]
so writing out $P(\mathbf x)=\sum_{|\alpha|=4} a_\alpha {\mathbf x}^\alpha$
and noting that each of the partials must be zero modulo $\la x_0x_1,x_0x_2,x_1x_2\ra$,
we only need to solve a linear system in the indeterminante coefficients.
We get
\[
P(\mathbf x)=
a_0x_0^2x_1^2+a_1x_0^2x_2^2+a_3x_1^2x_2^2+a_3x_0^2x_1x_2+a_4x_0x_1^2x_2+a_5x_0x_1x_2^2+a_6x_0x_1x_2x_3
\]
which is the determinant of
\[
M=\begin{bmatrix}
x_0&&&x_1\\
&x_1&&x_2\\
&&x_2&x_0\\
-a_2x_2&-a_1x_0&-a_0x_1&a_6x_3+a_3x_0+a_4x_2+a_5x_3
\end{bmatrix}
\]

In fact, if $a_6\neq0$ we can do better.
Setting $a_6=1$, a linear change of coordinates by the matrix
\[
\begin{bmatrix}
1&&&-a_3\\
&1&&-a_4\\
&&1&-a_5\\
&&&1
\end{bmatrix}
\]
leaves the singular locus intact but simplifies the general form to
\[
P'(\mathbf x)=a_0x_0^2x_1^2+a_1x_0^2x_2^2+a_2x_1^2x_2^2+x_0x_1x_2x_3
\]
The matrix $M$ then becomes
\[
M'=\begin{bmatrix}
x_0&&&x_1\\
&x_1&&x_2\\
&&x_2&x_0\\
-a_2x_2&-a_1x_0&-a_0x_1&x_3
\end{bmatrix}
\]
each of whose coordinates is a scaled variable.
\end{proof}

[Say how this is related to taking $\d=(1,1)$, what other methods apply, about the matrix $N_1$
can be taken as a submatrix of this matrix $M$ and how this changes if I precompose the map with
a map $\PP^2\To\PPP$ instead.]



\begin{lemma}
\label{lemma:push-gens}
1
\end{lemma}

\begin{proof}1
\end{proof}



%% STRIP BEGIN
%% BIBLIOGRAPHY
\bibliographystyle{unsrtnat}
\bibliography{../lib/refs}

\end{document}